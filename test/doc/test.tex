\documentclass[english,ngerman,ttfont=roboto]{tudscrmanual}
\ifpdftex{
  \usepackage[T1]{fontenc}
  \input glyphtounicode
  \pdfgentounicode=1
  \usepackage[ngerman=ngerman-x-latest]{hyphsubst}
}{
  \usepackage{fontspec}
}
\lstset{%
  inputencoding=utf8,extendedchars=true,
  literate=%
    {ä}{{\"a}}1 {ö}{{\"o}}1 {ü}{{\"u}}1
    {Ä}{{\"A}}1 {Ö}{{\"O}}1 {Ü}{{\"U}}1
    {~}{{\textasciitilde}}1 {ß}{{\ss}}1
}

\usepackage{widows-and-orphans}

\usepackage{bookmark}
\KOMAoptions{headings=optiontoheadandtoc}

\tracinglabels[all]
%\tracingmarkup
%\tracingbundle
\usepackage{blindtext}

\begin{document}
\begin{Bundle*}{\Package{tudscrsupervisor}}
\begin{Declaration}{\Environment{notice}[\OLParameter{Überschrift}]}
\begin{Declaration}{\Key{\Environment{notice}}{headline=\PSetName{Überschrift}}}
\printdeclarationlist%
%
\end{Declaration}
\end{Declaration}
\end{Bundle*}

\clearpage
\PrintIndex
\PrintChangelog

\end{document}

\begin{Declaration}{\Environment{tudpage}[\OLParameter{Sprache}]}
\begin{Declaration}[v2.02]{\Key{\Environment{tudpage}}{pagestyle=\PSet}}
\printdeclarationlist%
aaa
\end{Declaration}
\end{Declaration}

\Environment{atestenv}+language=\PName{Sprache}+(\Package{stuff})

\Macro{atestmacro}+bla=blubb+(\Package{stuff})

\Macro{test@macro}


\Environment{testenv}+language=\PName{Sprache}+(\Package{stuff})

\Macro{testmacro}+bla=blubb+(\Package{stuff})

\makeatletter
\Markup@Get@Arguments\tud@res@a{\Color{HKS41}[cddarkblue]}%
\meaning\tud@res@a

\Key*{\Macro{makecover}}{cdgeometry=false}

Das ist ein Paket \Length{testpaket} um etwas zu zeufen

\begin{Declaration}{\Color{HKS41}[cddarkblue]}
\printdeclarationlist%
bla
\end{Declaration}


\makeatother

Zur fehlerfreien Verwendung der \TUDScript-Klassen in der Version~\vTUDScript{} 
werden sowohl die \KOMAScript-Klassen der Version~\vKOMAScript{} oder später 
als auch die Hausschrift des \CDs \OpenSans aus dem Paket \Package{opensans}  



\Key{\Macro{testmacro}}{language=\PName{Sprache}}%(\Package{testpackage})

\Key{\Environment{testenv}}{language=\PName{Sprache}}%(\Package{testpackage})

\Option{foo}=\PName{bar}=


\begin{Declaration}[v2.07]{\Macro*{testindex}}
\printdeclarationlist
bbbbbbbbbbbbbbbbb
\end{Declaration}

\begin{Bundle}{\Package{bla}}
\begin{Declaration}[v2.07]{\Macro{testmacro}}
\begin{Declaration}[v2.07]{\Macro{testmacro}+language=\PName{Sprache}+}
\printdeclarationlist
bbbbbbbbbbbbbbbbb
\end{Declaration}
\end{Declaration}
\end{Bundle}

\begin{Declaration}[v2.07]{\Environment{testenv}[\LParameter]}
\begin{Declaration}[v2.07]{\Environment{testenv}+language=\PName{Sprache}+}
\printdeclarationlist
bbbbbbbbbbbbbbbbb
\end{Declaration}
\end{Declaration}

%\begin{Declaration}[%
%  v2.02;%
%  v2.04:Trennung einzelner Abschnitte mit \Macro{nextdeclaration};%
%]{\Environment{declarations}[\OLParameter{Sprache}]}

\bigskip

\Macro{testmacro}

\Macro{testmacro}+language=\PName{Sprache}+

\bigskip

\Environment{testenvA}

\Environment{testenvB}+language=\PName{Sprache}+


%\NewMacro{testmacroC}[\Parameter{gucke}]!language=\PName{Sprache}!
%
%\NewMacro{testmacroD}[\OParameter{gucke}]!language=\PName{Sprache}!
%
%\NewMacro{testmacroE}[\POParameter{gucke}]!language=\PName{Sprache}!

%\NewMacro{testmacroX}[\OParameter{gucke}]!language=\PName{Sprache}!(\Package{testpackage})


\clearpage
\PrintIndex

\end{document}
