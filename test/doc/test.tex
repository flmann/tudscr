\documentclass[english,ngerman,ttfont=roboto]{tudscrmanual}
\ifpdftex{
  \usepackage[T1]{fontenc}
  \input glyphtounicode
  \pdfgentounicode=1
  \usepackage[ngerman=ngerman-x-latest]{hyphsubst}
}{
  \usepackage{fontspec}
}
\lstset{%
  inputencoding=utf8,extendedchars=true,
  literate=%
    {ä}{{\"a}}1 {ö}{{\"o}}1 {ü}{{\"u}}1
    {Ä}{{\"A}}1 {Ö}{{\"O}}1 {Ü}{{\"U}}1
    {~}{{\textasciitilde}}1 {ß}{{\ss}}1
}

\usepackage{widows-and-orphans}

\usepackage{bookmark}
\KOMAoptions{headings=optiontoheadandtoc}

\tracinglabels[all]
%\tracingmarkup
%\tracingbundle
\usepackage{blindtext}

\begin{document}


\chapter{title}

\begin{Declaration}{\Environment{tudpage}[\OLParameter{Sprache}]}
\begin{Declaration}{\Key{\Environment{tudpage}}{language=\PSet{Sprache}}}
\begin{Declaration}{\Key{\Environment{tudpage}}{columns=\PSet{Anzahl}}}
\begin{Declaration}[v2.02]{%
  \Key{\Environment{tudpage}}{pagestyle=\PSet{Seitenstil}}%
}
\begin{Declaration}{\Key{\Environment{tudpage}}{cdfont=\PMisc}}(%
  \seeref{\Option{cdfont}'ppage'}%
)
\begin{Declaration}[v2.03]{\Key{\Environment{tudpage}}{cdhead=\PMisc}}(%
  \seeref{\Option{cdhead}'ppage'}%
)
\begin{Declaration}[v2.03]{\Key{\Environment{tudpage}}{cdfoot=\PMisc}}(%
  \seeref{\Option{cdfoot}'ppage'}%
)
\begin{Declaration}{\Key{\Environment{tudpage}}{headlogo=\PSet{Dateiname}}}(%
  \seeref{\Macro{headlogo}'ppage'}%
)
\begin{Declaration}[v2.03]{%
  \Key{\Environment{tudpage}}{footlogo=\PSet{Dateinamenliste}}
}(\seeref{\Macro{footlogo}'ppage'})
\begin{Declaration}[v2.02]{\Key{\Environment{tudpage}}{ddc=\PMisc}}(%
  \seeref{\Option{ddc}'ppage'}%
)
\begin{Declaration}[v2.02]{\Key{\Environment{tudpage}}{ddchead=\PMisc}}(%
  \seeref{\Option{ddchead}'ppage'}%
)
\begin{Declaration}[v2.02]{\Key{\Environment{tudpage}}{ddcfoot=\PMisc}}(%
  \seeref{\Option{ddcfoot}'ppage'}%
)
\printdeclarationlist%
\index{Layout!Kopfzeile}%
\index{Layout!Fußzeile}%
\index{Layout!Seitenstil}%
%
Die \Environment{tudpage}"~Umgebung hat ihren Ursprung in einer frühen Version 
von \TUDScript, als die \PageStyle{tudheadings}"=Seitenstile noch nicht 
verfügbar waren. Diese sollten mittlerweile bevorzugt gegenüber der hier 
beschriebenen Umgebung verwendet werden. Die \Environment{tudpage}"~Umgebung 
kann über verschiedene Parameter als optionales Argument angepasst werden. Wird 
das Paket \Package{babel} genutzt, kann die innerhalb der Umgebung angewandte 
Sprache mit \Key{\Environment{tudpage}}{language=\PSet{Sprache}} geändert 
werden, was zur Anpassung der sprachspezifischen Trennungsmuster und Bezeichner 
führt. Wurde das Paket \Package{multicol} geladen, wird mit dem Parameter 
\Key{\Environment{tudpage}}{columns=\PSet{Anzahl}} der Inhalt der Umgebung 
mit gegebenen Anzahl mehrspaltig gesetzt. Der Seitenstil, welcher durch die 
\Environment{tudpage}"~Umgebung verwendet wird, kann mit dem Parameter 
\Key{\Environment{tudpage}}{pagestyle=\PSet{Seitenstil}} angepasst werden, 
wobei \PValue{headings}, \PValue{plain} und \PValue{empty} gültige Werte sind. 

Die weiteren Parameter entsprechen in ihrem Verhalten den gleichnamigen 
Klassenoptionen oder Befehlen, wirken sich jedoch nur innerhalb der 
\Environment{tudpage}"~Umgebung aus. Das Verhalten sowie gültige 
Wertzuweisungen ist auf den angegebenen Seiten dokumentiert.
\end{Declaration}
\end{Declaration}
\end{Declaration}
\end{Declaration}
\end{Declaration}
\end{Declaration}
\end{Declaration}
\end{Declaration}
\end{Declaration}
\end{Declaration}
\end{Declaration}
\end{Declaration}

\end{document}


\Macro{notice}

\Macro{notice}+test+

\Macro{notice}+test=value+

\Macro{notice}+test+=newvalue=

\Macro{notice}=wrongvalue=



\bigskip\bigskip\bigskip


\Option{foo=bar}

\Option{foo}=stuff=

\Option{foo=\PMisc}


aaa


%\begin{Bundle*}{\Package{tudscrsupervisor}}
%\begin{Declaration}{\Environment{notice}[\OLParameter{Überschrift}]}
%\begin{Declaration}{\Key{\Environment{notice}}{headline=\PSet{Überschrift}}}
%\printdeclarationlist%
%%
%aaa
%\end{Declaration}
%\end{Declaration}
%\end{Bundle*}
%
%\begin{Declaration}{\Environment{tudpage}[\OLParameter{Sprache}]}
%\begin{Declaration}[v2.02]{\Key{\Environment{tudpage}}{pagestyle=\PSet{Seitenstil}}}
%\printdeclarationlist%
%aaa
%\end{Declaration}
%\end{Declaration}

\clearpage
\PrintIndex
\PrintChangelog

\end{document}
