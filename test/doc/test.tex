\documentclass[english,ngerman,ttfont=roboto]{tudscrmanual}
\ifpdftex{
  \usepackage[T1]{fontenc}
  \input glyphtounicode
  \pdfgentounicode=1
  \usepackage[ngerman=ngerman-x-latest]{hyphsubst}
}{
  \usepackage{fontspec}
}
\lstset{%
  inputencoding=utf8,extendedchars=true,
  literate=%
    {ä}{{\"a}}1 {ö}{{\"o}}1 {ü}{{\"u}}1
    {Ä}{{\"A}}1 {Ö}{{\"O}}1 {Ü}{{\"U}}1
    {~}{{\textasciitilde}}1 {ß}{{\ss}}1
}

\usepackage{bookmark}
\KOMAoptions{headings=optiontoheadandtoc}

\tracinglabels[all]
%\tracingmarkup
%\tracingbundle
\usepackage{blindtext}

\begin{document}
\newrobustcmd*\cdurl{%
  \begingroup%
    \hypersetup{hidelinks}%
    \href{https://tu-dresden.de/cd}{https://tu-dresden.de/cd}%
  \endgroup%
}
\faculty{\cdurl}
\subject{\TUDScript \vTUDScript{} basierend auf \KOMAScript}

\title{Ein {LaTeX}-Bundle für Dokumente im \TUDCD}
\ifdef{\tudprintflag}{%
  \subtitle{Benutzerhandbuch\thanks{\href{tudscr}{Online-Version}}}%
}{%
  \subtitle{Benutzerhandbuch\thanks{\href{tudscr_print}{Druckversion}}}%
}

\author{Falk Hanisch\TUDScriptContactNote}
\publishers{\GitHubRepo'{tudscr}[]}
\date{27.08.2019}


\makeatletter
\begingroup%
  \def\and{, }%
  \let\thanks\@gobble%
  \let\footnote\@gobble%
  \let\emailaddress\@gobble%
  \hypersetup{%
    pdfauthor = {\@author},%
    pdftitle = {\@title},%
    pdfsubject = {Benutzerhandbuch für \TUDScript},%
    pdfkeywords = {LaTeX, \TUDScript, Benutzerhandbuch},%
  }%
\endgroup%
\makeatother


\Key{\Macro{testmacro}}{language=\PName{Sprache}}%(\Package{testpackage})

%\begin{Declaration}[v2.07]{\NewMacro{testmacroB}}
%\begin{Declaration}[v2.07]{\NewMacro{testmacroB}+language=\PName{Sprache}+}
%\printdeclarationlist
%\end{Declaration}
%\end{Declaration}

\NewMacro{testmacroB}+language=\PName{Sprache}+

\NewMacro{testmacroC}

\Option{test}

\Option{testA=foo}

\Option{testB}=bla=

%\NewMacro{testmacroC}[\Parameter{gucke}]!language=\PName{Sprache}!
%
%\NewMacro{testmacroD}[\OParameter{gucke}]!language=\PName{Sprache}!
%
%\NewMacro{testmacroE}[\POParameter{gucke}]!language=\PName{Sprache}!

%\NewMacro{testmacroX}[\OParameter{gucke}]!language=\PName{Sprache}!(\Package{testpackage})


\clearpage
%\PrintIndex

\end{document}
