\documentclass[english,ngerman,ttfont=roboto]{tudscrmanual}
\ifpdftex{
  \usepackage[T1]{fontenc}
  \input glyphtounicode
  \pdfgentounicode=1
  \usepackage[ngerman=ngerman-x-latest]{hyphsubst}
}{
  \usepackage{fontspec}
}
\lstset{%
  inputencoding=utf8,extendedchars=true,
  literate=%
    {ä}{{\"a}}1 {ö}{{\"o}}1 {ü}{{\"u}}1
    {Ä}{{\"A}}1 {Ö}{{\"O}}1 {Ü}{{\"U}}1
    {~}{{\textasciitilde}}1 {ß}{{\ss}}1
}

\usepackage{widows-and-orphans}

\usepackage{bookmark}
\KOMAoptions{headings=optiontoheadandtoc}

\tracinglabels[all]
%\tracingmarkup
%\tracingbundle
\usepackage{blindtext}

\begin{document}

\begin{Declaration}[%
  v2.03!\Option{cd=bicolor}:%
    Farbeinsatz nur im Kopf mit farbig abgesetztem Querbalken;
  v2.03!\Option{cd=fullcolor}:%
    voller Farbeinsatz mit farbig abgesetztem Querbalken;%
  v2.04!\Option{cd=barcolor}:nur farbig abgesetzter Querbalken;%
  v2.06f:Überschriften verwenden auch bei \Option{cdfont=false} die Hausschrift,
  falls das Layout des \CDs nicht mit \Option{cd=false} deaktiviert wurde%
]{\Option{cd=\PMisc}}[true]
\printdeclarationlist%
%
Mit dieser Option wird festgelegt, ob und wie das \TUDCD im gesamten Dokument 
verwendet wird. Sie hat Einfluss auf die Ausprägung von Titel"~, Teil"~, und 
Kapitelseiten sowie die Überschriften der weiteren Gliederungsebenen. Im Layout 
des \CDs wird auch bei \Option{cdfont=false} die Hausschrift in Überschriften 
verwendet.
%
\begin{values}{\Option{cd}}
\itemfalse
  Es wird kein \CD sondern die Gestalt der \KOMAScript-Klassen genutzt.
\itemtrue*[nocolor/monochrome]
  Das Layout für Titel"~, Teil- und Kapitelseiten ist im \CD, es wird 
  schwarze Schrift für Titel sowie Teil- und Kapitelüberschriften verwendet.
  Die Ausprägung des Seitenkopfes ist abhängig von der Option \Option{cdhead}.
\item[lightcolor/pale]
  Die Einstellung entspricht weitestgehend der Option \Option{cd=true}, 
  allerdings wird die primäre Hausfarbe \Color{HKS41} für den Kopf des 
  \PageStyle{tudheadings}"=Seitenstils und Überschriften genutzt.
\item[barcolor]
  \ChangedAt{v2.04}
  Zusätzlich zur vorherigen Einstellung wird außerdem der Querbalken farbig 
  abgesetzt.
\item[bicolor/bichrome]
  \ChangedAt{v2.03}
  Der Kopf wird mit einem farbigen Hintergrund in der Hausfarbe gesetzt, auch 
  der Querbalken wird farbig hinterlegt. Für die Überschriften wird die 
  primären Hausfarbe verwendet.
\item[color]
  Der Titel sowie Teil- und Kapitelseiten werden allesamt farbig gestaltet, 
  der Seitenkopf wird in der primären Hausfarbe \Color{HKS41} gesetzt, der 
  Querbalken erhält Linien als Begrenzung.
\item[fullcolor/full]
  \ChangedAt{v2.03}
  Entspricht der vorherigen Einstellung, allerdings wird der Querbalken nicht 
  durch Linien begrenzt sondern farbig hinterlegt.
\end{values}
\end{Declaration}

\clearpage

\begin{Declaration}[%
  v2.03!\Option{cdnew=bicolor}:%
    Farbeinsatz nur im Kopf mit farbig abgesetztem Querbalken;
  v2.03!\Option{cdnew=fullcolor}:%
    voller Farbeinsatz mit farbig abgesetztem Querbalken;%
  v2.04!\Option{cdnew=barcolor}:nur farbig abgesetzter Querbalken;%
  v2.06f:Überschriften verwenden auch bei \Option{cdfont=false} die Hausschrift,
  falls das Layout des \CDs nicht mit \Option{cdnew=false} deaktiviert wurde%
]{\Option{cdnew=\PMisc}}[true]<aaa>
\begin{Declaration}{\Option{cdverynew=\PMisc}}[true]<aaa>
\printdeclarationlist%
%
Mit dieser Option wird festgelegt, ob und wie das \TUDCD im gesamten Dokument 
verwendet wird. Sie hat Einfluss auf die Ausprägung von Titel"~, Teil"~, und 
Kapitelseiten sowie die Überschriften der weiteren Gliederungsebenen. Im Layout 
des \CDs wird auch bei \Option{cdfont=false} die Hausschrift in Überschriften 
verwendet.
%
\begin{DeclarationValues}%
\itemvalfalse
  Es wird kein \CD sondern die Gestalt der \KOMAScript-Klassen genutzt.
\itemvaltrue*=nocolor/monochrome=
  Das Layout für Titel"~, Teil- und Kapitelseiten ist im \CD, es wird 
  schwarze Schrift für Titel sowie Teil- und Kapitelüberschriften verwendet.
  Die Ausprägung des Seitenkopfes ist abhängig von der Option \Option{cdhead}.
\itemval=lightcolor/pale=
  Die Einstellung entspricht weitestgehend der Option \Option{cdnew=true}, 
  allerdings wird die primäre Hausfarbe \Color{HKS41} für den Kopf des 
  \PageStyle{tudheadings}"=Seitenstils und Überschriften genutzt.
\itemval[v2.04]=barcolor=
  Zusätzlich zur vorherigen Einstellung wird außerdem der Querbalken farbig 
  abgesetzt.
\itemval=bicolor/bichrome=
  \ChangedAt{v2.03}
  Der Kopf wird mit einem farbigen Hintergrund in der Hausfarbe gesetzt, auch 
  der Querbalken wird farbig hinterlegt. Für die Überschriften wird die 
  primären Hausfarbe verwendet.
\itemval=color=
  Der Titel sowie Teil- und Kapitelseiten werden allesamt farbig gestaltet, 
  der Seitenkopf wird in der primären Hausfarbe \Color{HKS41} gesetzt, der 
  Querbalken erhält Linien als Begrenzung.
\itemval=fullcolor/full=
  \ChangedAt{v2.03}
  Entspricht der vorherigen Einstellung, allerdings wird der Querbalken nicht 
  durch Linien begrenzt sondern farbig hinterlegt.
\end{DeclarationValues}
\end{Declaration}
\end{Declaration}

\Option{cdnew=fullcolor}

\Option{cdverynew=fullcolor}


\clearpage
%\PrintIndex
\PrintChangelog

\end{document}
