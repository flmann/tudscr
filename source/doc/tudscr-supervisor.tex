\setchapterpreamble{\tudhyperdef'{sec:supervisor}}
\chapter[Das Paket \Package*{tudscrsupervisor} -- Studentische Betreuung]{%
  Betreuung studentischer Arbeiten
}
\begin{Bundle*}{\Package{tudscrsupervisor}}
%
Dieses Paket stellt für das Erstellen von Aufgabenstellungen und Gutachten 
wissenschaftlicher Arbeiten sowie offiziellen Aushängen im \CD passende 
Umgebungen und Befehle für den Anwender bereit. Deshalb richtet es sich 
vornehmlich an Mitarbeiter an der \TnUD, kann jedoch natürlich auch von 
Studenten genutzt werden.


\section{Aufgabenstellung für eine wissenschaftliche Arbeit}
\index{Aufgabenstellung|!(}%
%
\begin{Declaration}{\Environment{task}[\OLParameter{Überschrift}]}(%
  \Environment{tudpage}'auto'%
)
\begin{Declaration}{\Key{\Environment{task}}{headline=\PName{Überschrift}}}
\begin{Declaration}[v2.05]{\Key{\Environment{task}}{style=\PName{Stil}}}
\printdeclarationlist%
%
Mit der \Environment{task}"~Umgebung kann ein Aufgabenstellung für eine 
wissenschaftliche Arbeit ausgegeben werden. Diese basiert auf der Umgebung 
\Environment{tudpage} und akzeptiert deshalb im optionalen Argument alle 
Parameter, welche bei der Beschreibung von \Environment{tudpage}'full' 
erläutert wurden.

Für die Aufgabenstellung wird normalerweise eine Überschrift gesetzt, welche 
sich aus \Term{taskname} und~-- falls der Typ der Abschlussarbeit angegeben 
wurde~-- noch aus \Term{tasktext} und \Macro{thesis} zusammensetzt. Der 
Parameter \Key{\Environment{task}}{headline} kann genutzt werden, um diese 
automatisch generierte Überschrift anzupassen.

Als Kopf der Aufgabenstellung erscheint eine Tabelle mit den angegebenen 
Informationen zum Autor respektive zu den Autoren der Abschlussarbeit. Zwingend 
anzugeben sind dafür lediglich der oder die Verfasser (\Macro{author}())
sowie ein Titel der Abschlussarbeit (\Macro{title}), welcher am Ende der 
Tabelle in fetter Schrift aufgeführt wird. Optional werden noch die Felder für 
den Studiengang (\Macro{course}()), die Fachrichtung (\Macro{discipline}()) und 
die Matrikelnummer (\Macro{matriculationnumber}) sowie das Immatrikulationsjahr 
(\Macro{matriculationyear}) angefügt, wobei nicht befüllte Felder ignoriert 
werden. Der eigentliche Inhalt der Umgebung~-- sprich die Aufgabenstellung 
selbst~-- wird nach dem generierten Kopf ausgegeben.

Dem Inhalt der Aufgabenstellung folgt eine zeilenweise Auflistung des oder der 
definierten Gutachter beziehungsweise Prüfer (\Macro{referee}) sowie Betreuer 
(\Macro{supervisor}). Dabei wird vor dem jeweiligen Namen der dazugehörige 
Bezeichner (\Term{refereename}, \Term{refereeothername} respektive 
\Term{supervisorname}, \Term{supervisorothername}) gesetzt. 
\ChangedAt{v2.05}
Dies ist das voreingestellte Verhalten und kann über die Wahl des Parameters
\Key{\Environment{task}}{style=table} aktiviert werden. Wird hingegen der  
Parameter \Key{\Environment{task}}{style=inline} gesetzt, so erfolgt die 
Ausgabe von mehreren Prüfern und Betreuern in einer Zeile. Die Bezeichner sind 
problemlos anpassbar, siehe dazu \autoref{sec:localization}. Danach erscheinen 
das Ausgabedatum (\Macro{issuedate}) sowie der verpflichtende Abgabetermin 
(\Macro{duedate}). Zum Schluss wird die Unterschriftzeile für den 
Prüfungsausschussvorsitzenden (\Macro{chairman}) und den betreuenden 
Hochschullehrer (\Macro{professor}()) gesetzt. Für genannte Personen werden 
unter dem Namen selbst die Bezeichner ausgegeben (\Term{chairmanname} und 
\Term{professorname}).
\end{Declaration}
\end{Declaration}
\end{Declaration}

\begin{Declaration}{%
  \Macro{taskform}[\LParameter\Parameter{Ziele}\Parameter{Schwerpunkte}]%
}
\printdeclarationlist%
%
Zusätzlich zur der frei gestaltbaren Umgebung \Environment{task} zur Erstellung
einer Aufgabenstellung wird ein separater Befehl für eine standardisierte 
Ausgabe zur Verfügung gestellt. Dieser strukturiert die Aufgabenstellung in die 
zwei Bereiche \emph{Ziele} und \emph{Schwerpunkte} der Arbeit mit dazugehörigen 
Überschriften (\Term{objectivesname}, \Term{focusname}).

Im optionalen Argument können alle Parameter der Umgebung \Environment{task} 
verwendet werden. Im ersten obligatorischen Argument sollte ein Text mit einer 
kurzen thematischen Einordnung und dem eigentlichen Ziel der Arbeit erscheinen, 
im zweiten Argument sollen die thematischen Schwerpunkte in Stichpunkten 
benannt werden. Der Inhalt des zweiten notwendigen Argumentes wird in einer 
\Environment{itemize}(\Package{koma-script},\Package{enumitem})"~Umgebung
gesetzt. Deshalb \emph{muss} jedem Stichpunkt 
\Macro{item}(\Package{koma-script},\Package{enumitem}) vorangestellt 
werden.
\end{Declaration}
%
\begin{Example}
Die empfohlene Verwendung des Befehls \Macro{taskform} ist wie folgt:
\begin{Code}[escapechar=§]
\taskform{%
  Motivation der Arbeit im ersten Absatz§\dots§
  
  Ziele der Arbeit im zweiten Absatz§\dots§
}{%
  \item Schwerpunkt 1
  \item Schwerpunkt 2
}
\end{Code}
Hierzu sei auch auf das Minimalbeispiel in \autoref{sec:exmpl:task} verwiesen.
\index{Aufgabenstellung|!)}%
\end{Example}

\begin{Declaration}{\Macro{chairman}[\Parameter{Prüfungsausschussvorsitzender}]}
\printdeclarationlist%
%
Wird dieses Feld genutzt, wird neben dem betreuenden Hochschullehrer 
(\Macro{professor}()) auch der Vorsitzende des Prüfungsausschusses am Ende der 
Aufgabenstellung aufgeführt. Dies wird zumeist für Abschlussarbeiten wie 
beispielsweise \masterthesisname{} oder \diplomathesisname{} benötigt.
\end{Declaration}

\begin{Declaration}{\Macro{issuedate}[\Parameter{Ausgabedatum}]}
\begin{Declaration}{\Macro{duedate}[\Parameter{Abgabetermin}]}
\printdeclarationlist%
%
Mit diesen beiden Befehlen sollte das Datum der Ausgabe der Aufgabenstellung 
sowie der spätest mögliche Abgabetermin angegeben werden. Wurde das Paket 
\Package{isodate} oder \Package{datetime2} geladen, wird die damit eingestellte 
Ausgabeformatierung des Datums mit \Macro{printdate}(\Package{isodate}) 
beziehungsweise \Macro{DTMDate}(\Package{datetime2}) für \Macro{duedate} und 
\Macro{issuedate} verwendet.
\end{Declaration}
\end{Declaration}


\section{Gutachten für wissenschaftliche Arbeiten}
\index{Gutachten|!(}%
%
\begin{Declaration}{\Environment{evaluation}[\OLParameter{Überschrift}]}(%
  \Environment{tudpage}'auto'%
)
\begin{Declaration}{%
  \Key{\Environment{evaluation}}{headline=\PName{Überschrift}}%
}
\begin{Declaration}{\Key{\Environment{evaluation}}{grade=\PName{Note}}}
\printdeclarationlist%
%
Diese Umgebung wird für das Erstellen eines Gutachtens einer wissenschaftlichen 
Arbeit bereitgestellt. Auch diese unterstützt alle Parameter, welche für die 
Umgebung \Environment{tudpage}'full' beschrieben wurden.

Für ein Gutachten wird gewöhnlich eine Überschrift aus \Term{evaluationname} 
und~-- falls der Abschlussarbeitstyp angegeben wurde~-- \Term{evaluationtext} 
sowie \Macro{thesis} generiert. Diese automatisch generierte Überschrift kann 
mit dem Parameter \Key{\Environment{evaluation}}{headline} ersetzt werden. Am 
Ende des Gutachtens wird die mit \Key{\Environment{evaluation}}{grade} 
gegebene Note in fetter Schrift ausgezeichnet.

Am Anfang der \Environment{evaluation}"~Umgebung wird die gleiche Tabelle mit 
Autorenangaben ausgegeben, wie dies bei der \Environment{task}"~Umgebung der 
Fall ist. Nach dem Tabellenkopf folgt auch hier der eigentliche Inhalt, sprich 
das Gutachten der Abschlussarbeit. Abgeschlossen wird die Umgebung mit der 
gegebenen Note~-- welche innerhalb von \Term{gradetext} ausgegeben wird~-- 
sowie der Orts- und Datumsangabe (\Macro{place}, \Macro{date}) und der 
darauffolgenden Unterschriftzeile für den oder die Gutachter (\Macro{referee}), 
welche wiederum mit den entsprechenden sprachabhängigen Bezeichner 
(\Term{refereename}, \Term{refereeothername}) ergänzt werden.
\end{Declaration}
\end{Declaration}
\end{Declaration}

\begin{Declaration}{%
  \hskip-.06em\Macro{evaluationform}[%
    \LParameter\Parameter{Aufgabe}\Parameter{Inhalt}%
    \Parameter{Bewertung}\Parameter{Note}\hskip-.06em%
  ]%
}
\printdeclarationlist%
%
Neben der individuell nutzbaren Umgebung \Environment{evaluation} wird ein 
separater Befehl zur Erstellung eines standardisierten Gutachtens 
bereitgestellt. Dieser strukturiert die Ausgabe in die vier Bereiche 
\emph{Aufgabe}, \emph{Inhalt}, \emph{Bewertung} und \emph{Note} und versieht 
diese jeweils mit der dazugehörigen Überschrift beziehungsweise Textausgabe 
(\Term{taskname}, \Term{contentname}, \Term{assessmentname} und 
\Term{gradetext}). Das optionale Argument unterstützt alle Parameter der 
\Environment{evaluation}"~Umgebung.
\end{Declaration}
%
\begin{Example}
Die empfohlene Verwendung des Befehls \Macro{evaluationform} ist wie folgt:
\begin{Code}[escapechar=§]
\evaluationform{%
  Kurzbeschreibung der Aufgabenstellung§\dots§
}{%
  Zusammenfassung von Inhalt und Struktur§\dots§
}{%
  Bewertung der schriftlichen Abschlussarbeit§\dots§
}{%
  Zahl (Note)
}
\end{Code}
Hierzu sei auch auf das Minimalbeispiel in \autoref{sec:exmpl:evaluation} 
verwiesen.
\index{Gutachten|!)}%
\end{Example}

\begin{Declaration}{\Macro{grade}[\Parameter{Note}]}
\printdeclarationlist%
%
Neben der Angabe der Note für ein Gutachten über den Parameter 
\Key{\Environment{evaluation}}{grade} der Umgebung \Environment{evaluation} 
kann dafür auch dieser global wirkende Befehl verwendet werden.
\end{Declaration}


\section{Aushang}
\index{Aushang|!(}%
%
\begin{Declaration}{\Environment{notice}[\OLParameter{Überschrift}]}(%
  \Environment{tudpage}'auto'%
)
\begin{Declaration}{\Key{\Environment{notice}}{headline=\PName{Überschrift}}}
\printdeclarationlist%
%
Für das Anfertigen eines Aushangs kann diese Umgebung verwendet werden. Diese 
basiert abermals auf der Umgebung \Environment{tudpage} und unterstützt alle 
deren Parameter.

Wurde ein Datum angegeben, wird dieses standardmäßig rechtsbündig oberhalb des 
Textbereiches angezeigt (\seeref{\Key{\Environment{tudpage}}{cdhead}}). Danach 
erfolgt die Ausgabe der Überschrift, welche für gewöhnlich dem Inhalt von 
\Term{noticename} entspricht, allerdings sehr einfach mit dem Parameter 
\Key{\Environment{notice}}{headline} geändert werden kann. Nach der Überschrift 
folgt der Inhalt der Umgebung. Wurde mit \Macro{contactperson} ein oder mehrere 
Ansprechpartner angegeben, werden diese Informationen am Ende der Umgebung 
gesetzt.
\end{Declaration}
\end{Declaration}

\begin{Declaration}{%
  \Macro{noticeform}[\LParameter\Parameter{Inhalt}\Parameter{Schwerpunkte}]%
}
\printdeclarationlist%
%
Auch für diese Umgebung gibt es einen Befehl für eine normierte Form. Diese 
soll vor allem Verwendung für den Aushang studentischer Arbeitsthemen finden. 
Für das optionale Argument können sämtliche Parameter verwendet werden, die 
auch die \Environment{notice}"~Umgebung unterstützt.

Das erste obligatorische Argument wird für eine kurze Inhaltsbeschreibung 
verwendet. Neben dem Text sollte hier wenn möglich eine thematisch passende 
Abbildung eingefügt werden (\Macro{includegraphics}(\Package{graphicx})). 
Das zweite Argument wird~-- wie schon bei \Macro{taskform}~-- dazu verwendet, 
einige Schwerpunkte aufzuzählen. Auch hier kommt nach der gliedernden 
Überschrift (\Term{focusname}) eine 
\Environment{itemize}(\Package{koma-script},\Package{enumitem})"~Umgebung zum 
Einsatz, allen Schwerpunkten muss ein 
\Macro{item}(\Package{koma-script},\Package{enumitem}) vorangestellt werden.
\end{Declaration}
%
\begin{Example}
Die empfohlene Verwendung des Befehls \Macro{noticeform} ist wie folgt:
\begin{Code}[escapechar=§]
\noticeform{%
  Kurzbeschreibung des Inhaltes der studentischen Arbeit§\dots§
  
  Bild (optional), einzubinden mit:
    \includegraphics[§\PName{Einstellungen}§]{§\PName{Datei}§}
}{%
  \item Schwerpunkt 1
  \item Schwerpunkt 2
}
\end{Code}
Hierzu sei auch auf das Minimalbeispiel in \autoref{sec:exmpl:notice} verwiesen.
\index{Aushang|!)}%
\end{Example}

\begin{Declaration}[v2.02]{\Macro{contactperson}[\Parameter{Kontaktperson(en)}]}
\begin{Declaration}{\Macro{office}[\Parameter{Dienstsitz}]}
\begin{Declaration}[v2.02]{\Macro{telephone}[\Parameter{Telefonnummer}]}
\begin{Declaration}[v2.05]{\Macro{telefax}[\Parameter{Telefaxnummer}]}

\printdeclarationlist%
%
Am Ende eines Aushangs können mit \Macro{contactperson} Kontaktinformationen 
für eine oder mehrere Ansprechpartner angegeben werden. Soll mehr als eine 
Kontaktperson genannt werden, so müssen diese innerhalb des Befehls
\Macro{contactperson} mit dem Befehl \Macro{and} getrennt werden. Für jede 
Person kann innerhalb des Argumentes von \Macro{contactperson} der Dienstsitz 
(\Macro{office}), die dienstliche Telefon- (\Macro{telephone}) sowie "~faxnummer
(\Macro{telefax}) und die geschäftliche E"~Mail"=Adresse (\Macro{emailaddress}) 
angegeben werden.
\end{Declaration}
\end{Declaration}
\end{Declaration}
\end{Declaration}



\section{Zusätzliche sprachabhängige Bezeichner}
\index{Bezeichner|!(}%
%
Für das Paket \Package{tudscrsupervisor} werden für die zusätzlichen Befehle 
und Umgebungen weitere Bezeichner definiert. Für eine etwaige Anpassung dieser 
sei auf \autoref{sec:localization} verwiesen.

\begin{Declaration}{\Term{taskname}}
\begin{Declaration}{\Term{tasktext}}
\printdeclarationlist%
%
Die Bezeichnung der Aufgabenstellung selbst ist in \Term{taskname} enthalten. 
Für die Generierung einer Überschrift wird dieser verwendet. Wurde außerdem mit 
\Macro{thesis} oder \Macro{subject} der Typ der Abschlussarbeit%
\footnote{%
  \Option{subjectthesis} oder spezieller Wert aus \autoref{tab:thesis}%
}
angegeben, wird die Überschrift zusammen mit dem Bezeichner \Term{tasktext}
um die Typisierung erweitert. Falls gewünscht, kann die automatisch generierte 
Überschrift mit dem Parameter \Key{\Environment{task}}{headline} der Umgebung 
\Environment{task} überschrieben werden.
\TermTable{taskname,tasktext}
\end{Declaration}
\end{Declaration}

\begin{Declaration}[v2.04]{\Term{namesname}}
\begin{Declaration}{\Term{titlename}}
\printdeclarationlist%
%
Diese beiden Bezeichner werden in der Tabelle mit den Autoreninformationen zu 
Beginn der Aufgabenstellung verwendet.
\TermTable{namesname,titlename}
\end{Declaration}
\end{Declaration}

\begin{Declaration}{\Term{issuedatetext}}
\begin{Declaration}{\Term{duedatetext}}
\printdeclarationlist%
%
Am Ende der Aufgabenstellung wird nach dem oder der Betreuer das Ausgabedatum 
und der Abgabetermin (\Macro{issuedate}, \Macro{duedate}) der Abschlussarbeit 
mit folgenden Bezeichner erläutert.
\TermTable{issuedatetext,duedatetext}
\end{Declaration}
\end{Declaration}

\begin{Declaration}{\Term{chairmanname}}
\printdeclarationlist%
%
Wurde der Prüfungsausschussvorsitzende (\Macro{chairman}) angegeben, erfolgt 
unter dem Namen selbst die Ausgabe des Bezeichners.
\TermTable{chairmanname}
\end{Declaration}

\begin{Declaration}{\Term{focusname}}
\begin{Declaration}{\Term{objectivesname}}
\printdeclarationlist%
%
Die Standardformen für Aufgabenstellung (\Macro{taskform}) respektive Aushang 
(\Macro{noticeform}) nutzen für die gesetzten Überschriften diese Bezeichner.
\TermTable{focusname,objectivesname}
\end{Declaration}
\end{Declaration}

\begin{Declaration}{\Term{evaluationname}}
\begin{Declaration}{\Term{evaluationtext}}
\printdeclarationlist%
%
Die Bezeichnung des Gutachten selbst ist in \Term{evaluationname} enthalten. 
Für die Generierung der Überschrift wird der Bezeichner \Term{evaluationtext} 
sowie der mit \Macro{thesis} oder gegebenenfalls mit \Macro{subject} gegebenen 
Typ der Abschlussarbeit verwendet. Diese automatisch generierte Überschrift 
kann mit dem Parameter \Key{\Environment{evaluation}}{headline} der 
Umgebung \Environment{evaluation} durch den Anwender überschrieben werden.
\TermTable{evaluationname,evaluationtext}
\end{Declaration}
\end{Declaration}

\begin{Declaration}{\Term{contentname}}
\begin{Declaration}{\Term{assessmentname}}
\printdeclarationlist%
%
Bei der standardisierten Form des Gutachtens (\Macro{evaluationform}) werden 
die darin~-- für eine strukturierte Gliederung~-- erzeugten Überschriften mit 
den Bezeichnern \Term{taskname}, \Term{contentname} und \Term{assessmentname} 
gesetzt.
\TermTable{taskname,contentname,assessmentname}
\end{Declaration}
\end{Declaration}

\begin{Declaration}{\Term{gradetext}}
\printdeclarationlist%
%
Wird für das Gutachten einer wissenschaftlichen Arbeit die erzielte Note 
entweder mit dem Befehl \Macro{grade}[\Parameter{Note}] oder alternativ dazu 
mit dem Parameter \Key{\Environment{evaluation}}{grade=\PName{Note}} der 
Umgebung \Environment{evaluation} angegeben, so wird diese innerhalb von 
\Term{gradetext} verwendet.
\grade{\PName{Note}}
\TermTable*{gradetext}[.8\textwidth]
\end{Declaration}

\begin{Declaration}{\Term{noticename}}
\begin{Declaration}[v2.02]{\Term{contactpersonname}}
\printdeclarationlist%
%
Die Bezeichnung des Aushangs selbst ist in \Term{noticename} enthalten. Für 
die Generierung einer Überschrift wird dieser verwendet. Falls gewünscht, kann 
diese mit dem Parameter \Key{\Environment{notice}}{headline} der Umgebung 
\Environment{notice} überschrieben werden. Wurde eine Kontaktperson mit dem 
Befehl \Macro{contactperson} angegeben, wird als Überschrift der Kontaktdaten 
der Bezeichner \Term{contactpersonname} verwendet.

\TermTable{noticename,contactpersonname}
\end{Declaration}
\end{Declaration}
\index{Bezeichner|!)}%
\end{Bundle*}
