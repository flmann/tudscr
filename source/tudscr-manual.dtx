% \CheckSum{4488}
% \iffalse meta-comment
%
%  TUD-Script -- Corporate Design of Technische Universität Dresden
% ----------------------------------------------------------------------------
%
%  Copyright (C) Falk Hanisch <hanisch.latex@outlook.com>, 2012-2019
%
% ----------------------------------------------------------------------------
%
%  This work may be distributed and/or modified under the conditions of the
%  LaTeX Project Public License, version 1.3c of the license. The latest
%  version of this license is in http://www.latex-project.org/lppl.txt and
%  version 1.3c or later is part of all distributions of LaTeX 2005/12/01
%  or later and of this work. This work has the LPPL maintenance status
%  "author-maintained". The current maintainer and author of this work
%  is Falk Hanisch.
%
% ----------------------------------------------------------------------------
%
%  Dieses Werk darf nach den Bedingungen der LaTeX Project Public Lizenz
%  in der Version 1.3c, verteilt und/oder verändert werden. Die aktuelle
%  Version dieser Lizenz ist http://www.latex-project.org/lppl.txt und
%  Version 1.3c oder später ist Teil aller Verteilungen von LaTeX 2005/12/01
%  oder später und dieses Werks. Dieses Werk hat den LPPL-Verwaltungs-Status
%  "author-maintained", wird somit allein durch den Autor verwaltet. Der
%  aktuelle Verwalter und Autor dieses Werkes ist Falk Hanisch.
%
% ----------------------------------------------------------------------------
%
% \fi
%
% \CharacterTable
%  {Upper-case    \A\B\C\D\E\F\G\H\I\J\K\L\M\N\O\P\Q\R\S\T\U\V\W\X\Y\Z
%   Lower-case    \a\b\c\d\e\f\g\h\i\j\k\l\m\n\o\p\q\r\s\t\u\v\w\x\y\z
%   Digits        \0\1\2\3\4\5\6\7\8\9
%   Exclamation   \!     Double quote  \"     Hash (number) \#
%   Dollar        \$     Percent       \%     Ampersand     \&
%   Acute accent  \'     Left paren    \(     Right paren   \)
%   Asterisk      \*     Plus          \+     Comma         \,
%   Minus         \-     Point         \.     Solidus       \/
%   Colon         \:     Semicolon     \;     Less than     \<
%   Equals        \=     Greater than  \>     Question mark \?
%   Commercial at \@     Left bracket  \[     Backslash     \\
%   Right bracket \]     Circumflex    \^     Underscore    \_
%   Grave accent  \`     Left brace    \{     Vertical bar  \|
%   Right brace   \}     Tilde         \~}
%
% \iffalse
%%% From File: tudscr-manual.dtx
%<*dtx>
% \fi
%
\ifx\ProvidesFile\undefined\def\ProvidesFile#1[#2]{}\fi
\ProvidesFile{tudscr-manual.dtx}[2019/09/19 v2.06e TUD-Script\space%
%
% \iffalse
%</dtx>
%<package&identify>\ProvidesPackage{tudscrtutorial}[%
%<*package&identify>
%!TUD@Version
%</package&identify>
%<package&identify>  package
%<*dtx|package&identify>
% \fi
  (internal tutorials)%
]
% \iffalse
%</dtx|package&identify>
%<*dtx>
\documentclass[english,ngerman,xindy]{tudscrdoc}
\ifpdftex{
  \usepackage[T1]{fontenc}
  \usepackage[ngerman=ngerman-x-latest]{hyphsubst}
}{
  \usepackage{fontspec}
}
\usepackage{babel}
\usepackage{tudscrfonts}
\KOMAoptions{parskip=half-}
\usepackage{bookmark}
\usepackage[babel]{microtype}

\CodelineIndex
\RecordChanges
\GetFileInfo{tudscr-manual.dtx}
\title{\file{\filename}}
\author{Falk Hanisch\qquad\expandafter\mailto\expandafter{\tudscrmail}}
\date{\fileversion\nobreakspace(\filedate)}

\begin{document}
  \maketitle
  \tableofcontents
  \DocInput{\filename}
\end{document}
%</dtx>
% \fi
%
% \selectlanguage{ngerman}
%
% \changes{v2.02}{2014/12/17}{Erstellung der \app{texindy}-Stildatei während
%   der Kompilierung}^^A
% \changes{v2.02}{2014/11/18}{\pkg{tudscrtutorial}: Index für Tutorials}^^A
% \changes{v2.05}{2015/08/04}{Geteilte Deklarationen für Klasse und Paket}^^A
%
% \section{Handbuch und Leitfäden zu \TUDScript}
%
% Es werden die Klasse \cls{tudscrmanual} für das Handuch sowie das Paket
% \pkg{tudscrtutorial} für das Setzen von einigen Anwenderleitfäden bzw.
% Tutorials erzeugt. Da es zwischen der Klasse und dem Paket es eine große
% Menge an Überschneidungen gibt, basieren diese auf der gleichen Quelldatei. 
% Einiges davon wird auch für die Quelltextdokumentationsklasse \cls{tudscrdoc} 
% verwendet.
%
% \StopEventually{\PrintIndex\PrintChanges\PrintToDos}
%
% \iffalse
%<*body|class&doc>
% \fi
%
% \subsection{Notwendige Pakete und Befehle}
%
% Basis für die Handbuchklasse \cls{tudscrmanual} ist \cls{tudscrreprt}. 
% Zusätzlich wird das Paket \pkg{tudscrtutorial} erstellt, um an das Handbuch
% angelehnte Tutorials in einem Unterordner zu erstellen. Sowohl für die Klasse
% als auch das Paket werden einige Pakete geladen, um Einstellungen vorzunehmen 
% und nützliche Befehle zu definieren.
%
% Das Paket \pkg{xparse} erlaubt eine sehr freie Deklaration von Makros mit
% nahezu beliebig arrangierbaren (optionalen) Argumenten.
%    \begin{macrocode}
%<*!doc>
\PassOptionsToPackage{log-declarations=false}{xparse}
\RequirePackage{xparse}[2013/12/31]
%</!doc>
%    \end{macrocode}
% Bugfixes für unterschiedliche Pakete.
%    \begin{macrocode}
\RequirePackage{scrhack}[%
%!TUD@KOMAVersion
]
%    \end{macrocode}
% Die Sprachunterstützung für Klasse und Paket.
%    \begin{macrocode}
%<*!doc>
\RequirePackage{babel}[2014/03/24]
%</!doc>
%    \end{macrocode}
% Das Paket \pkg{marginnote} stellt nicht fließenden Randnotizen bereit, welche 
% für die Kennzeichnung von Änderungen, Hinweise oder ToDo-Notizen verwendet 
% werden. Die Randnotizen werden auf dem linken~-- weil größeren~-- Seitenrand
% gesetzt.
%    \begin{macrocode}
\PassOptionsToPackage{quiet}{marginnote}
\RequirePackage{marginnote}[2012/03/29]
%<!doc>\reversemarginpar
%    \end{macrocode}
% Das Paket für intelligente Leerzeichen am Ende von Makros mit \cs{xspace}.
%    \begin{macrocode}
\RequirePackage{xspace}[2009/10/20]
\xspaceaddexceptions{"=}
%    \end{macrocode}
% Für das Paket \pkg{geometry} erfolgen durch \TUDScript vielerlei Anpassungen. 
% Um bei einer signifikanten Änderung des Paketes passend reagieren zu können, 
% wird auf das aktuelle Paketdatum geprüft und für neue Versionen eine Warnung
% erzeugt.
% \ToDo{Kann raus, sobald \pkg{geometry} nicht mehr notwendig}[v2.07]
%    \begin{macrocode}
%<*!(doc|package)>
\@ifpackagelater{geometry}{2018/04/17}{%
  \ClassWarning{\TUD@Class@Name}{%
    Package `geometry' was updated so some patches\MessageBreak%
    are maybe outdated%
  }%
}{}%
%</!(doc|package)>
%    \end{macrocode}
% \begin{macro}{\vTUDScript}
% \begin{macro}{\vKOMAScript}
% Diese Befehle geben in der Dokumentation die aktuelle Version von \TUDScript
% sowie die mindestens notwendige Version von \KOMAScript{} aus.
%    \begin{macrocode}
\newcommand*\vTUDScript{v\TUDScriptVersionNumber}
\newcommand*\vKOMAScript{v\TUD@KOMAVersionNumber}
%    \end{macrocode}
% \end{macro}^^A \vKOMAScript
% \end{macro}^^A \vTUDScript
% \begin{macro}{\tud@english}
% Mit diesem Befehl wird der Inhalt des Arumentes mit englischen Trennmustern 
% gesetzt, falls die Sprache geladen wurde.
%    \begin{macrocode}
\newcommand*\tud@english[1]{#1}
\AtBeginDocument{%
  \iflanguageloaded{english}{%
    \renewcommand*\tud@english[1]{\foreignlanguage{english}{#1}}%
  }{}%
}
%    \end{macrocode}
% \end{macro}^^A \tud@english
% Das Paket \pkg{shellesc} ermöglicht mit dem Befehl \cs{ShellEscape} die 
% Verwendung von Systembefehlen auf der Kommandozeile unabhängig von der 
% genutzten Engine. Sollte \app{lualatex} als Dokumentprozessor eingesetzt
% werden, sind dennoch die \app{pdflatex}-Primitiven \cs{pdf(@)strcmp} und
% \cs{pdf(@)shellescape} nötig, wofür das Paket \pkg{pdftexcmds} geladen wird.
%    \begin{macrocode}
\RequirePackage{shellesc}[2016/06/07]
\RequirePackage{pdftexcmds}[2016/05/21]
%    \end{macrocode}
%
% \iffalse
%</body|class&doc>
%<*!doc>
%<*body>
%<*class>
% \fi
%
% \begin{macro}{\tud@list@sort}
% \changes{v2.02}{2014/07/25}{neu}^^A
% \begin{macro}{\tud@list@@sort}
% \changes{v2.02}{2014/07/25}{neu}^^A
% \begin{macro}{\tud@templist}
% \changes{v2.02}{2014/07/25}{neu}^^A
% \begin{macro}{\if@tud@list@sorted}
% \changes{v2.02}{2014/07/25}{neu}^^A
% Der Befehl \cs{tud@list@sort} erwartet eine \pkg{etoolbox}-Liste und sortiert 
% diese mit dem Makro \cs{sort@list} in alphabetischer Reihenfolge in eine 
% kommagetrennte Auflistung. Diese wird anschließend wieder in eine
% \pkg{etoolbox}-Liste umgewandelt.
%    \begin{macrocode}
\newcommand*\tud@templist{}
\let\tud@templist\relax
\newcommand*\tud@list@sort[1]{%
  \ifdefvoid{#1}{%
    \ClassWarning{tudscrmanual}{%
      The given list \string#1\space\MessageBreak%
      is empty, nothing to sort here%
    }%
  }{%
    \let\tud@reserved#1%
    \let\tud@templist\relax%
    \forlistloop\tud@list@@sort{\tud@reserved}%
    \let\tud@reserved\relax%
    \@for\@tempa:=\tud@templist\do{\listeadd\tud@reserved{\@tempa}}%
    \let#1\tud@reserved%
  }%
}
%    \end{macrocode}
% Hier erfolgt die eigentliche Sortierung der Liste.
%    \begin{macrocode}
\newif\if@tud@list@sorted
\newcommand*\tud@list@@sort[1]{%
%    \end{macrocode}
% Der erste Eintrag wird direkt der Liste hinzugefügt.
%    \begin{macrocode}
  \ifx\tud@templist\relax%
    \def\tud@templist{#1}%
%    \end{macrocode}
% Weitere Einräge werden mit \cs{pdfstrcmp} an der richtigen Stelle eingefügt.
% Dabei wird die kommagetrennte Liste \cs{tud@templist} durchlaufen und
% \cs{@tempb} als Hilfsmakro verwendet, in welches die aktuelle Liste innerhalb 
% der \cs{@for}-Schleife stückweise expandiert wird. Wenn das aktuelle Element
% an der dafür passenden Stelle eingesetzt wurde, wird \cs{if@tud@list@sorted} 
% gesetzt.
%    \begin{macrocode}
  \else%
    \@tud@list@sortedfalse%
    \let\@tempb\@empty%
    \@for\@tempa:=\tud@templist\do{%
%    \end{macrocode}
% Ist der Eintrag bereits erfolgt, wird der verbliebene Teil der Liste 
% angehangen.
% \ToDo{\cs{expandafter}\cs{edef}\cs{expandafter} ist quatsch, oder?}[v2.07]
%    \begin{macrocode}
      \if@tud@list@sorted%
        \expandafter\edef\expandafter\@tempb\expandafter{\@tempb,\@tempa}%
      \else%
%    \end{macrocode}
% Liegt der Eintrag in alphabetischer Reihenfolge vor dem aktuellen der 
% durchlaufenenen Liste, so wird dieser davor eingefügt. Dabei muss der Fall, 
% dass das Element der erste Eintrag in der temporären Liste \cs{@tempb} ist, 
% eine Sonderbehandlung erfolgen. Anschließend wird \cs{@tud@list@sortedtrue} 
% gesetzt.
%    \begin{macrocode}
        \expandafter\ifnum\pdf@strcmp{#1}{\@tempa}<\z@\relax%
          \ifx\@tempb\@empty%
            \expandafter\edef\expandafter\@tempb\expandafter{%
              #1,\@tempa%
            }%
          \else%
            \expandafter\edef\expandafter\@tempb\expandafter{%
              \@tempb,#1,\@tempa%
            }%
          \fi%
          \@tud@list@sortedtrue%
        \else%
%    \end{macrocode}
% Ist Eintrag in alphabetischer Reihenfolge nach dem aktuellen der geprüften 
% Liste, so wird dieser (noch) nicht eingefügt. Der Sonderfall des ersten
% Elementes wird abgedeckt.
%    \begin{macrocode}
          \ifx\@tempb\@empty%
            \expandafter\edef\expandafter\@tempb\expandafter{\@tempa}%
          \else%
            \expandafter\edef\expandafter\@tempb\expandafter{\@tempb,\@tempa}%
          \fi%
        \fi%
      \fi%
    }%
%    \end{macrocode}
% Wurde der Eintrag nach dem Durchlaufen der Liste nicht eingefügt, so erfolgt 
% dies hier am Ende der Liste.
%    \begin{macrocode}
    \if@tud@list@sorted\else%
      \expandafter\edef\expandafter\@tempb\expandafter{\@tempb,#1}%
    \fi%
%    \end{macrocode}
% Die temporäre Liste \cs{@tempb} wird auf \cs{tud@templist} überschrieben.
%    \begin{macrocode}
    \let\tud@templist\@tempb%
  \fi%
}
%    \end{macrocode}
% \end{macro}^^A \if@tud@list@sorted
% \end{macro}^^A \tud@templist
% \end{macro}^^A \tud@list@@sort
% \end{macro}^^A \tud@list@sort
%
% \iffalse
%</class>
%<*package>
% \fi
%
% \subsection{Zusätzliche Pakete für das Paket \pkg{tudscrtutorial}}
%
% Mit dem Paket \pkg{scrwfile} lassen sich Probleme mit zu wenig verfügbaren 
% Streams für das Schreiben externer Datein beheben. Hintergrund ist die
% Verwendung des Paketes \pkg{glossaries} in einem der Tutorials, welches eine
% Vielzahl solcher Streams benötigt. Das Paket \pkg{morewrites} hat keinen 
% Einfluss auf \pkg{glossaries}.
%    \begin{macrocode}
\RequirePackage{scrwfile}[2013/08/05]
%    \end{macrocode}
% Die Umgebungen \env{Preamble}, \env{Trunk} und \env{Hint} sowie deren
% Varianten nutzen das Paket \pkg{filecontents}, um ihren Inhalt in eine
% temporäre Datei zu schreiben und diese direkt mit dem Paket \pkg{listings}
% darzustellen sowie ggf. den Quelltext auszuführen. Beim Überschreiben einer
% Datei wird für gewöhnlich eine Warnung erzeugt. Dies wird mit dem Patch 
% unterbunden.
%    \begin{macrocode}
\RequirePackage{filecontents}[2011/10/08]
\begingroup%
  \catcode`\*=11%
  \catcode`\^^M\active%
  \catcode`\^^L\active\let^^L\relax%
  \catcode`\^^I\active%
  \patchcmd{\filec@ntents}{%
    \@latex@warning@no@line{Overwriting file `\@currdir#1'}%
  }{}{}{\tud@patch@wrn{filec@tents}}%
  \global\let\filec@ntents\filec@ntents%
\endgroup%
%    \end{macrocode}
%
% \iffalse
%</package>
%</body>
%<*option>
% \fi
%
% \subsection{%
%   Optionen für die Ausgabe von \cls{tudscrmanual} und \pkg{tudscrtutorial}%
% }
%
% \begin{option}{final}
% \begin{option}{print}
% \begin{macro}{\tudfinalflag}
% \begin{macro}{\tudprintflag}
% Diese beiden Optionen dienen zur finalen und/oder der unbunten Ausgabe.
%    \begin{macrocode}
\TUD@key{final}[true]{%
  \TUD@set@ifkey{final}{@tempswa}{#1}%
  \ifx\FamilyKeyState\FamilyKeyStateProcessed%
    \if@tempswa%
      \let\tudfinalflag\relax%
    \fi%
  \fi%
}
\TUD@key{print}[true]{%
  \TUD@set@ifkey{print}{@tempswa}{#1}%
  \ifx\FamilyKeyState\FamilyKeyStateProcessed%
    \if@tempswa%
      \let\tudprintflag\relax%
    \fi%
  \fi%
}
%    \end{macrocode}
% Wurde eine oder beide der Optionen \opt{final} oder \opt{print} gesetzt bzw.
% vor einem externen \app{pdflatex}-Aufruf die Flags \cs{tudfinalflag} und/oder
% \cs{tudprintflag} definiert, werden die entsprechenden Einstellungen zur
% finalen Ausgabe und/oder unbunten Druck vorgenommen und anschließend die 
% entsprechenden Optionen unschädlich gemacht.
%    \begin{macrocode}
\AtEndPreamble{%
  \ifdef{\tudfinalflag}{%
    \TUDoptions{ToDo=false}%
    \TUD@key{ToDo}{\FamilyKeyStateProcessed}%
  }{%
    \KOMAoptions{overfullrule}%
  }%
  \ifdef{\tudprintflag}{%
    \ifnum\tud@cd@num>\z@\relax%
      \TUDoptions{cd=true}%
    \fi%
    \TUD@key{cd}{\FamilyKeyStateProcessed}%
    \ifundef{\hypersetup}{}{\hypersetup{hidelinks}}%
  }{}%
}
%    \end{macrocode}
% \end{macro}^^A \tudprintflag
% \end{macro}^^A \tudfinalflag
% \end{option}^^A print
% \end{option}^^A final
%
% \iffalse
%<*class>
% \fi
%
% \subsection{Standardoptionen der Klasse \cls{tudscrmanual}}
%
% Es werden einige Optionen für das Handbuch standardmäßig aktiviert. Dazu 
% werden diese an die entsprechende Elternklasse übergeben.
%    \begin{macrocode}
\PassOptionsToClass{%
  cdfoot=true,%
  chapterpage=true,%
  chapterprefix=true,%
  headings=optiontoheadandtoc,%
  captions=tableheading,%
  numbers=noenddot,%
  cd=color,%
}{\TUD@Class@Parent}
\PassOptionsToPackage{automark}{scrlayer-scrpage}
%    \end{macrocode}
%
% \iffalse
%</class>
%</option>
%<*body>
%<*class>
% \fi
%
% \subsection{Debug-Traces für die Klasse \cls{tudscrmanual}}
%
% Ab und an ist es beim Erstellen des Handbuchs recht sinnvoll, verschiedene 
% Dinge direkt im Logfile zu überprüfen. Hierfür werden folgend einige Befehle 
% zur Ablaufkontrolle definiert.
%
% \begin{macro}{\tracinglabels}
% \changes{v2.05}{2015/10/29}{neu}^^A
% \begin{macro}{\tud@trace@lbl@created@add}
% \changes{v2.05}{2015/10/29}{neu}^^A
% \begin{macro}{\tud@trace@lbl@missing@add}
% \changes{v2.05}{2015/10/29}{neu}^^A
% \begin{macro}{\tud@trace@lbl@created@list}
% \changes{v2.05}{2015/10/29}{neu}^^A
% \begin{macro}{\tud@trace@lbl@missing@list}
% \changes{v2.05}{2015/10/29}{neu}^^A
% Mit \cs{tracinglabels} wird die Nachverfolgung aller erstellten Labels in der 
% Dokumentation aktiviert werden. Die Label werden zum einen direkt im Logfile
% ausgegeben und am Ende des selbigen in einer sortierten Liste ausgegeben. Die 
% Sternversion des Befehls erzeugt keine sortierte Liste. Über das optionale 
% Argument kann mit verschiedenen Schlüsseln zudem eingestellt werden, ob alle
% Label (\val{all}) oder nur die erstellten (\val{created}) beziehungsweise die
% fehlenden (\val{missing}) nachverfolgt werden sollen.
%    \begin{macrocode}
\newcommand*\tud@trace@lbl@created@add[1]{}
\newcommand*\tud@trace@lbl@missing@add[1]{}
\NewDocumentCommand\tracinglabels{!s !O{missing}}{%
  \newcommand*\tud@trace@lbl@created@list{}%
  \newcommand*\tud@trace@lbl@missing@list{}%
  \@tempswafalse%
  \ifstr{#2}{created}{\@tempswatrue}{}%
  \ifstr{#2}{all}{\@tempswatrue}{}%
  \if@tempswa%
    \renewcommand*\tud@trace@lbl@created@add[1]{%
      \typeout{%
        +++++ label created: ##1 on page \thepage%
      }%
      \xifinlist{##1}{\tud@trace@lbl@created@list}{}{%
        \listxadd\tud@trace@lbl@created@list{##1}%
      }%
    }%
    \IfBooleanF{#1}{%
      \AfterEndDocument{%
        \typeout{+++++ labels created (sorted) +++++}%
        \tud@list@sort\tud@trace@lbl@created@list%
        \forlistloop\typeout{\tud@trace@lbl@created@list}%
      }%
    }%
  \fi%
  \@tempswafalse%
  \ifstr{#2}{missing}{\@tempswatrue}{}%
  \ifstr{#2}{all}{\@tempswatrue}{}%
  \if@tempswa%
    \renewcommand*\tud@trace@lbl@missing@add[1]{%
      \typeout{%
        +++++ label missing: ##1 on page \thepage%
      }%
      \xifinlist{##1}{\tud@trace@lbl@missing@list}{}{%
        \listxadd\tud@trace@lbl@missing@list{##1}%
      }%
    }%
    \IfBooleanF{#1}{%
      \AfterEndDocument{%
        \typeout{+++++ labels missing (sorted) +++++}%
        \tud@list@sort\tud@trace@lbl@missing@list%
        \forlistloop\typeout{\tud@trace@lbl@missing@list}%
      }%
    }%
  \fi%
}
\@onlypreamble\tracinglabels
%    \end{macrocode}
% \end{macro}^^A \tud@trace@lbl@missing@list
% \end{macro}^^A \tud@trace@lbl@created@list
% \end{macro}^^A \tud@trace@lbl@missing@add
% \end{macro}^^A \tud@trace@lbl@created@add
% \end{macro}^^A \tracinglabels
% \begin{macro}{\tracingmarkup}
% \changes{v2.05}{2015/10/29}{neu}^^A
% \begin{macro}{\tud@trace@markup}
% \changes{v2.05}{2015/10/29}{neu}^^A
% Mit \cs{tracinglabels} wird die Nachverfolgung aller Aufrufe von
% \cs{Process@@MarkupDeclare} und \cs{Process@@MarkupInline} durch
% \cs{Process@Markup} inklusive all ihrer Argumente aktiviert. Dies ist für die 
% Kontrolle des erzeugten Markups sinnvoll.
%    \begin{macrocode}
\newcommand*\tud@trace@markup[2]{}
\newcommand*\tracingmarkup{%
  \renewcommand*\tud@trace@markup[2]{%
    \typeout{+++++ markup ##1 on page \thepage^^J##2}%
  }%
}
\@onlypreamble\tracingmarkup
%    \end{macrocode}
% \end{macro}^^A \tud@trace@markup
% \end{macro}^^A \tracingmarkup
% \begin{macro}{\tracingbundle}
% \changes{v2.05}{2015/10/29}{neu}^^A
% \begin{macro}{\tud@trace@bdl@add}
% \changes{v2.05}{2015/10/29}{neu}^^A
% \begin{macro}{\tud@trace@bdl@list}
% \changes{v2.05}{2015/10/29}{neu}^^A
% Wird \cs{tracingbundle} in der Präambel verwendet, werden alle direkt oder 
% indirekt durch \cs{Process@Markup} im Handbuch erzeugten Querverweise auf
% Klassen oder Pakete aus dem \TUDScript-Bundle nachverfolgt und am Ende des
% Logfiles ausgegeben. Damit kann im Zweifel übeprüft werden, ob eventuell eine
% falsche Referenz genutzt wurde, wodurch u.\,U. Querverweise oder Einträge in
% Index und Änderungsliste fehlerhaft werden oder erst gar nicht erscheinen.
%    \begin{macrocode}
\newcommand*\tud@trace@bdl@add[1]{}
\newcommand*\tud@trace@bdl@list{}
\newcommand*\tracingbundle{%
  \renewcommand*\tud@trace@bdl@add[1]{%
    \begingroup%
      \def\Class####1{class:####1}%
      \def\Package####1{package:####1}%
      \xifinlist{##1}{\tud@trace@bdl@list}{}{%
        \listxadd\tud@trace@bdl@list{##1}%
      }%
    \endgroup%
  }%
  \AfterEndDocument{%
    \typeout{+++++ bundle list +++++}%
    \forlistloop\typeout{\tud@trace@bdl@list}%
  }%
}
\@onlypreamble\tracingbundle
%    \end{macrocode}
% \end{macro}^^A \tud@trace@bdl@list
% \end{macro}^^A \tud@trace@bfundle
% \end{macro}^^A \tracingbundle
%
% \iffalse
%</class>
%</body>
%</!doc>
%<*body|class&doc>
% \fi
%
% \subsection{Anpassungen für das Paket \pkg{hyperref} und Querverweise}
%
% Das Paket \pkg{hyperref} wird für alle möglichen Links und Querverweise 
% geladen, \pkg{bookmark} verbessert das Erstellen der Outline-Einträge.
%    \begin{macrocode}
\PassOptionsToPackage{hyperindex=false,colorlinks,linkcolor=blue}{hyperref}
\PassOptionsToPackage{open,openlevel=0}{bookmark}
%    \end{macrocode}
% Es folgen einige Befehle, die an das Paket \pkg{hyperref} gekoppelt sind. 
% Sobald das Paket geladen wurde, werden die Makros mit der jeweiligen
% Funktionalität definiert. 
%    \begin{macrocode}
\AfterPackage*{hyperref}{%
%    \end{macrocode}
% Zuerst die Einstellungen für \cls{tudscrmanual} und \pkg{tudscrtutorial}.
%
% \iffalse
%<*!doc>
% \fi
%
% Die Bezeichner für die Abschnittsebenen werden umbenannt.
%    \begin{macrocode}
  \renewcaptionname{ngerman}{\sectionautorefname}{Unterkapitel}%
  \renewcaptionname{ngerman}{\subsectionautorefname}{Abschnitt}%
  \renewcaptionname{ngerman}{\subsubsectionautorefname}{Unterabschnitt}%
%    \end{macrocode}
% \begin{macro}{\tudhyperdef}
% \changes{v2.02}{2014/10/27}{neu}^^A
% \begin{macro}{\tudhyperref}
% \changes{v2.02}{2014/10/27}{neu}^^A
% \begin{macro}{\tud@manualname}
% \changes{v2.05}{2015/1q/19}{neu}^^A
% Diese Befehle dienen sowohl zum Definieren von Textankern im Handbuch als 
% auch  zum Referenzieren auf diese. Bei der Erstellung des Ankers mit 
% \cs{tudhyperdef} wird zusätzlich auch noch ein Label (\cs{label}) erzeugt.
% Der Anker wird mit \cs{Hy@raisedlink}~-- wie es auch bei pkg{hyperref}-Labels 
% geschieht~-- über die Grundlinie gehoben. Die Sternversion sollte nach
% Überschriften verwendet werden und verschiebt die vertikale Position des
% Ankers noch weiter nach oben.
%    \begin{macrocode}
  \newcommand*\tud@manualname{tudscr}%
%<*class>
  \NewDocumentCommand\tudhyperdef{s t' m}{%
    \@tempswafalse%
    \IfBooleanT{#1}{\@tempswatrue\setlength\@tempdima{3\baselineskip}}%
    \IfBooleanT{#2}{\@tempswatrue\setlength\@tempdima{15\baselineskip}}%
    \if@tempswa%
      \raisebox{\@tempdima}[0pt][0pt]{\hyperdef{\tud@manualname}{#3}{}}%
      \par\nobreak\vskip\dimexpr-\parskip-\baselineskip\relax%
      \@afterindentfalse\@afterheading%
    \else%
      \Hy@raisedlink{\hyperdef{\tud@manualname}{#3}{}}\ignorespaces%
    \fi%
    \label{#3}%
    \tud@trace@lbl@created@add{#3}%
  }%
  \newcommand*\tudhyperref[2]{\hyperref{}{\tud@manualname}{#1}{#2}}%
%</class>
%    \end{macrocode}
% Aus einem Tutorial erfolgen alle Querverweise mit \cs{tudhyperref} auf das 
% Anwenderhandbuch \enquote*{tudscr} im übergeordneten Ordner.
%    \begin{macrocode}
%<*package>
  \newcommand*\tudhyperref[2]{%
    \hyperref{../\tud@manualname.pdf}{\tud@manualname}{#1}{#2}%
  }%
%</package>
%    \end{macrocode}
% \end{macro}^^A \tud@manualname
% \end{macro}^^A \tudhyperref
% \end{macro}^^A \tudhyperdef
% \begin{macro}{\autorefname}
% \begin{macro}{\auto@refname}
% \begin{macro}{\auto@@refname}
% Der Befehl \cs{autorefname} dient dazu, den Verweistyp beziehungsweise den 
% Bezeichner des aktuellen~-- oder optional eines speziellen~-- Labels ohne die 
% dazugehörige Nummerierung zu erhalten. Das zugrunde liegende Funktionsprinzip 
% wurde bei \hrfn{http://tex.stackexchange.com/q/33776/}{LaTeX Stack Exchange} 
% vorgestellt und hier übernommen.
%    \begin{macrocode}
  \newcommand*\autorefname[1][current]{%
    \ifstr{#1}{current}{%
      \expandafter\HyPsd@@autorefname\@currentHref\@nil%
    }{%
      \auto@refname\HyPsd@@autorefname{#1}%
    }%
    \unskip\xspace%
  }%
  \newcommand*\auto@refname[2]{%
    \expandafter\ifx\csname r@#2\endcsname\relax%
      ??%
    \else%
      \expandafter\expandafter\expandafter\auto@@refname%
          \csname r@#2\endcsname{}{}{}{}\@nil#1\@nil%
    \fi%
  }%
  \newcommand*\auto@@refname{}%
  \def\auto@@refname#1#2#3#4#5\@nil#6\@nil{#6#4.\@nil}%
%    \end{macrocode}
% \end{macro}^^A \auto@@refname
% \end{macro}^^A \auto@refname
% \end{macro}^^A \autorefname
% Nun die Einstellungen für \cls{tudscrdoc}.
%
% \iffalse
%</!doc>
%<*doc>
% \fi
%
% \begin{macro}{\hypersourcedef}
% \changes{v2.05}{2016/04/26}{neu}^^A
% \begin{macro}{\hypersource@def}
% \changes{v2.05}{2016/04/26}{neu}^^A
% \begin{macro}{\hypersource@ref@codeline}
% \changes{v2.05}{2016/04/26}{neu}^^A
% \begin{macro}{\hypersource@ref@page}
% \changes{v2.05}{2016/04/26}{neu}^^A
% Diese Makros werden verwendet, um im Index respektive in der Änderungsliste 
% mithilfe von \pkg{hyperref} Hyperlinks auf Codezeilen respektive Seiten im 
% Dokument zu setzen. 
%
% Mit \cs{hypersourcedef} wird ein Textanker gesetzt, falls dieser noch nicht 
% definiert wurde, was durch \cs{hypersource@def} abgesichert wird. Als 
% Argument wird hierfür die aktuelle Codezeile verwendet.
%    \begin{macrocode}
  \newcommand*\hypersource@def{}%
  \newcommand*\hypersourcedef[1]{%
    \ifstr{#1}{\hypersource@def}{}{%
      \Hy@raisedlink{\hyperdef{}{source:#1}{}}%
      \xdef\hypersource@def{#1}%
    }%
  }%
%    \end{macrocode}
% Mit \cs{hypersource@ref@\dots} wird auf Anker im Dokument verlinkt. Dies 
% erfolgt indirekt über die Attribute \val{main} bzw. \val{usage}, welche bei
% der Definitionen von Befehlen etc. im Index genutzt werden. Hierfür werden
% die dazugehörigen Makros umdefiniert.
%    \begin{macrocode}
  \newcommand*\hypersource@ref@codeline[2][\@firstofone]{%
    \begingroup%
      \def\@tempa##1{\hyperref{}{}{source:##1}{#1{##1}}}%
      \forcsvlist\@tempa{#2}%
    \endgroup%
  }%
  \newcommand*\hypersource@ref@page[2][\@firstofone]{%
    \begingroup%
      \def\@tempa##1{\hyperlink{page.##1}{#1{##1}}}%
      \forcsvlist\@tempa{#2}%
    \endgroup%
  }%
  \renewcommand*\main[1]{%
    \ifcodeline@index%
      \hypersource@ref@codeline[\underline]{#1}%
    \else%
      \hypersource@ref@page[\underline]{#1}%
    \fi%
  }%
  \renewcommand*\usage[1]{\hypersource@ref@page[\textit]{#1}}%
%    \end{macrocode}
% \end{macro}^^A \hypersource@ref@page
% \end{macro}^^A \hypersource@ref@codeline
% \end{macro}^^A \hypersource@def
% \end{macro}^^A \hypersourcedef
%
% \iffalse
%</doc>
% \fi
%
% Das war's. Damit sind die Ausführungen für das Paket \pkg{hyperref} beendet.
%    \begin{macrocode}
}
%    \end{macrocode}
% Sollte das Paket \pkg{hyperref} nicht geladen werden, wird eine Rückfallebene 
% definiert, wodurch die Klasse dennoch verwendet werden kann.
%    \begin{macrocode}
\TUD@UnwindPackage{hyperref}{%
%    \end{macrocode}
% Zunächst die direkt innerhalb der Dokumentation verwendeten Befehle nebst 
% dazugehöriger Warnung.
%    \begin{macrocode}
%<*!doc>
%<*class>
  \ClassWarningNoLine{tudscrmanual}%
%</class>
%<*package>
  \PackageWarningNoLine{tudscrtutorial}%
%</package>
  {%
    It is strongly recommended to load package `hyperref'. \MessageBreak%
    Nevertheless, essential commands are rudimentarily\MessageBreak%
    defined. At least the package `url' is loaded%
  }%
%</!doc>
%    \end{macrocode}
% \begin{macro}{\hypersetup}
% \begin{macro}{\href}
% \begin{macro}{\phantomsection}
% \begin{macro}{\texorpdfstring}
% \begin{macro}{\nolinkurl}
% Einige vom Paket \pkg{hyperref} definierten Befehle werden vorgehalten.
%    \begin{macrocode}
  \providecommand*\hypersetup[1]{}%
  \providecommand*\href[3][]{#3}%
  \providecommand*\phantomsection{}%
  \providecommand*\texorpdfstring[2]{#1}%
  \RequirePackage{url}[2013/09/16]%
  \providecommand*\nolinkurl[1]{\url{#1}}%
%    \end{macrocode}
% \end{macro}^^A \nolinkurl
% \end{macro}^^A \texorpdfstring
% \end{macro}^^A \phantomsection
% \end{macro}^^A \href
% \end{macro}^^A \hypersetup
%
% \iffalse
%<*!doc>
% \fi
%
% \begin{macro}{\hyperpage}
% \begin{macro}{\autoref}
% Der Befehl \cs{autoref} wird auch in der Sternversion genutzt.
%    \begin{macrocode}
  \providecommand*\hyperpage[1]{#1}%
  \providecommand*\autoref{??\xspace\kernel@ifstar{\ref}{\ref}}%
%    \end{macrocode}
% \end{macro}^^A \autoref
% \end{macro}^^A \hyperpage
% \begin{macro}{\hyperdef}
% \begin{macro}{\hyperref}
% \begin{macro}{\tud@hyperref@a}
% \begin{macro}{\tud@hyperref@b}
% Mit \cs{hyperdef} wird ein ein Label direkt vor dem Text eingefügt.
%    \begin{macrocode}
  \providecommand*\hyperdef[3]{\label{#1.#2}#3}%
%    \end{macrocode}
% Der Befehl \cs{hyperref} existiert in zwei Varianten. Entweder mit einem 
% optionalen und einem obligatorischen Argument oder mit vier Argumenten.
%    \begin{macrocode}
  \providerobustcmd*\hyperref{%
    \kernel@ifnextchar[{\tud@hyperref@a}{\tud@hyperref@b}%
  }%
  \newcommand*\tud@hyperref@a[2][]{#2}%
  \newcommand*\tud@hyperref@b[4]{#4}%
%    \end{macrocode}
% \end{macro}^^A \tud@hyperref@b
% \end{macro}^^A \tud@hyperref@a
% \end{macro}^^A \hyperref
% \end{macro}^^A \hyperdef
% \begin{macro}{\partautorefname}
% \begin{macro}{\chapterautorefname}
% \begin{macro}{\sectionautorefname}
% \begin{macro}{\subsectionautorefname}
% \begin{macro}{\subsubsectionautorefname}
% Die notwendigen, lokalen Bezeichner der Gliederungsebenen.
%    \begin{macrocode}
  \tud@localization@german{\partautorefname}{Teil}%
  \tud@localization@german{\chapterautorefname}{Kapitel}%
  \tud@localization@german{\sectionautorefname}{Unterkapitel}%
  \tud@localization@german{\subsectionautorefname}{Abschnitt}%
  \tud@localization@german{\subsubsectionautorefname}{Unterabschnitt}%
  \tud@localization@english{\partautorefname}{Part}%
  \tud@localization@english{\chapterautorefname}{chapter}%
  \tud@localization@english{\sectionautorefname}{section}%
  \tud@localization@english{\subsectionautorefname}{subsection}%
  \tud@localization@english{\subsubsectionautorefname}{subsubsection}%
%    \end{macrocode}
% \end{macro}^^A \subsubsectionautorefname
% \end{macro}^^A \subsectionautorefname
% \end{macro}^^A \sectionautorefname
% \end{macro}^^A \chapterautorefname
% \end{macro}^^A \partautorefname
% Anschließend folgen die für Anwenderdokumentation eigens definierten Befehle.
%    \begin{macrocode}
%<*class>
  \ProvideDocumentCommand\tudhyperdef{s t' m}{\label{#3}}%
%</class>
  \providecommand*\tudhyperref[2]{#2}%
  \providecommand*\autorefname[1][]{??\xspace}%
%
% \iffalse
%</!doc>
%<*doc>
% \fi
%
%    \end{macrocode}
% Und hier der Teil für die Quelltextdokumentation.
%    \begin{macrocode}
  \providecommand*\hypersourcedef[1]{}%
  \providecommand*\hypersource@def{}%
  \providecommand*\hypersource@ref@codeline[2][]{}%
  \providecommand*\hypersource@ref@page[2][]{}%
%
% \iffalse
%</doc>
% \fi
%
%    \begin{macrocode}
}
%    \end{macrocode}
% \begin{macro}{\hrfn}
% \changes{v2.02}{2014/08/16}{neu}^^A
% \changes{v2.04}{2015/02/18}{Schriftgröße mit \cs{scalebox}}^^A
% Mit diesem Makro können Hyperlinks im Fließtext erzeugt werden, welche 
% zusätzlich noch die dazugeörige URL als Fußnote anzeigen.
%    \begin{macrocode}
\newcommand*\hrfn[2]{%
  \href{#1}{\trim@spaces{#2}}%
  \footnote{%
    \begingroup%
%    \end{macrocode}
% Die Gruppe wird verwendet, um Unterstriche zu entschärfen, die im Zweifel zu 
% Fehlern führen können.
%    \begin{macrocode}
      \def\_{_}%
      \edef\@tempa{#1}%
      \strut\expandafter\url\expandafter{\@tempa}%
    \endgroup%
  }%
}
%    \end{macrocode}
% \end{macro}^^A \hrfn
% \begin{macro}{\mailto}
% Ein Makro zur Angabe einer verlinkten E"~Mail-Adresse.
%    \begin{macrocode}
\newrobustcmd*\mailto[1]{\mbox{\href{mailto:#1}{\nolinkurl{#1}}}}%
%    \end{macrocode}
% \end{macro}^^A \mailto
% Wenn \pkg{hyperref} geladen wird, erzeugt das Paket \pkg{babel} innerhalb von 
% \cs{pdfstringdef} ziemlich viele und absolut unnötige Infos in der Form
% \texttt{Package babel Info: Redefining ngerman shorthand}. Dieser Patch dient
% zum Unterdrücken dieser Infos.
% \begin{macro}{\pdfstringdef}
%    \begin{macrocode}
\AfterPackage*{hyperref}{%
  \patchcmd{\pdfstringdef}{\csname HyPsd@babel@}{%
    \let\bbl@info\@gobble\csname HyPsd@babel@%
  }{}{\tud@patch@wrn{pdfstringdef}}%
}
%    \end{macrocode}
% \end{macro}^^A \pdfstringdef
%
% \iffalse
%</body|class&doc>
%<*body>
% \fi
%
% \begin{macro}{\fullref}
% Verbesserte Referenzierungen auf Seiten.
%    \begin{macrocode}
\RequirePackage{varioref}[2011/10/02]
\renewcommand*\fullref[1]{\hyperref[#1]{\autoref*{#1}\space\vpageref{#1}}}
%    \end{macrocode}
% \end{macro}^^A \fullref
% Das Paket \pkg{microtype} wird für den optischen Randausgleich verwendet. Es 
% sollte erst nach \pkg{hyperref} geladen werden.
%    \begin{macrocode}
\AfterPackage*{hyperref}{\RequirePackage[babel]{microtype}[2013/05/23]}
%    \end{macrocode}
% Wurde \pkg{hyperref} nicht geladen, dann soll auf \pkg{microtype} trotzdem 
% nicht verzichtet werden.
%    \begin{macrocode}
\TUD@UnwindPackage{hyperref}{\RequirePackage[babel]{microtype}[2013/05/23]}
%    \end{macrocode}
%
% \iffalse
%</body>
%<*!doc>
%<*body>
%<*class>
% \fi
%
% \subsection{Deklarationsumgebungen für die Klasse \cls{tudscrmanual}}
%
% Die Umgebungen beiden \env{Declaration} und \env{Declaration*} sowie 
% \env{Obsolete} und \env{Bundle} werden für die Beschreibung von Optionen,
% Umgebungen, Befehlen etc. verwendet, welche von \TUDScript dem Anwender zur 
% Verfügung gestellt werden.
%
% \begin{macro}{\if@tud@declare}
% \begin{macro}{\if@openindex}
% \begin{macro}{\tud@declare@num}
% \changes{v2.05}{2015/11/01}{neu}^^A
% Dies sin ein paar Hilfsmakros zur Steuerung des Markup sowie der Ausgabe von
% Deklarationen, Index und Änderungsliste, welche bei den nächsten Befehlen und 
% Umgebungen genutzt werden. Mit \cs{if@tud@declare} wird festgelegt, ob die
% Befehle für das Markup in der Ausgabe als Deklaration oder anderweiteig
% erfolgt. Der Schalter \cs{if@openindex} wird zu Beginn einer Deklaration auf
% \val{true} gesetzt und vor dem Beenden auf \val{false}. Damit wird bestimmt,
% ob ein öffnender oder ein schließender Indexeintrag erzeugt werden soll.
%    \begin{macrocode}
\newif\if@tud@declare
\newif\if@openindex
%    \end{macrocode}
% Das Makro \cs{tud@declare@num} bestimmt innerhalb einer Deklaration, wie 
% das Markup erfolgen soll. Die möglichen Werte sind bei der Beschreibung von
% \cs{Process@@MarkupDeclare} zu finden.
%    \begin{macrocode}
\newcommand*\tud@declare@num{0}
%    \end{macrocode}
% \end{macro}^^A \tud@declare@num
% \end{macro}^^A \if@openindex
% \end{macro}^^A \if@tud@declare
% \begin{environment}{Declaration}
% \begin{environment}{Obsolete}
% \changes{v2.05}{2015/11/01}{neu}^^A
% \begin{macro}{\tud@declaration@list}
% \begin{macro}{\if@tud@obsolete}
% \changes{v2.05}{2015/11/01}{neu}^^A
% Es werden einige Hilfsmakros für die Ausgabe einer Deklaration sowie der 
% dazugehörigen Änderungsliste initialisiert.
%    \begin{macrocode}
\newcommand*\tud@declaration@list{}
\let\tud@declaration@list\relax
%    \end{macrocode}
% Die \env{Declaration}-Umgebung dient zur Deklaration von Optionen, Umgebungen,
% Befehlen, Parametern, Bezeichnern, Längen, Schriftelementen und Farben. Diese 
% werden mit den entsprechenden Auszeichnungsbefehlen als obligatorisches 
% Argument übergeben. Zuvor kann noch vorher im ersten optionalen Argument eine
% Beschreibung für die Änderungsliste in der Form \oarg{Verison!Beschreibung}
% angegeben werden. Danach kann im dritten Argument~-- das zweite optionale~-- 
% die Voreinstellung für eine Option oder dergleichen angegeben werden. Das
% nächste optionale Argument erlaubt die Angabe einer bedingten Voreinstellung, 
% also einer anderen Voreinstellung als der zuerst angegeben unter bestimmten
% Voraussetzungen. Diese wird in der Form \oarg{Bedingung:Voreinstellung}
% angegeben. Das letzte Argument ist ebenfalls optional, wird aber in der Form
% \marg{Zusatz} angegeben und enthält ggf. zusätzliche Informationen.
%
% Die Umgebung \env{Obsolete} hat eine ganz ähnliche Funktion, wird allerdings
% für die Deklaration veralteter Befehle, Umgebungen, Optionen etc. verwendet.
% Da sich beide Umgebungen stark ähneln, werden zum Start bzw. zum Abschluss
% die Makros \cs{tud@declare@start} bzw. \cs{tud@declare@end} aufgerufen. Zu 
% Beginn wird nichts weiter ausgegeben sondern lediglich alle übergebenen
% Argumente in jeweiligen Listen gesammelt. Die eigentliche Ausgabe der 
% Deklaration erfolgt mit dem Befehl \cs{printdeclarationlist} bezieungsweise 
% \cs{printobsoletelist}. 
%    \begin{macrocode}
\NewDocumentEnvironment{Declaration}{omoog}{%
  \listadd\tud@declaration@list{#2}%
  \tud@declare@start[#1]{#2}[#3][#4]{#5}%
}{%
  \tud@declare@end{#2}{\printdeclarationlist}%
}
%    \end{macrocode}
% Für die Umgebung \env{Obsolete} sieht die Eingabe etwas anders aus. Das erste 
% obligatorische Argument sollte eine Versionsnummer enthalten, ab wann die 
% Deklaration veraltet ist. Das erste optionale Argument wird genutzt, wenn 
% für eine weiterhin gültige Deklaration lediglich ein bestimmter Wert 
% entfällt. Wird das erste Argument leer gelassen, erfolgt lediglich die 
% Ausgabe der Deklaration ohne einen Eintrag in die Änderungsliste. Dies ist 
% für Umgebungen und Befehle gedacht, bei denen lediglich ein oder mehrere 
% Parameter geändert wurden bzw. entfallen.
%    \begin{macrocode}
\newif\if@tud@obsolete
\NewDocumentEnvironment{Obsolete}{momoog}{%
  \@tud@obsoletetrue%
  \ifxblank{#1}{%
%    \end{macrocode}
% Ohne Versionsnummer erfolgt die Ausgabe ohne Label. Für den Eintrag wird auch 
% keine Änderungsnotiz erzeugt.
%    \begin{macrocode}
    \listadd\tud@declaration@list{%
      \def\tud@declare@num{1}%
      #3%
      \def\tud@declare@num{0}%
    }%
    \tud@declare@start{#3}[#4][#5]{#6}%
  }{%
    \listadd\tud@declaration@list{#3}%
%    \end{macrocode}
% Wurde ein optionales Argument angegeben, bezieht sich der Änderungseintrag 
% auf ebendieses Argument, die eigentliche, obsolete Deklaration wird als 
% Untereintrag genutzt.
%    \begin{macrocode}
    \IfValueTF{#2}{%
      \Changed@At@CreateList[#2]{#1!#3}%
    }{%
      \Changed@At@CreateList[#3]{#1}%
    }%
    \IfValueTF{#6}{%
      \tud@declare@start{#3}[#4][#5]{#6}%
%    \end{macrocode}
% Ohne eine zusätzliche Angabe imletzten optionalen Argument wird standardmäßig
% der Entfall der Deklaration angegeben. Alternativ dazu kann im ersten 
% Argument der Versionsangabe mit einem Dopppelpunkt von dieser getrennt eine 
% neue bzw. aktuell gültige Deklaration als Querverweis angegeben werden.
%    \begin{macrocode}
    }{%
      \toks@{\tud@declare@start{#3}[#4][#5]}%
      \def\@tempa{\emph{entf\"allt}}%
      \in@{:}{#1}%
      \ifin@%
        \def\@tempb##1:##2\@nil{%
          \IfArgIsEmpty{##2}{}{%
            \def\@tempa{\seeref{##2'page'}}%
          }%
        }%
        \@tempb#1\@nil%
      \fi%
      \eaddto@hook\toks@{\expandafter{\@tempa}}%
      \the\toks@%
    }%
  }%
  \@tud@obsoletefalse%
}{%
  \@tud@obsoletetrue%
  \tud@declare@end{#3}{\printobsoletelist}%
  \@tud@obsoletefalse%
}
%    \end{macrocode}
% \end{macro}^^A \if@tud@obsolete
% \end{macro}^^A \tud@declaration@list
% \end{environment}^^A Obsolete
% \end{environment}^^A Declaration
% \begin{macro}{\tud@preset@list}
% \begin{macro}{\if@tud@preset@list}
% \begin{macro}{\tud@additional@list}
% \begin{macro}{\if@tud@additional@list}
% Dies sind die temporären listen, die für alle Deklarationen verwendet werden.
%    \begin{macrocode}
\newcommand*\tud@preset@list{}
\let\tud@preset@list\relax
\newif\if@tud@preset@list
\newcommand*\tud@additional@list{}
\let\tud@additional@list\relax
\newif\if@tud@additional@list
%    \end{macrocode}
% \end{macro}^^A \if@tud@additional@list
% \end{macro}^^A \tud@additional@list
% \end{macro}^^A \if@tud@preset@list
% \end{macro}^^A \tud@preset@list
% \begin{macro}{\tud@declare@start}
% \changes{v2.05}{2015/11/01}{neu}^^A
% \begin{macro}{\tud@declare@end}
% \changes{v2.05}{2015/11/01}{neu}^^A
% Dies sind die Makros für die eigentliche Abarbeitung der Deklarationsbefehle 
% zu Beginn und Ende der Umgebungen \env{Declaration} und \env{Obsolete}.
%    \begin{macrocode}
\NewDocumentCommand\tud@declare@start{o m r[] r[] m}{%
%    \end{macrocode}
% Die optional angegebenen Änderungen werden mit \cs{Changed@At@CreateList} in 
% der Liste \cs{tud@changedat@list} gesichert und später sowohl für die
% Randnotiz als auch die Änderungsliste verarbeitet.
%    \begin{macrocode}
  \Changed@At@CreateList[#2]{#1}%
%    \end{macrocode}
% Hier noch die Listen für Voreinstellungen\dots
%    \begin{macrocode}
  \IfValueTF{#3}{%
    \@tud@preset@listtrue%
    \def\@tempa{Voreinstellung: \PValue{#3}}%
    \IfValueT{#4}{%
      \def\@tempb[##1:##2]{##1: \PValue{##2}}%
      \eappto\@tempa{ | \expandonce{\@tempb[#4]}}%
    }%
    \listeadd\tud@preset@list{\expandonce\@tempa}%
  }{%
    \listadd\tud@preset@list{\relax}%
  }%
%    \end{macrocode}
% \dots sowie zusätliche Informationen.
%    \begin{macrocode}
  \IfValueTF{#5}{%
    \@tud@additional@listtrue%
    \listadd\tud@additional@list{(#5)}%
    \in@{\Environment}{#2}%
    \ifin@\listadd\tud@additional@list{\tabularnewline}\fi%
  }{%
    \listadd\tud@additional@list{\relax}%
    \in@{\Environment}{#2}%
    \ifin@%
      \listadd\tud@additional@list{\relax}%
      \listadd\tud@additional@list{\relax}%
    \fi%
  }%
}
%    \end{macrocode}
% Am Ende der Umgebungen wird das übergebenen Hauptargument erneut ausgeführt,
% was zum Beenden der Indexeinträge für die jeweilige Deklaration führt
% (\cs{@openindexfalse}, siehe \cs{Process@Index}). Dafür werden die Makros für
% die Angabe obligatorischer und optionaler Parameter lokal umdefiniert.
%    \begin{macrocode}
\newcommand*\tud@declare@end[2]{%
  \ifx\tud@declaration@list\relax\else%
    \ClassError{tudscrmanual}{\string#2\space is missing}{%
      \string\tud@declaration@list\space is not empty. Did you\MessageBreak%
      forget to print this list with\MessageBreak%
      \string#2?%
    }%
  \fi%
  \vskip-\lastskip%
  \@tud@declaretrue\@openindexfalse#1\@tud@declarefalse%
  \pagebreak[1]%
}
%    \end{macrocode}
% \end{macro}^^A \tud@declare@end
% \end{macro}^^A \tud@declare@start
% \begin{environment}{Declaration*}
% \changes{v2.02}{2014/10/09}{neu}^^A
% \changes{v2.05}{2015/08/04}{Indexaufteilung/-markup für Klassen und Pakete}^^A
% Die Sternversion der Umgebung \env{Declaration*} ist für die vereinfachte 
% Deklaration von Klassen, Paketen etc. gedacht, bei denen keine Ausgabe 
% sondern lediglich Hyperlink, Indexeintrag und Änderungsnotiz erzeugt werden 
% sollen. Hierbei werden die beiden Befehle \cs{tud@declare@special@start} und 
% \cs{tud@declare@special@end} genutzt, wobei der Anker des erzeugten Labels
% nicht erhöht sondern auf der aktuellen Grundlinie erzeugt wird.
%    \begin{macrocode}
\NewDocumentEnvironment{Declaration*}{om}{%
  \tud@declare@special@start[#1]{#2}{2}%
}{%
  \tud@declare@special@end{#2}%
}
%    \end{macrocode}
% \end{environment}^^A Declaration*
% \begin{environment}{Bundle*}
% \changes{v2.05}{2015/11/01}{neu}^^A
% Die Umgebung \env{Bundle*} wird verwendet, wenn ein zusätzliches Paket oder 
% eine weitere Klasse in Ergänzung zu den Hauptklassen dokumentiert wird. 
% Hierfür wird die Umgebung \env{Bundle} erst geöffnet, nachdem mit dem Makro
% \cs{tud@declare@special@start} das obligatorische Argument~-- sprich die 
% Klasse oder das Paket~-- deklariert wird. Die Umgebung \env{Bundle*} sollte 
% dabei direkt nach einer Gliederungsüberschrift genutzt werden, in welcher das 
% zu deklarierende Element (Klasse, Paket etc.) genannt wird, da der Anker des 
% erzeugten Labels weiter nach oben auf die Höhe der Überschirft verschoben
% wird. Zum Abschluss der Deklaration wird \cs{tud@declare@special@end} nach
% dem Schließen der Umgebung \env{Bundle} genutzt. 
%    \begin{macrocode}
\NewDocumentEnvironment{Bundle*}{om}{%
  \tud@declare@special@start[#1]{#2}{3}%
  \Bundle{#2}%
}{%
  \endBundle%
  \tud@declare@special@end{#2}%
}
%    \end{macrocode}
% \end{environment}^^A Bundle*
% \begin{macro}{\tud@declare@special@start}
% \changes{v2.05}{2015/11/01}{neu}^^A
% \begin{macro}{\tud@declare@special@end}
% \changes{v2.05}{2015/11/01}{neu}^^A
% Die beiden dienen zur vereinfachten Deklaration. Es wird keine Ausgabe 
% sondern lediglich Hyperlink, Indexeintrag und Änderungsnotiz erzeugt. Das 
% optionale Argument dient einem Änderungseintrag, das erste obligaotrische 
% gilt der eigentlichen Deklaration, das zweite zur Einstellung des Makros 
% \cs{tud@declare@num}, mit welchem das Verhalten für das Erzeugen der Labels 
% gesteuert wird.
%    \begin{macrocode}
\NewDocumentCommand\tud@declare@special@start{o m m}{%
  \Changed@At@CreateList(#2){#1}%
  \def\tud@declare@num{#3}%
  \@tud@declaretrue\@openindextrue#2\@tud@declarefalse%
  \def\tud@declare@num{0}%
}
\newcommand*\tud@declare@special@end[1]{%
  \ifx\tud@changedat@list\relax\else%
    \ClassError{tudscrmanual}{\string\printchangedatlist\space is missing}{%
      \string\tud@changedat@list\space is not empty. Did you\MessageBreak%
      forget to print this list with \string\printchangedatlist?%
    }%
  \fi%
  \@tud@declaretrue\@openindexfalse#1\@tud@declarefalse%
}
%    \end{macrocode}
% \end{macro}^^A \tud@declare@special@end
% \end{macro}^^A \tud@declare@special@start
% \begin{macro}{\printdeclarationlist}
% \begin{length}{\tud@lastskip}
% \changes{v2.05}{2015/11/01}{neu}^^A
% \begin{macro}{\index}
% \changes{v2.05}{2015/11/01}{neu}^^A
% \begin{macro}{\label}
% \changes{v2.05}{2015/11/01}{neu}^^A
% Mit dem Befehl \cs{printdeclarationlist} erfolgt die eingentliche Ausgabe 
% aller Deklarationen. Zweck ist es, mehrere \env{Declaration}-Umgebungen 
% ineinander verschachteln zu können und eine Ausgabe aller auf einmal zu 
% erzeugen. Die durch die ggf. nacheinander folgenden Deklarationen wurden in 
% entsprechenden Listen gesammelt und werden hier jetzt formatiert ausgegben.
%
% Da die Verwendung der Befehle \cs{index} und \cs{label} nach Überschriften 
% dazu führt, dass \cs{addvspace} nicht mehr richtig verendet werden kann, 
% wird hier etwas gebastelt, damit das trotzdem funktioniert. Dabei wird 
% einfach bei der verwendung der genannten Befehle der zuvor gesetzte vertikale 
% Abstand in \cs{tud@lastskip} gesichert.
%    \begin{macrocode}
\newskip\tud@lastskip
\pretocmd{\index}{\tud@lastskip=\lastskip}{}{\tud@patch@wrn{index}}
\pretocmd{\label}{\tud@lastskip=\lastskip}{}{\tud@patch@wrn{label}}
\NewDocumentCommand\printdeclarationlist{!s !d()}{%
  \ifx\tud@declaration@list\relax\else%
    \ifhmode%
      \vskip\medskipamount%
    \else%
%    \end{macrocode}
% Im vertikalen modus wird dann einfach vom eigentlich gewünschten Abstand der 
% durch \cs{label} bzw. \cs{index} erzwungene vertikale Freiraum abgezogen. 
% Sollte dann noch ein positiver Wert bestehen, wird dieser einfach zusätzlich
% gesetzt.
%    \begin{macrocode}
      \@tempskipa=\glueexpr\medskipamount-\tud@lastskip\relax%
      \ifdim\@tempskipa>\z@\relax%
        \addpenalty{\@beginparpenalty}%
        \addvspace{\@tempskipa}%
      \fi%
    \fi%
%    \end{macrocode}
% Anschließend wird die Länge auf jeden Fall zurückgesetzt.
%    \begin{macrocode}
    \global\tud@lastskip=\z@%
    \@afterindentfalse\@afterheading%
    \@tud@declaretrue\@openindextrue%
%    \end{macrocode}
% Die deklarierten Optionen, Umgebungen, Befehle etc. werden umrahmt. Dabei
% erfolgt die Ausgabe mithilfe einer Box, um die Größe des Deklarationsrahmens
% zu speichern und die Änderungsmarkierung am Seitenrand auf die richtige Höhe
% zu platzieren. Da innerhalb von Tabellen der Wert von \cs{baselineskip} auf
% \makeatletter\the\z@\makeatother~gesetzt wird, muss für einen richtig
% platzierten Link dieser in \cs{HyperRaiseLinkDefault} gesichert werden.
%    \begin{macrocode}
    \def\@tempa##1{\ignorespaces##1\tabularnewline}%
    \edef\HyperRaiseLinkDefault{\the\baselineskip}%
    \sbox\z@{%
      \begin{tabular}{|l|}%
        \hline%
        \forlistloop\@tempa{\tud@declaration@list}%
        \hline%
      \end{tabular}%
    }\usebox\z@%
    \@tud@declarefalse%
%    \end{macrocode}
% Danach werden ggf. die Voreinstellungen und Zusatzinformationen in etwas 
% kleinerer Schrift gesetzt.
%    \begin{macrocode}
    \ifboolexpr{bool {@tud@preset@list} or bool {@tud@additional@list}}{%
      \def\@tempa##1{\small\ignorespaces##1\tabularnewline}%
      \hskip1.2em%
      \if@tud@preset@list%
        \begin{tabular}{@{}l@{}}%
          \forlistloop\@tempa{\tud@preset@list}%
        \end{tabular}%
        \hspace{\tabcolsep}%
      \fi%
      \if@tud@additional@list%
        \begin{tabular}{@{}l@{}}%
          \forlistloop\@tempa{\tud@additional@list}%
        \end{tabular}%
      \fi%
    }{}%
%    \end{macrocode}
% Alle für die Deklaration verwendeten Listen werden nach ihrer Abarbeitung 
% zurückgesetzt.
%    \begin{macrocode}
    \global\let\tud@declaration@list\relax%
    \global\let\tud@preset@list\relax%
    \global\let\tud@additional@list\relax%
    \global\@tud@preset@listfalse%
    \global\@tud@additional@listfalse%
%    \end{macrocode}
% Die Einträge in der Änderungsliste sowie die dazugehörige Randnotiz werden
% innerhalb der Deklaration mit \cs{printchangedatlist} erzeugt. Das optionale 
% Argument sorgt für die Verschiebung der Randnotiz auf die richtige Höhe, um 
% den Höhenversatz zwischen Randnotiz und Deklarationsrahmen auszugleichen.
%    \begin{macrocode}
    \print@changedatlist{#1}{%
      \dimexpr.5\ht\strutbox+.5\dp\strutbox-.5\ht0-.5\dp0\relax%
    }%
%    \end{macrocode}
% Zum Schluss wird das optionale Argument (in runden Klammern) neben der
% Deklarationsbox ausgegeben.
%    \begin{macrocode}
    \IfValueT{#2}{#2}%
    \par\nobreak%
    \vskip\medskipamount%
    \@afterindentfalse\@afterheading%
  \fi%
}
%    \end{macrocode}
% \end{macro}^^A \label
% \end{macro}^^A \index
% \end{length}^^A \tud@lastskip
% \end{macro}^^A \printdeclarationlist
% \begin{macro}{\printobsoletelist}
% \changes{v2.05}{2015/11/01}{neu}^^A
% Damit werden Deklarationen der Umgebung \env{Obsolete} ausgegeben.
%    \begin{macrocode}
\newcommand*\printobsoletelist{%
  \@tud@obsoletetrue%
  \printdeclarationlist*%
  \@tud@obsoletefalse%
}
%    \end{macrocode}
% \end{macro}^^A \printobsoletelist
% \begin{macro}{\printchangedatlist}
% \changes{v2.05}{2015/11/01}{neu}^^A
% Damit werden Änderungsnotizen der Umgebungen \env{Declaration*}, \env{Bundle} 
% und \env{Bundle*} ausgegeben.
%    \begin{macrocode}
\NewDocumentCommand\printchangedatlist{!s !O{\z@}}{%
  \print@changedatlist{#1}{#2}%
}
%    \end{macrocode}
% \end{macro}^^A \printchangedatlista
%
% \iffalse
%</class>
% \fi
%
% \subsection{Markup von Klassen, Paketen, Optionen und weiteren Elementen}
%
% Es folgen die Definitionen für Befehle und Umgebungen für Klasse und Paket, 
% welche sich überschneidenden und für beide benötigt werden. Allerdings sind 
% diese für Klasse und Paket in ihrer Implementierung teilweise unterschiedlich.
%
% Als erstes werden für Klasse und Paket einige Hilfmakros definiert.
%
% \begin{macro}{\bsc}
% \changes{v2.02}{2014/07/22}{\cs{newrobustcmd} aus \pkg{etoolbox} anstatt
%   \cs{DeclareRobustCommand}}^^A
% Eine einfacher zu verwendende Kurzform.
%    \begin{macrocode}
\newrobustcmd*\bsc{\@backslashchar}
%    \end{macrocode}
% \end{macro}^^A \bsc
% \begin{macro}{\suffix}
% \changes{v2.05}{2015/10/27}{neu}^^A
% Für alle möglichen, zusätzlichen Informationen bei Deklarationen, im Index 
% oder der normalen in Ausgabe im Fließtext wird eine etwas kleinere Schrift 
% verwendet.
%    \begin{macrocode}
\newrobustcmd*\suffix[1]{\begingroup~\scriptsize(#1)\endgroup}
%    \end{macrocode}
% \end{macro}^^A \suffix
% \begin{macro}{\NewExpandableDocumentCommand}
% \changes{v2.02}{2014/11/04}{neu}^^A
% \changes{v2.05i}{2017/03/12}{wird mittlerweile von \pkg{xparse} definiert}^^A
% Für Labels, Index- und Änderungs- sowie PDF-Outline-Einträge etc. müssen ganz
% bestimmte Markup-Befehle durch eine expandierbare Version ersetzt werden. Um 
% dies möglichst einfach zu gestalten, wird \cs{NewExpandableDocumentCommand} 
% definiert.
%    \begin{macrocode}
\providecommand*\NewExpandableDocumentCommand[3]{%
  \NewDocumentCommand#1{#2}{}%
  \DeclareExpandableDocumentCommand#1{#2}{#3}%
}
%    \end{macrocode}
% \end{macro}^^A \NewExpandableDocumentCommand
%
% \subsubsection{Definition der Markup-Befehle}
%
% \changes{v2.05}{2015/11/02}{Markup-Befehle komplett überarbeitet}^^A
%
% Im Folgenden werden allerhand Befehle mit Hilfe des Paketes \pkg{xparse} 
% definiert, um bestimmte Begriffe, Klassen, Pakete, Optionen, Umgebungen, 
% Befehle, Parameter, Bezeichner, Längen, Schriftelemente und Farben speziell
% auszuzeichnen. Diese Befehle lauten für Klasse und Paket aus Gründen der
% Konsistenz zwar gleich, unterschieden sich jedoch ein klein wenig in der
% Implementierung.
%
% Die Sternversion all dieser Befehlen tragen nichts in den Index ein. Für 
% (fast) alle der folgenden Befehle gilt, dass diese mit zwei optionalen
% Argumenten am Ende genutzt werden können. Sollte ein Label existieren, wird 
% automatisch ein Hyperlink erzeugt, welcher mit der optionalen Angabe von
% |'|\meta{Referenzvariante}|'| um einen textuellen Querverweis ergänzt werden
% kann. Mit \verb=|=\meta{Indexmarkup}\verb=|= kann die Erscheinung im Index
% angepasst werden. Bei einigen Befehlen kann zusätzlich zuvor im Markup mit
% dem optionalen Argument \parg{Bundleelement} die Zuweisung auf ein bestimmtes
% Paket oder eine Klasse aus dem \TUDScript-Bundle erfolgen.
%
% Die eigentliche Behandlung erfolgt mit dem Befehl \cs{Process@Markup}, 
% welcher die tatsächliche Auszeichnung im Fließtext bzw. als Deklaration 
% übernimmt. Das Auszeichnungsformat wird dafür zuvor mit \cs{Markup@SetFormat}
% festgelegt. Ein Eintrag in den Index wird mit \cs{Process@Index} realisiert.
% Für die Änderungsliste wird der Befehl \cs{Process@ChangedAt} genutzt, um die 
% entsprechende Formatierung zu gewährleisten, wobei dies nur für die Klasse
% \cls{tudscrmanual} und nicht für das Paket \pkg{tudscrtutorial} gilt.
%
% \begin{macro}{\ProcessorKeyVal}
% \changes{v2.05}{2015/11/01}{neu}^^A
% \begin{macro}{\tud@keyval@error}
% \changes{v2.05}{2015/11/01}{neu}^^A
% Bei Optionen und Parametern können spezielle Werte einfach im Hauptargument 
% durch |=| getrennt angegeben werden. Um diese auszuwerten, wird dieser Befehl
% definiert, welcher als Argumentprozessor verwendet wird.
%    \begin{macrocode}
\newcommand*\ProcessorKeyVal[1]{%
  \begingroup%
    \in@{=}{#1}%
%    \end{macrocode}
% Wird im Argument ein |=| gefunden, so wird alles Darauffolgende als Wert 
% erkannt und dem eigentlichen Hauptargument bei der Ausgabe angehangen. Das
% Makro, welches diesen Argumentprozessor verwendet, erhält als Ausgabe also
% entweder \marg{Schlüssel} oder \marg{Schlüssel}|=|\meta{Wert}|=| falls
% \cs{ProcessorKeyVal} mit \marg{Schlüssel=Wert} verwendet wurde.
%    \begin{macrocode}
    \ifin@%
      \def\@tempa##1=##2\@nil{\toks@{{##1}=##2=}}%
    \else%
      \def\@tempa##1\@nil{\toks@{{##1}}}%
    \fi%
    \@tempa#1\@nil%
    \edef\tud@reserved{%
      \noexpand\endgroup%
      \def\noexpand\ProcessedArgument{\the\toks@}%
    }%
  \tud@reserved%
}
%    \end{macrocode}
% Es gibt auch noch eine interne Variante für ein optioneles Werte-Argument. 
% Dabei sollte vermieden werden, dass im Markup das Argument für einen Wert 
% doppelt angegeben wird. In diesem Fall wird dieser Fehler ausgegeben.
%    \begin{macrocode}
\newcommand*\tud@keyval@error{%
%<*class>
  \ClassError{tudscrmanual}%
%</class>
%<*package>
  \PackageError{tudscrtutorial}%
%</package>
    {Wrong usage of optional argument for value}%
    {There are two optional arguments for a value given!}%
}
%    \end{macrocode}
% \end{macro}^^A \tud@keyval@error
% \end{macro}^^A \ProcessorKeyVal
% \begin{environment}{Bundle}
% \changes{v2.05}{2015/11/01}{neu}^^A
% \begin{macro}{\tud@bdl@curr}
% \changes{v2.05}{2015/08/04}{neu}^^A
% \begin{macro}{\tud@bdl@dflt}
% \changes{v2.05}{2015/11/01}{neu}^^A
% \begin{macro}{\tud@if@bdl}
% \changes{v2.05}{2015/08/04}{neu}^^A
% Die Umgebung \env{Bundle} kann nicht verschachtelt werden und prüft zuerst 
% dementsprechend die Verwendung. Anschließend wird lediglich das Makro 
% \cs{tud@bdl@curr} auf das übergebene Argument gesetzt, um innerhalb der 
% Umgebung erstellte Label und Indexeinträge zu beeinflussen.
%    \begin{macrocode}
\newenvironment{Bundle}[1]{%
  \tud@if@bdl{%
%<*class>
    \ClassError{tudscrmanual}%
%</class>
%<*package>
    \PackageError{tudscrtutorial}%
%</package>
      {Nested environment `Bundle'}{%
      It is not possible to nest this environment, when\MessageBreak%
      a cross-label was given before.%
    }%
  }{%
    \gdef\tud@bdl@curr{#1}%
  }%
  \ignorespaces%
}{%
  \global\let\tud@bdl@curr\tud@bdl@dflt%
  \aftergroup\ignorespaces%
}
%    \end{macrocode}
% Diese Makros werden für das Definieren eines Bundle-Elements und ggf. das 
% Prüfen des selbigen benötigt.
% \ToDo{Suffixe wie Klasse, Paket etc. lokalisieren (locale)}[v2.??]
%    \begin{macrocode}
\newcommand*\tud@bdl@curr{}
\newcommand*\tud@bdl@dflt{tudscr}
\let\tud@bdl@curr\tud@bdl@dflt
\newcommand*\tud@if@bdl[2]{\ifstr{\tud@bdl@curr}{\tud@bdl@dflt}{#2}{#1}}
%    \end{macrocode}
% \end{macro}^^A \tud@if@bdl
% \end{macro}^^A \tud@bdl@dflt
% \end{macro}^^A \tud@bdl@curr
% \end{environment}^^A Bundle
% \begin{macro}{\Application}
% \changes{v2.02}{2014/10/08}{überarbeitet}^^A
% \begin{macro}{\@Application}
% Die Auszeichnung und der Indexeintrag einer Anwendungssoftware.
%    \begin{macrocode}
\NewExpandableDocumentCommand\@Application{s m}{#2}
\NewDocumentCommand\Application{s m !d() !d<> !d||}{%
  \Markup@SetFormat{\sbsfont}%
  \IfValueTF{#4}{%
    \Process@Markup{\Application{#2}}(#3)<#4>%
  }{%
    \Process@Markup{\Application{#2}}(#3)%
  }%
  \Process@Index{#1}{\Application{#2}}[Anwendungssoftware](#3)|#5|%
}
%    \end{macrocode}
% \end{macro}^^A \@Application
% \end{macro}^^A \Application
% \begin{macro}{\Distribution}
% \changes{v2.02}{2014/10/08}{überarbeitet}^^A
% \begin{macro}{\@Distribution}
% Die Auszeichnung und der Indexeintrag einer \LaTeX-Distribution. Das 
% optionale Argument kann für das Anhängen einer Versionsnummer o.\,ä. im 
% Fließtext genutzt werden.
%    \begin{macrocode}
\NewExpandableDocumentCommand\@Distribution{s m}{#2}
\NewDocumentCommand\Distribution{s m !o !d() !d||}{%
  \Markup@SetFormat{\sbnfont}%
  \IfValueTF{#3}{%
    \Process@Markup{\Distribution{#2}}[~#3](#4)%
  }{%
    \Process@Markup{\Distribution{#2}}(#4)%
  }%
  \Process@Index{#1}{\Distribution{#2}}[Distribution](#4)|#5|%
}
%    \end{macrocode}
% \end{macro}^^A \@Distribution
% \end{macro}^^A \Distribution
% \begin{macro}{\Engine}
% \changes{v2.05}{2015/11/04}{neu}^^A
% \begin{macro}{\@Engine}
% \changes{v2.05}{2015/11/04}{neu}^^A
% Die Auszeichnung und der Indexeintrag für ein bestimmtes Textsatzsystem.
%    \begin{macrocode}
\NewExpandableDocumentCommand\@Engine{s m}{#2}
\NewDocumentCommand\Engine{s m !d() !d||}{%
  \Markup@SetFormat{\sbnfont}%
  \Process@Markup{\Engine{\hologo{#2}}}(#3)%
  \Process@Index{#1}{\Engine{#2}}[Textsatzsystem](#3)|#4|%
}
%    \end{macrocode}
% \end{macro}^^A \@Engine
% \end{macro}^^A \Engine
% \begin{macro}{\Path}
% Pfade werden ohne zusätzliches Markup ausgegeben.
%    \begin{macrocode}
\newrobustcmd*\Path[1]{\mbox{\texttt{#1}}}
%    \end{macrocode}
% \end{macro}^^A \Path
% \begin{macro}{\File}
% \changes{v2.02}{2014/10/08}{überarbeitet}^^A
% \begin{macro}{\@File}
% Die Auszeichnung und der Indexeintrag einer Datei.
%    \begin{macrocode}
\NewExpandableDocumentCommand\@File{s m}{#2}
\NewDocumentCommand\File{s m !d() !d||}{%
  \Markup@SetFormat{\sbnfont}%
  \Process@Markup{\File{#2}}(#3)%
  \Process@Index{#1}{\File{#2}}[Datei](#3)|#4|%
}
%    \end{macrocode}
% \end{macro}^^A \@File
% \end{macro}^^A \File
% \begin{macro}{\Class}
% \changes{v2.02}{2014/10/08}{überarbeitet}^^A
% \begin{macro}{\@Class}
% Die Auszeichnung und der Indexeintrag einer Klasse. Das optionale Argument 
% zwischen Apostrophen kann für die Formatierung eines Querverweises im Text
% genutzt werden, das optionale Argument zwischen senkrechten Strichen dient
% der Formatierung des Indexeintrages.
%    \begin{macrocode}
\NewExpandableDocumentCommand\@Class{s m}{#2}
\NewDocumentCommand\Class{s m !d() !d'' !d||}{%
%<*class>
  \if@tud@changedat%
    \Process@ChangedAt{\Class{#2}}[Klasse]%
  \else%
%</class>
    \Markup@SetFormat{\sbnfont}%
    \Process@Markup{\Class{#2}}(#3)'#4'%
    \Process@Index{#1}{\Class{#2}}[Klasse](#3)|#5|%
%<*class>
  \fi%
%</class>
}
%    \end{macrocode}
% \end{macro}^^A \@Class
% \end{macro}^^A \Class
% \begin{macro}{\Package}
% \changes{v2.02}{2014/07/10}{Ausgabe für Änderungsliste hinzugefügt}^^A
% \begin{macro}{\@Package}
% Für die Auszeichnungen von Paketen gelten vorherigen Aussagen äquivalent. 
% Für Pakete wird ergänzend ein Hyperlink auf CTAN erzeugt, wenn für dieses 
% kein Label im Dokument besteht. Genaueres ist der Beschreibung und Definition 
% von \cs{Process@Markup} zu entnehmen. Das optionale Argument in einfachen
% Guillemets dient zum Anpassen des CTAN-Links, welcher normalerweise aus
% obligatorischen Argument generiert wird.
%    \begin{macrocode}
\NewExpandableDocumentCommand\@Package{s m}{#2}
\NewDocumentCommand\Package{s m !d() !d<> !d'' !d||}{%
%<*class>
  \if@tud@changedat%
    \Process@ChangedAt{\Package{#2}}[Paket]%
  \else%
%</class>
    \Markup@SetFormat{\sbnfont}%
    \IfValueTF{#4}{%
      \Process@Markup{\Package{#2}}(#3)<#4>'#5'%
    }{%
      \Process@Markup{\Package{#2}}(#3)<#2>'#5'%
    }%
    \Process@Index{#1}{\Package{#2}}[Paket](#3)|#6|%
%<*class>
  \fi%
%</class>
}
%    \end{macrocode}
% \end{macro}^^A \@Package
% \end{macro}^^A \Package
% \begin{macro}{\Option}
% \changes{v2.02}{2014/10/08}{überarbeitet}^^A
% \begin{macro}{\Option@Value}
% \changes{v2.05}{2015/11/02}{neu}^^A
% \begin{macro}{\@Option}
% \changes{v2.02}{2014/11/02}{neu}^^A
% Im Gegensatz zu den vorherigen Befehlen, kann im Hauptargument ein spezieller 
% Wert für eine Option optional durch ein |=| getrennt angegeben werden. Durch
% den Argumentprozessor \cs{ProcessorKeyVal} wird dieses vom eigentlichen
% Schlüssel getrennt. Dabei wird der gegebene Schlüssel immer in der Form
% \marg{Schlüssel} ausgegeben. Sollte ein optionaler Wert gegeben worden sein, 
% wird dieser in der Form |=|\meta{Wert}|=| einfach angehängt und als optionales
% Argument von \cs{Option@Value} weiter verarbeitet. Mit dem optionalen Argument
% in runden Klammern kann ggf. auf eine Option aus einem  \TUDScript-Paket
% verwiesen werden.
%    \begin{macrocode}
\NewExpandableDocumentCommand\@Option{s m}{#2}
\NewDocumentCommand\Option{s >{\ProcessorKeyVal}m !d== !d() !d'' !d||}{%
  \Option@Value{#1}#2=#3=(#4)'#5'|#6|%
}
%    \end{macrocode}
% Für die interne Verwendung kann das optionale Argument für den Wert auch 
% direkt mit |=|\meta{Wert}|=| angehangen werden. Um die gleichzeitige 
% Verwendung beider Varianten zu unterdrücken, wird in diesem Fall ein Fehler 
% erzeugt.
% \ToDo{%
%   Eigentlich sollte der Wert hier formatiert werden. Wünschenswert wäre die 
%   Angabe von \val{Option=Wert} und \val{Option=<Spezialwert>}, damit das 
%   Gezerre mit der Spezialbehandlung von \cs{PSet} etc. wegfallen könnte,
%   siehe \cs{tud@declare@start}
% }[v2.??]%
%    \begin{macrocode}
\NewDocumentCommand\Option@Value{m m d== d== d() d'' d||}{%
  \IfValueT{#4}{\tud@keyval@error}%
%<*class>
  \if@tud@changedat%
    \Process@ChangedAt{\Option{#2}}=#3=[Option](#5)%
  \else%
%</class>
    \Markup@SetFormat{\ttfamily}%
    \Process@Markup{\Option{#2}}=#3=(#5)'#6'%
    \Process@Index{#1}{\Option{#2}}=#3=(#5)|#7|%
%<*class>
  \fi%
%</class>
}
%    \end{macrocode}
% \end{macro}^^A \@Option
% \end{macro}^^A \Option@Value
% \end{macro}^^A \Option
% \begin{macro}{\Environment}
% \changes{v2.02}{2014/10/08}{überarbeitet}^^A
% \begin{macro}{\@Environment}
% \changes{v2.02}{2014/11/02}{neu}^^A
% Bei diesem Makro dient das optionale Argument für die Ausgabe der möglichen 
% Umgebungsargumente bzw. -parameter \emph{bei der Deklaration}. 
%    \begin{macrocode}
\NewExpandableDocumentCommand\@Environment{s m}{#2}
\NewDocumentCommand\Environment{s m !o !d() !d'' !d||}{%
%<*class>
  \if@tud@changedat%
    \Process@ChangedAt{\Environment{#2}}[Umgebung](#4)%
  \else%
%</class>
%    \end{macrocode}
% Außerdem wird für den Fall, dass der \cs{Environment}-Befehl innerhalb der
% Umgebung \env{Declaration} verwendet wird, eine spezielle Ausgabe erzeugt. 
% Nur hier kommt das optionale Argument von \cs{Markup@SetFormat} nach dem
% Hauptargument zum Tragen. Die resultierende Ausgabe hat die Gesatlt:
%
% \begin{tabular}{l}%
%   \cs{begin}\marg{Umgebung}\tabularnewline
%   \dots\tabularnewline
%   \cs{end}\marg{Umgebung}\tabularnewline
% \end{tabular}%
%    \begin{macrocode}
    \Markup@SetFormat{\ttfamily}(%
      \ttfamily\bsc{}begin\textbraceleft#2\textbraceright\IfValueT{#3}{#3}%
      \tabularnewline\ttfamily\dots%
      \tabularnewline\ttfamily\bsc{}end\textbraceleft#2\textbraceright%
    )%
    \Process@Markup{\Environment{#2}}[#3](#4)'#5'%
    \Process@Index{#1}{\Environment{#2}}[Umgebung](#4)|#6|%
%<*class>
  \fi%
%</class>
}
%    \end{macrocode}
% \end{macro}^^A \@Environment
% \end{macro}^^A \Environment
% \begin{macro}{\Macro}
% \changes{v2.02}{2014/10/08}{überarbeitet}^^A
% \begin{macro}{\@Macro}
% \changes{v2.02}{2014/11/02}{neu}^^A
% Die Auszeichnung und der Indexeintrag eines Befehls. Das ordinäre optionale 
% Argument ist für das Anhängen von Parametern o.\,ä. nach dem eigentlichen 
% Makro zu verwenden. Das optionale Argument in runden Klammern dient dem
% Hyperlink zu einem Befehl aus einem anderen Paket oder einer anderen Klasse
% aus dem \TUDScript-Bundle.
%    \begin{macrocode}
\NewExpandableDocumentCommand\@Macro{s m}{#2}
\NewDocumentCommand\Macro{s m !o !d() !d'' !d||}{%
%<*class>
  \if@tud@changedat%
    \Process@ChangedAt{\Macro{#2}}[Befehl](#4)%
  \else%
%</class>
    \Markup@SetFormat[\bsc]{\ttfamily}%
    \Process@Markup{\Macro{#2}}[#3](#4)'#5'%
    \Process@Index{#1}{\Macro{#2}}(#4)|#6|%
%<*class>
  \fi%
%</class>
}
%    \end{macrocode}
% \end{macro}^^A \@Macro
% \end{macro}^^A \Macro
% \begin{macro}{\Length}
% \changes{v2.02}{2014/10/08}{überarbeitet}^^A
% \begin{macro}{\@Length}
% Die Auszeichnung und der Indexeintrag einer \LaTeX-Länge.
%    \begin{macrocode}
\NewExpandableDocumentCommand\@Length{s m}{#2}
\NewDocumentCommand\Length{s m !d() !d'' !d||}{%
%<*class>
  \if@tud@changedat%
    \Process@ChangedAt{\Length{#2}}[L\"ange](#3)%
  \else%
%</class>
    \Markup@SetFormat[\bsc]{\ttfamily}[L\"ange]%
    \Process@Markup{\Length{#2}}(#3)'#4'%
    \Process@Index{#1}{\Length{#2}}[L\"ange](#3)|#5|%
%<*class>
  \fi%
%</class>
}
%    \end{macrocode}
% \end{macro}^^A \@Length
% \end{macro}^^A \Length
% \begin{macro}{\Counter}
% \changes{v2.02}{2014/10/08}{überarbeitet}^^A
% Die Auszeichnung und der Indexeintrag einer \LaTeX-Zählers.
%    \begin{macrocode}
\NewExpandableDocumentCommand\@Counter{s m}{#2}
\NewDocumentCommand\Counter{s m !d() !d'' !d||}{%
%<*class>
  \if@tud@changedat%
    \Process@ChangedAt{\Counter{#2}}[Z\"ahler](#3)%
  \else%
%</class>
    \Markup@SetFormat{\ttfamily}[Z\"ahler]%
    \Process@Markup{\Counter{#2}}(#3)'#4'%
    \Process@Index{#1}{\Counter{#2}}[Z\"ahler](#3)|#5|%
%<*class>
  \fi%
%</class>
}
%    \end{macrocode}
% \end{macro}^^A \Counter
%
% \iffalse
%<*class>
% \fi
%
% \minisec{Exklusive Auszeichnungen für die Klasse}
% Alle weiteren Befehle werden ausschließlich für die Klasse \cls{tudscrmanual}
% definiert.
%
% \begin{macro}{\Key}
% \changes{v2.02}{2014/10/08}{überarbeitet}^^A
% \begin{macro}{\Key@Value}
% \changes{v2.05}{2015/11/02}{neu}^^A
% \begin{macro}{\@Key}
% Die Auszeichnung und der Indexeintrag eines Parameters für Umgebungen und 
% Befehle. Das erste Argument ist die Umgebung oder der Befehl, wofür der 
% Parameter gültig ist. Das zweite Argument ist der Parameter selbst. Die 
% optionale Zuweisung eines Wertes kann äquivalent zu \cs{Option} mit dem
% Trennzeichen~|=| im Hauptargument erfolgen.
% \ToDo{%
%   Eigentlich sollte der Wert hier formatiert werden. Die Angabe von
%   \val{Parameter=Wert} und \val{Parameter=<Spezialwert>} wäre gut, damit
%   das Gezerre mit der Spezialbehandlung von \cs{PSet} etc. wegfallen könnte,
%   siehe \cs{tud@declare@start}
% }[v2.??]%
%    \begin{macrocode}
\NewExpandableDocumentCommand\@Key{s m m}{#2!#3}
\NewDocumentCommand\Key{s m >{\ProcessorKeyVal}m !d== !d() !d'' !d||}{%
  \Key@Value{#1}{#2}#3=#4=(#5)'#6'|#7|%
}
\NewDocumentCommand\Key@Value{m m m d== d== d() d'' d||}{%
  \IfValueT{#5}{\tud@keyval@error}%
  \if@tud@changedat%
    \Process@ChangedAt{\Key{#2}{#3}}=#4=[Parameter](#6)%
  \else%
    \Markup@SetFormat{\ttfamily}[Parameter]%
    \Process@Markup{\Key{#2}{#3}}=#4=(#6)'#7'%
    \Process@Index{#1}{\Key{#2}{#3}}=#4=(#6)|#8|%
  \fi%
}
%    \end{macrocode}
% \end{macro}^^A \@Key
% \end{macro}^^A \Key@Value
% \end{macro}^^A \Key
% \begin{macro}{\Term}
% \changes{v2.02}{2014/10/08}{überarbeitet}^^A
% \begin{macro}{\@Term}
% Die Auszeichnung und der Indexeintrag eines sprachabhängigen Bezeichners.
%    \begin{macrocode}
\NewExpandableDocumentCommand\@Term{s m}{#2}
\NewDocumentCommand\Term{s m !d() !d'' !d||}{%
  \if@tud@changedat%
    \Process@ChangedAt{\Term{#2}}[Bezeichner](#3)%
  \else%
    \Markup@SetFormat[\bsc]{\ttfamily}[Bezeichner]%
    \Process@Markup{\Term{#2}}(#3)'#4'%
    \Process@Index{#1}{\Term{#2}}(#3)|#5|%
  \fi%
}
%    \end{macrocode}
% \end{macro}^^A \@Term
% \end{macro}^^A \Term
% \begin{macro}{\PageStyle}
% \changes{v2.02}{2014/07/25}{neu}^^A
% \begin{macro}{\@PageStyle}
% Die Auszeichnung und der Indexeintrag eines Seitenstils.
%    \begin{macrocode}
\NewExpandableDocumentCommand\@PageStyle{s m}{#2}
\NewDocumentCommand\PageStyle{s m !d() !d'' !d||}{%
  \if@tud@changedat%
    \Process@ChangedAt{\PageStyle{#2}}[Seitenstil](#3)%
  \else%
    \Markup@SetFormat{\ttfamily}[Seitenstil]%
    \Process@Markup{\PageStyle{#2}}(#3)'#4'%
    \Process@Index{#1}{\PageStyle{#2}}[Seitenstil](#3)|#5|%
  \fi%
}
%    \end{macrocode}
% \end{macro}^^A \@PageStyle
% \end{macro}^^A \PageStyle
% \begin{macro}{\Font}
% \changes{v2.02}{2014/10/08}{überarbeitet}^^A
% \begin{macro}{\@Font}
% Die Auszeichnung und der Indexeintrag eines Schriftelementes.
%    \begin{macrocode}
\NewExpandableDocumentCommand\@Font{s m}{#2}
\NewDocumentCommand\Font{s m !d() !d'' !d||}{%
  \if@tud@changedat%
    \Process@ChangedAt{\Font{#2}}[Schriftelement](#3)%
  \else%
    \Markup@SetFormat{\ttfamily}[Schriftelement]%
    \Process@Markup{\Font{#2}}(#3)'#4'%
    \Process@Index{#1}{\Font{#2}}[Schriftelement](#3)|#5|%
  \fi%
}
%    \end{macrocode}
% \end{macro}^^A \@Font
% \end{macro}^^A \Font
% \begin{macro}{\Color}
% \changes{v2.02}{2014/10/08}{überarbeitet}^^A
% \begin{macro}{\@Color}
% Die Auszeichnung und der Indexeintrag einer Farbe des \CDs. Das optionale 
% Argument kann sowohl bei der Deklaration als auch im Fließtext für das 
% Anhängen eines Suffix verwendet werden. Alle mit \cs{Color} ausgezeichneten 
% Farben referenzieren standardmäßig auf das Paket \pkg{tudscrcolor}.
%    \begin{macrocode}
\NewExpandableDocumentCommand\@Color{s m}{#2}
\NewDocumentCommand\Color{s m !o !D(){\Package{tudscrcolor}} !d'' !d||}{%
  \if@tud@changedat%
    \Process@ChangedAt{\Color{#2}}[Farbe](#4)%
  \else%
    \Markup@SetFormat{\ttfamily}[Farbe](%
      \begingroup\ttfamily{#2\IfValueT{#3}{~(#3)}}\endgroup%
    )%
    \Process@Markup{\Color{#2}}[#3](#4)'#5'%
    \Process@Index{#1}{\Color{#2}}[Farbe](#4)|#6|%
  \fi%
}
%    \end{macrocode}
% \end{macro}^^A \@Color
% \end{macro}^^A \Color
%
% \iffalse
%</class>
% \fi
%
% \subsubsection{Zuordnung der Markup-Befehle für Label und Index}
%
% \begin{macro}{\tud@attr@get}
% \changes{v2.05}{2015/11/03}{neu}^^A
% Der Befehl \cs{tud@attr@get} ordnet den einzelnen Markup-Befehlen ihren 
% spezifischen Präfix für ein Label oder in der Sternversion den passenden 
% Index zu. Im ersten obligatorischen Argument wird ein Makro angegeben, 
% welches die Zuordnung enthalten soll, das zweite ist der Markup-Befehl selbst.
%    \begin{macrocode}
\NewDocumentCommand\tud@attr@get{s m m}{%
  \begingroup%
%    \end{macrocode}
% Das temporäre Makro \cs{@tempa} definiert zunächst alle Markup-Befehle in 
% einer Gruppe neu und expandiert beim Aufruf jedes dieser lokal umdefinierten 
% Markup-Befehle~-- abhängig vom Aufruf von \cs{tud@attr@get} mit oder ohne
% Stern~-- entweder zum Markup-Befehl passenden Index oder zum entsprechenden
% Labelpräfix in \cs{@tempb}.
%    \begin{macrocode}
    \def\@tempa##1##2##3{%
      \def##1####1{%
        \edef\@tempb{\IfBooleanTF{#1}{##2}{##3}}%
      }%
    }%
    \@tempa\Application{\jobname}{app}%
    \@tempa\Distribution{\jobname}{dst}%
    \@tempa\Engine{\jobname}{eng}%
    \@tempa\File{files}{fle}%
    \@tempa\Class{files}{cls}%
    \@tempa\Package{files}{pkg}%
    \@tempa\Option{options}{opt}%
    \@tempa\Environment{macros}{env}%
    \@tempa\Macro{macros}{cmd}%
    \@tempa\Length{misc}{len}%
    \@tempa\Counter{misc}{cnt}%
%    \end{macrocode}
% Da \cs{Key} mit zwei Hauptargumenten aufgerufen wird, ist nur die Expansion
% des ersten Argumentes in \cs{@tempb} notwendig.
%    \begin{macrocode}
%<*class>
    \def\Key##1##2{%
      \edef\@tempb{\IfBooleanTF{#1}{macros}{key}}%
    }%
    \@tempa\Term{terms}{term}%
    \@tempa\PageStyle{elements}{pgs}%
    \@tempa\Font{elements}{font}%
    \@tempa\Color{elements}{clr}%
%</class>
%    \end{macrocode}
% Nachdem alle Markup-Befehle lokal redefiniert wurden, wird der nun angegebene 
% Befehl ausgeführt, was zum eigentlichen Definieren von \cs{@tempb} führt.
%    \begin{macrocode}
    #3%
%    \end{macrocode}
% Damit alle Änderungen der temporären Makros lokal bleiben, wird das Ergebnis 
% nach der Gruppe in das erste obligatorische Argument expandiert.
%    \begin{macrocode}
    \edef\tud@reserved{%
      \noexpand\endgroup%
      \def\noexpand#2{\@tempb}%
    }%
  \tud@reserved%
}
%    \end{macrocode}
% \end{macro}^^A \tud@attr@get
%
% \subsubsection{Markup von Variablen, Parametern etc.}
%
% \ToDo{%
%   Verwendung der Befehle überprüfen, evtl. umbennenen. Wozu \cs{PValueName}?%
%   Evtl. sollte \cs{PValue} innerhalb von \cs{PName} lokal geändert werden?
%   Was passiert, wenn auch \cs{PName} von \cs{tud@doifPValue} akzeptiert?
%   Zeile 542--546 in tudscr-hints, 294 in tudscr-packages
% }[v2.??]%
% \ToDo{%
%   nach '=\cs{P(Value)?Name}' in *.tex suchen und ggf. 
%   überarbeiten; \cs{PValueName} umbenennen%
% }[v2.??]%
% \ToDo{%
%   bei der Deklaration evtl. ein Hilfsmakro erstellen und den
%   Default-Eintrag hinterlegen, um ggf. darauf zu prüfen%
% }[v2.??]%
% \ToDo{%
%   neues Makro \cs{PLength} mit \cs{PName\{Längenwert\}}
% }[v2.??]%
% \begin{macro}{\PValue}
% \begin{macro}{\PName}
% \begin{macro}{\PValueName}
% \changes{v2.05}{2015/11/01}{neu}^^A
% \begin{macro}{\PSet}
% \begin{macro}{\PBoolean}
% \begin{macro}{\PBName}
% \changes{v2.02}{2014/11/12}{neu}^^A
% \begin{macro}{\Parameter}
% \begin{macro}{\OParameter}
% \begin{macro}{\LParameter}
% \begin{macro}{\OLParameter}
% \begin{macro}{\PParameter}
% \begin{macro}{\POParameter}
% \changes{v2.03}{2015/01/25}{neu}^^A
% \begin{macro}{\textOR}
% Diese Befehle dienen zum Auszeichnen von obligatorischen und optionalen 
% Parametern und Befehlen oder bestimmten Wertzuweisungen.
%    \begin{macrocode}
\newrobustcmd*\PValue[1]{\mbox{\texttt{#1}}}
\newrobustcmd*\PName[1]{\PValue{\textsl{<#1>}}}
\newrobustcmd*\PValueName[1]{\PName{#1}}
\newrobustcmd*\PSet{\PName{Einstellung}}
\newrobustcmd*\PBoolean{\PName{Ein-Aus-Wert}}
\newrobustcmd*\PBName[1]{\PBoolean\textOR\PName{#1}}
\newrobustcmd*\Parameter[1]{%
  \mbox{\texttt{\textbraceleft}\PName{#1}\texttt{\textbraceright}}%
}
\newrobustcmd*\OParameter[1]{\mbox{\texttt{[}\PName{#1}\texttt{]}}}
\newrobustcmd*\LParameter{\mbox{\texttt{[}\PName{Parameterliste}\texttt{]}}}
\newrobustcmd*\OLParameter[1]{%
  \mbox{\texttt{[}\PName{#1}\textOR\PName{Parameterliste}\texttt{]}}%
}
\newrobustcmd*\PParameter[1]{\mbox{\texttt{\textbraceleft#1\textbraceright}}}
\newrobustcmd*\POParameter[1]{\mbox{\texttt{[#1]}}}
\newrobustcmd*\textOR{\PValue{\,\textbardbl\,}}
%    \end{macrocode}
% \end{macro}^^A \textOR
% \end{macro}^^A \POParameter
% \end{macro}^^A \PParameter
% \end{macro}^^A \OLParameter
% \end{macro}^^A \LParameter
% \end{macro}^^A \OParameter
% \end{macro}^^A \Parameter
% \end{macro}^^A \PBName
% \end{macro}^^A \PBoolean
% \end{macro}^^A \PSet
% \end{macro}^^A \PValueName
% \end{macro}^^A \PName
% \end{macro}^^A \PValue
% \begin{macro}{\tud@doifPValue}
% \changes{v2.05}{2015/11/14}{neu}^^A
% Der Befehl wird von den Makros \cs{tud@lbl@get@curr} und \cs{tud@idx@get}
% genutzt, um bedingten Quelltext auszuführen, falls im ersten obligatorischen 
% Argument entweder Parameter entweder direkt oder aber mit einem der beiden 
% Parameter-Markup-Befehle \cs{PValue} oder \cs{PValueName} angegeben wurde. 
% Alle anderen Auszeichnungsbefehle für Parameter sollen ingnoriert werden.
%    \begin{macrocode}
\newcommand*\tud@doifPValue[3][]{%
  \IfValueT{#2}{%
    \begingroup%
%    \end{macrocode}
% Zunächst wird der Inhalt fast aller ausgezeichneten Parameter unterdrückt.
% Lediglich das Argument der beiden Makros \cs{PValue} und \cs{PValueName} wird 
% durchgereicht.Dies wird bei der Generierung von Labeln sowie der Erstellung
% für Einträge im Index und der Änderungsliste benötigt, um die Angabe von
% speziellen Werten bei Schlüsseln zu erhalten.
%    \begin{macrocode}
      \let\PValue\@firstofone%
      \let\PName\@gobble%
      \let\PValueName\@firstofone%
      \let\PSet\@empty%
      \let\PBoolean\@empty%
      \let\PBName\@gobble%
      \let\Parameter\@gobble%
      \let\OParameter\@gobble%
      \let\LParameter\@empty%
      \let\OLParameter\@gobble%
      \let\PParameter\@gobble%
      \let\POParameter\@gobble%
      \let\textOR\relax%
      \let\emph\@firstofone%
%    \end{macrocode}
% Im optionalen Argument können weitere Angaben gemacht werden, um zum Beispiel 
% weitere lokale Redefinitionen vorzunehmen.
%    \begin{macrocode}
      #1%
%    \end{macrocode}
% Falls tatsächlich ein passendes Argument angegeben wurde, wird der Quelltext 
% des zweiten obligatorischen Argumentes ausgeführt. Es ist zu beachten, dass
% dieser \emph{expandiert} wird!
%    \begin{macrocode}
      \ifxblank{#2}{%
        \let\tud@reserved\endgroup%
      }{%
        \protected@edef\tud@reserved{\noexpand\endgroup#3}%
      }%
    \tud@reserved%
  }%
}
%    \end{macrocode}
% \end{macro}^^A \tud@doifPValue
% \begin{macro}{\Markup@Gobble}
% \changes{v2.05}{2015/11/01}{neu}^^A
% Dieser Befehl definiert alle Markup-Befehle in der Form um, dass diese nur 
% noch das eigentliche Hauptargument unformatiert durchreichen. Dies wird für 
% die Erstellung von Label und Indexeinträgen benötigt. Die Sternversion führt 
% dies auch noch für die Auszeichnung von Parametern etc. durch.
%    \begin{macrocode}
\NewDocumentCommand\Markup@Gobble{s}{%
  \let\Application\@Application%
  \let\Distribution\@Distribution%
  \let\Engine\@Engine%
  \let\File\@File%
  \let\Class\@Class%
  \let\Package\@Package%
  \let\Option\@Option%
  \let\Environment\@Environment%
  \let\Macro\@Macro%
  \let\Length\@Length%
  \let\Counter\@Counter%
%<*class>
  \let\Key\@Key%
  \let\Term\@Term%
  \let\PageStyle\@PageStyle%
  \let\Font\@Font%
  \let\Color\@Color%
%</class>
  \IfBooleanF{#1}{%
    \let\PValue\@firstofone%
    \let\PName\@firstofone%
    \let\PValueName\@firstofone%
    \let\PSet\relax%
    \let\PBoolean\relax%
    \let\PBName\@firstofone%
    \let\Parameter\@firstofone%
    \let\OParameter\@firstofone%
    \let\LParameter\relax%
    \let\OLParameter\@firstofone%
    \let\PParameter\@firstofone%
    \let\POParameter\@firstofone%
    \let\textOR\relax%
    \let\hologoRobust\@firstofone%
    \def\_{-}%
    \protected\def~{~}%
  }%
}
%    \end{macrocode}
% Mit dem zuvor definierten Makro \cs{Markup@Gobble} wird gleich dafür Sorge 
% getragen, dass die Auszeichnungsbefehle für PDF-Outline-Einträge korrekt 
% funktionieren.
%    \begin{macrocode}
\AfterPackage*{hyperref}{%
  \expandafter\pdfstringdefDisableCommands\expandafter{\Markup@Gobble}%
}
%    \end{macrocode}
% \end{macro}^^A \Markup@Gobble
%
% \subsubsection{Erstellung und Validierung von Labeln}
%
% \begin{macro}{\tud@lbl@tmp}
% \changes{v2.05}{2015/11/03}{neu}^^A
% \begin{macro}{\tud@lbl@fmt}
% \changes{v2.05}{2015/11/03}{neu}^^A
% In \cs{tud@lbl@tmp} werden nachfolgend die durch \cs{tud@lbl@get@curr} oder 
% \cs{tud@attr@get} erzeugten (Sub-)Label gespeichert. Mit dem Hilfsmakro 
% \cs{tud@lbl@fmt} wird innerhalb der beiden Befehle \cs{tud@lbl@@create} und 
% \cs{tud@lbl@get@@curr} dafür gesorgt, dass geschützte Leerzeichen sowie
% eingabekodierungsabhängige Zeichen korrekt für ein Label umgesetzt werden.
% Auch durch \cs{NoCaseChange} in Überschriften geschützte Inhalte werden 
% direkt übernommen.
%    \begin{macrocode}
\newcommand*\tud@lbl@tmp{}
\newcommand*\tud@lbl@fmt{%
  \def~{-}%
  \let\IeC\@firstofone%
  \def\"##1{##1e}%
  \def\ss{ss}%
  \def\dots{...}%
  \let\NoCaseChange\@firstofone%
}
%    \end{macrocode}
% \end{macro}^^A \tud@lbl@fmt
% \end{macro}^^A \tud@lbl@tmp
% \begin{macro}{\tud@lbl@get@curr}
% \changes{v2.05}{2015/11/03}{neu}^^A
% \begin{macro}{\tud@lbl@get@@curr}
% \changes{v2.05}{2015/11/03}{neu}^^A
% Für den Befehl \cs{tud@lbl@get@curr} wird das zweite obligatorische 
% Argument (|#3|) für gewöhnlich in der Form
% |{\Bundle|\marg{Klasse/Paket}|:\Markup|\marg{Element}|}| angegeben. Dieses
% wird beim Einlesen direkt an die beiden Argumentprozessoren 
% \cs{SplitArgument}|{2}{:}| gefolgt von \cs{tud@lbl@create} übergeben. Danach
% enthält das zweite obligatorische Argument ein Label der Form
% \meta{Bundlepräfix}|:|\meta{Labelpräfix}|:|\meta{Element}. Auch die Angabe
% eines Makros als zweites obligatorische Argument, welches das Label in der
% richtigen Formatierung enthält, ist möglich.
%    \begin{macrocode}
\NewDocumentCommand\tud@lbl@get@curr{%
  s m >{\tud@lbl@create}>{\SplitArgument{2}{:}}m d==%
}{%
%    \end{macrocode}
% Mit \cs{tud@lbl@get@@curr} wird das im zweiten obligatorischen Argument 
% (|#3|) generierte Label in das im ersten obligatorischen Argument |#2|
% angegebenen Makro gespeichert und ggf. diesem das letzte optionale Argument
% für einen speziellen Wert zu einem Schlüssel angehangen.
%    \begin{macrocode}
  \tud@lbl@get@@curr#2{#3}=#4=%
%    \end{macrocode}
% Die Sternversion von \cs{tud@lbl@get@curr} wird für das Definieren von Labeln 
% bei Deklarationen etc. genutzt. Sollte diese verwendet worden sein, ist die 
% Generierung beendet.
%    \begin{macrocode}
%<*class>
  \IfBooleanF{#1}{%
%    \end{macrocode}
% Andernfalls wird geprüft, ob das erstellte Label überhaupt existiert. Sollte 
% dies nicht der Fall sein wird ein passendes Label für den gegebenen Schlüssel 
% ohne speziellen Wert gesucht.
%    \begin{macrocode}
    \@tempswafalse%
    \tud@if@lbl@exists{#2}{%
      \@tempswatrue%
    }{%
      \IfValueT{#4}{%
        \tud@lbl@get@@curr\tud@reserved{#3}%
        \tud@if@lbl@exists{\tud@reserved}{%
          \let#2\tud@reserved%
          \@tempswatrue%
        }{}%
      }%
    }%
%    \end{macrocode}
% Wurde auch danach noch kein passendes Label gefunden, gibt es u.\,U. eine 
% letzte Möglichkeit, ein passendes Label zu finden. Wird in der Dokumentation 
% gerade ein Paket oder eine sich von den Hauptklassen unterscheidende Klasse
% aus dem \TUDScript-Bundle erläutert (Umgebungen \env{Bundle}, \env{Bundle*}), 
% so wird eventuell auf ein Element aus ebendiesen Hauptklassen referenziert. 
% In diesem Fall wird \cs{tud@lbl@get@dflt} aufgerufen, um den Bundlepräfix
% für das Label entsprechend zu setzen.
%    \begin{macrocode}
    \if@tempswa\else%
      \tud@if@bdl{\tud@lbl@get@dflt#2{#3}=#4=}{}%
    \fi%
  }%
%</class>
}
%    \end{macrocode}
% Mit \cs{tud@lbl@get@@curr} wird das generierte Label in Makro gespeichert, 
% welches beim Aufruf von \cs{tud@lbl@get@curr} angegebenen wurde.
%    \begin{macrocode}
\NewDocumentCommand\tud@lbl@get@@curr{m m d==}{%
  \def#1{#2}%
%    \end{macrocode}
% Sollte optional mit \cs{PValue} bzw. \cs{PValueName} ein spezieller Wert für
% einen Schlüssel übergeben worden sein, wird dieser ggf. angehängt.
%    \begin{macrocode}
  \tud@doifPValue[\tud@lbl@fmt]{#3}{\noexpand\appto\noexpand#1{=#3}}%
}
%    \end{macrocode}
% \end{macro}^^A \tud@lbl@get@@curr
% \end{macro}^^A \tud@lbl@get@curr
% \begin{macro}{\tud@lbl@get@dflt}
% \changes{v2.05}{2015/11/04}{neu}^^A
% \begin{macro}{\tud@lbl@get@@dflt}
% \changes{v2.05}{2015/11/04}{neu}^^A
% Diese beiden Makros werden genutzt, um mit \cs{tud@lbl@get@curr} innerhalb 
% der Umgebungen \env{Bundle} und \env{Bundle*} für den Fall, dass ein 
% gesuchtes Label für das aktuelle Bundle-Element nicht existiert, nach diesem
% Element in den existierenden Labels der Hauptklassen zu suchen.
%
% Dabei wird das bereits generierte Label durch \cs{tud@lbl@get@dflt} in 
% seine drei Hauptbestandteile zur WEiterverarbeitung zerlegt.
%    \begin{macrocode}
\NewDocumentCommand\tud@lbl@get@dflt{m >{\SplitArgument{2}{:}}m r==}{%
  \tud@lbl@get@@dflt{#1}#2=#3=%
}
%    \end{macrocode}
% Durch \cs{tud@lbl@get@@dflt} wird als erstes geprüft, ob der Bundlepräfix 
% bereits dem Standardpräfix für die Hauptklassen entspricht. Nur wenn dies 
% nicht der Fall ist, wird die Labelgenerierung abermals ausgeführt.
%    \begin{macrocode}
\NewDocumentCommand\tud@lbl@get@@dflt{m m m m r==}{%
  \edef\tud@reserved{\expandafter\detokenize\expandafter{\tud@bdl@dflt}}%
  \ifstr{\tud@reserved}{#2}{}{%
%    \end{macrocode}
% Dabei erfolgt der Aufruf logischerweise mit dem richtigen Bundlepräfix. 
%    \begin{macrocode}
    \tud@lbl@get@curr#1{\tud@bdl@dflt:#3:#4}=#5=%
  }%
}
%    \end{macrocode}
% \end{macro}^^A \tud@lbl@get@@dflt
% \end{macro}^^A \tud@lbl@get@dflt
% \begin{macro}{\tud@if@lbl@exists}
% \changes{v2.05}{2015/11/05}{neu}^^A
% Mit dem Befehl \cs{tud@if@lbl@exists} wird eine Fallunterscheidung bezüglich
% der Existenz eines Labels in der Form
% \meta{Bundlepräfix}|:|\meta{Labelpräfix}|:|\meta{Element} durchgeführt. Das
% zu prüfende Label ist in gleicher Weise wie beim Makro \cs{tud@lbl@get@curr}
% anzugeben. Je nachdem, ob ein existierendes Label gefunden wird oder nicht,
% wird das zweite oder das dritte obligatorische Argument ausgeführt.
%    \begin{macrocode}
%<*class>
\NewDocumentCommand\tud@if@lbl@exists{%
  >{\tud@lbl@create}>{\SplitArgument{2}{:}}m%
}{%
  \ifcsdef{r@#1}{%
    \expandafter\@firstoftwo%
  }{%
    \expandafter\@secondoftwo%
  }%
}
%</class>
%    \end{macrocode}
% \end{macro}^^A \tud@if@lbl@exists
% \begin{macro}{\tud@lbl@create}
% \changes{v2.05}{2015/11/03}{neu}^^A
% \begin{macro}{\tud@lbl@@create}
% \changes{v2.05}{2015/11/03}{neu}^^A
% Das Makro \cs{tud@lbl@@create} ist ein Argumentprozessor, welcher als Eingabe 
% genau drei Argumente für das Erstellen eines gültigen Labels erwartet. Er
% wird sowohl von \cs{tud@lbl@get@curr} als auch von \cs{tud@if@lbl@exists}
% verwendet. Vor der Nutzung von \cs{tud@lbl@create} muss das Argument zuvor
% bereits mit \cs{SplitArgument}|{2}{:}| aufgerufen worden sein. Diese
% Eigenheit macht die Definition von \cs{tud@lbl@create} notwendig, um das
% Ergebnis der zuvor vorgenommenen Argumentaufspaltung verarbeiten zu können,
% in dem dieses an \cs{tud@lbl@@create} durchgereicht wird. Eine Integration 
% der Spaltung des Argumentes in \cs{tud@lbl@create} ist aufgrund des 
% mehrmaligen rekursiven Aufrufs von \cs{tud@lbl@create} nicht möglich.
%
% Mit dem Argumentprozessor \cs{tud@lbl@create} wird es ermöglicht, an die 
% beiden Befehle \cs{tud@lbl@get@curr} und \cs{tud@if@lbl@exists} sowohl ein
% einzelnes Makro, welches das Label bereits beinhaltet als auch einen
% Markup-Befehl ggf. mit einem Bundlepräfix zu übergeben. In beiden Fällen
% wird aus dem übergebenen Argument ein eindeutiges Label in der Form 
% \meta{Bundlepräfix}|:|\meta{Labelpräfix}|:|\meta{Element} generiert und als
% resultierendes Argument an das aufrufenden Makro übergeben.
%    \begin{macrocode}
\NewDocumentCommand\tud@lbl@create{m}{\tud@lbl@@create#1}
%    \end{macrocode}
% Es gibt insgesamt vier verschiedene Varianten der Argumentenübergabe an den
% Argumentprozessor \cs{tud@lbl@@create}.
% \begin{description}\let\itshape\slshape
%    \item[\cs{tud@lbl@@create}]\ignorespaces%
%    |{\LabelMakro}{-NoValue-}{-NoValue-}|\strut\newline%
%      Das Label ist bereits in \cs{LabelMakro} enthalten und soll expandiert 
%      werden. Dabei spielt es keine Rolle, ob die durch |:| getrennten
%      Bestandteile innerhalb des Labels schon als einfacher String oder in der 
%      Markupform gegeben sind.
%    \item[\cs{tud@lbl@@create}]\ignorespaces%
%    |{-NoValue-}{\Markup|\marg{Element}|}{-NoValue-}|\strut\newline%
%      Das Label soll für ein Markupelement generiert werden, wobei in jedem
%      Fall der Inhalt aus \cs{tud@bdl@curr} als Bundlepräfix genutzt wird.
%    \item[\cs{tud@lbl@@create}]\ignorespaces%
%    |{\Class/\Package|\marg{\dots}|}|\ignorespaces%
%    |{\Markup|\marg{Element}|}{-NoValue-}|\strut\newline%
%      Das Label soll für ein Markupelement generiert werden. 
%      Dabei wurde explizit ein Paket oder eine Klasse aus dem Bundle angegeben 
%      oder implizit das Makro \cs{tud@bdl@curr} als Argument verwendet,
%      welches innerhalb der Umgebungen \env{Bundle} und \env{Bundle*} auf
%      das/die aktuell beschriebene Paket/Klasse gesetzt wurde.
%    \item[\cs{tud@lbl@@create}]\ignorespaces%
%    \marg{Bundlepräfix}\marg{Labelpräfix}\marg{Element}\strut\newline 
%      Das Label ist bereits vollständig bestimmt und die Argumente werden 
%      entweder als String oder als Markup übergeben. Auch Mischformen sind 
%      möglich. In diesem Fall wird es einfach wieder zusammengesetzt. Diese
%      Variante tritt durch rekursiven Aufruf des Argumentprozessors auf.
% \end{description}
%    \begin{macrocode}
\NewDocumentCommand\tud@lbl@@create{m m m}{%
  \begingroup%
%    \end{macrocode}
% Zu Beginn werden die Markup-Befehle auf ihre expandierbaren Varianten gesetzt 
% und die zusätzlichen Redefinitionen für die Labelgenerierung aktiviert.
%    \begin{macrocode}
    \Markup@Gobble%
    \tud@lbl@fmt%
%    \end{macrocode}
% Sind alle drei Bestandteile bereits vorhanden, wird das Label einfach wieder
% zusammengesetzt.
%    \begin{macrocode}
    \IfValueTF{#3}{%
      \edef\@tempa{#1:#2:#3}%
    }{%
%    \end{macrocode}
% Hier geschieht die eigentliche Erzeugung der Labels. Dabei wird mit 
% \cs{tud@attr@get} der für den jeweiligen Markup-Befehl aus dem zweiten
% obligatorischen Argument spezifische Labelpräfix generiert. Abhängig davon,
% ob im ersten obligatorischen Argument ein spezieller Bundlepräfix angegeben
% wurde, wird dieser entweder verwendet oder aber implizit auf den Inhalt aus
% \cs{tud@bdl@dflt} zurückgegriffen.
%    \begin{macrocode}
      \IfValueTF{#2}{%
        \tud@attr@get\@tempb{#2}%
        \IfValueTF{#1}{%
          \ifxblank{#1}{%
            \edef\@tempa{\tud@bdl@dflt:\@tempb:#2}%
          }{%
            \edef\@tempa{#1:\@tempb:#2}%
          }%
        }{%
          \edef\@tempa{\tud@bdl@curr:\@tempb:#2}%
        }%
      }{%
%    \end{macrocode}
% Wurde lediglich ein Makro angegeben, wird dieses einfach expandiert.
%    \begin{macrocode}
        \edef\@tempa{#1}%
      }%
    }%
%    \end{macrocode}
% Das Ergebnis der Labelgenrierung wurde für alle der behandelten Fälle in 
% \cs{@tempa} geschrieben. Abschließnd wird \cs{ProcessedArgument} nach der 
% Gruppe mit dessen Inhalt definiert. Dabei werden bei der Expansion die 
% catcodes aller Zeichen in Tokens der Klasse \val{other} gewandelt, um diese 
% später sicher vergleichen zu können.
%    \begin{macrocode}
    \edef\@tempc{%
      \noexpand\endgroup%
      \def\noexpand\ProcessedArgument{%
        \expandafter\detokenize\expandafter{\@tempa}%
      }%
    }%
  \@tempc%
}
%    \end{macrocode}
% \end{macro}^^A \tud@lbl@@create
% \end{macro}^^A \tud@lbl@create
%
% \subsubsection{Abarbeitung der Markup-Befehle}
%
% Die Festlegung des individuellen Markups der Einzelnen Elemente erfolgt bei 
% der Definition mit \cs{Markup@SetFormat}. Bei der Nutztung der Markup-Befehle
% im Fließtext wird zum einen das in \cs{Markup@Format} individuell definierte
% Markup umgesetzt und zum anderen wird dabei gleich automatisch ein Eintrag
% für das Element im Index erzeugt.
%
% \begin{macro}{\ifdisposition}
% \changes{v2.06}{2018/08/02}{neu}^^A
% \begin{macro}{\if@tud@disposition}
% \begin{macro}{\tud@sec@format}
% Für Markup oder andere Inhalte, die lediglich im Fließtext jedoch nicht in 
% Überschriften ausgegeben werden, wird der Schalter \cs{if@tud@disposition} 
% und der dazugehörige auswertende Befehl \cs{ifdisposition} definiert. Das
% Setzen des Schalters wird in \cs{tud@sec@format} gepatcht.
%    \begin{macrocode}
\newif\if@tud@disposition
\newcommand*\ifdisposition{%
  \if@tud@disposition%
    \expandafter\@firstoftwo%
  \else%
    \expandafter\@secondoftwo%
  \fi%
}
\pretocmd{\tud@sec@format}{%
  \@tud@dispositiontrue%
}{}{\tud@patch@wrn{tud@sec@format}}
%    \end{macrocode}
% \end{macro}^^A \tud@sec@format
% \end{macro}^^A \if@tud@disposition
% \end{macro}^^A \tud@if@disposition
%
% In der Klasse \cls{tudscrmanual} werden die Markup-Befehle zusätzlich auch 
% als Argument der Umgebung \env{Declaration} genutzt. Für diese Verwendung
% kommt die Definiton von \cs{Markup@Declare} und ggf. \cs{Markup@Suffix} zum 
% Einsatz. 
%
% \begin{macro}{\Process@Markup}
% \changes{v2.02}{2014/10/08}{Markup für Index und Fließtext getrennt}^^A
% \changes{v2.02}{2014/10/09}{Anpassung für Umgebung \env{Declaration*}}^^A
% \changes{v2.05}{2015/11/05}{Neuimplementierung}^^A
% \begin{macro}{\Markup@Suppress}
% \changes{v2.05}{2015/11/05}{neu}^^A
% \begin{macro}{\Markup@@Suppress}
% \changes{v2.05}{2015/11/05}{neu}^^A
% Der Befehl \cs{Process@Markup} ist das Herzstück der Auszeichnungen aller
% Elemente. Das obligatorische Argument ist das auszuzeichnende Element selbst,
% welches im weiteren Verlauf der Abarbeitung expandiert wird. Darauf folgen 
% mehrere optionale Argumente. Mit |=|\meta{Wert}|=| kann ein Wert für einen
% Schlüssel angegeben werden, was aktuell von \cs{Option} und \cs{Key} genutzt
% wird. Darauf folgend kann in einem ordinären optionalen Argument ein Suffix
% für die Fließtextausgabe angegeben werden, was von \cs{Distribution}  sowie
% \cs{Environment}, \cs{Macro} und \cs{Color} verwendet wird.
%
% Mit dem optionalen Argument in runden Klammern kann eine Klasse oder ein 
% Paket aus dem \TUDScript-Bundle angegeben werden, auf welche sich das
% aktuelle auszuzeichnende Element bezieht und ggf. das Label und der
% Indexeintrag dementsprechend erzeugt werden sollen. Verwendet wird es mit: 
% |(\Class|\marg{Klasse}|)| oder |(\Package|\marg{Paket}|)| 
%
% Das optionale Argument in Guillemets wird von \cs{Package} benötigt, um ggf.
% die automatisch generierten Links auf CTAN anzupassen. Zuletzt kann mit 
% |'|\meta{Referenzvariante}|'| der automatisch erzeugte Querverweis in
% verschiedenen Varianten formatiert werden. Der vollständige Funktionsaufruf:
% \newline\begingroup\scriptsize
%   \cs{Process@Markup}\marg{Element}|=|\meta{Wert}|=|\ignorespaces%
%   \oarg{Ausgabesuffix}\parg{Bundleelement}\ignorespaces%
%   |<|\meta{CTAN-Paket}|>||'|\meta{Referenzvariante}|'|\ignorespaces%
% \endgroup
%
% In eingen Fällen ist es beim Aufruf eines Markup-Befehls nicht erwünscht, 
% dass dieser auch tatsächlich ausgegeben wird. Für diesen Fall wird gleich zu
% Beginn der Befehl \cs{Markup@Suppress} definiert, der dies ermöglicht.
%    \begin{macrocode}
\newcommand*\Markup@Suppress{\let\Process@Markup\Markup@@Suppress}
\NewDocumentCommand\Markup@@Suppress{m d== o d() d<> d''}{}
%    \end{macrocode}
% Nun folgt die eigentliche Befehlsdefinition.
%    \begin{macrocode}
\NewDocumentCommand\Process@Markup{m d== o d() d<> d''}{%
%    \end{macrocode}
% Zu Beginn wird mit dem Schalter \cs{if@tud@declare} geprüft, ob die der 
% Markup-Befehl in einer der Deklarationsumgebungen oder anderweitig genutzt 
% wird. Für den ersten Fall erfolgt die Weiterverarbeitung der Argumente mit
% \cs{Process@@MarkupDeclare}, andernfalls kommt \cs{Process@@MarkupInline} zum
% Einsatz. Da beide Makros die gleiche Reihenfolge der Argumente verlangen,
% wird der jeweilige Befehlsaufruf über ein Token-Register realisiert, womit
% das Zusammensetzen der notwendigen Übergabewerte erleichtert wird. 
%
% Die Fallunterscheidung betrifft des Weiteren die Erzeugung des Labels. Für 
% eine Deklaration wird die Sternversion von \cs{tud@lbl@get@curr} verwendet, 
% welche ein Label aus den gegebenen Argumenten zusammensetzt. Ist keine
% Deklaration aktiv, versucht \cs{tud@lbl@get@curr} ein gültiges, schon 
% vorhandenes Label zu  verwenden. Das generierte Label wird anschließend als 
% erstes Argument für das weiterverabeitende Makro genutzt.
%    \begin{macrocode}
%<*class>
  \if@tud@declare%
    \tud@toks@{\Process@@MarkupDeclare}%
    \IfValueTF{#4}{%
      \tud@lbl@get@curr*\tud@lbl@tmp{#4:#1}=#2=%
    }{%
      \tud@lbl@get@curr*\tud@lbl@tmp{\tud@bdl@curr:#1}=#2=%
    }%
  \else%
%</class>
    \tud@toks@{\Process@@MarkupInline}%
    \tud@lbl@get@curr\tud@lbl@tmp{#4:#1}=#2=%
%<*class>
  \fi%
%</class>
  \eaddto@hook\tud@toks@{\expandafter{\tud@lbl@tmp}}%
%    \end{macrocode}
% Danach wird der Inhalt des im ersten Argument übergebenen Markup-Befehls in 
% einer Gruppe ein temporäres Makro expandiert und als zweites Argument für die 
% Weiterverarbeitung genutzt.
%    \begin{macrocode}
  \begingroup%
    \Markup@Gobble*%
    \protected@edef\tud@reserved{%
      \noexpand\endgroup%
      \def\noexpand\tud@reserved{#1}%
    }%
  \tud@reserved%
  \eaddto@hook\tud@toks@{\expandafter{\tud@reserved}}%
%    \end{macrocode}
% Es folgen gegebenenfalls die optionalen Argument |=|\meta{Wert}|=| sowie
% \oarg{Ausgabesuffix}~-- falls diese angegeben wurden.
%    \begin{macrocode}
  \IfValueT{#2}{\addto@hook\tud@toks@{=#2=}}%
  \IfValueT{#3}{\addto@hook\tud@toks@{[#3]}}%
%    \end{macrocode}
% Erfolgt gerade eine Deklaration, wird mit dem Schalter \cs{if@openindex} 
% eigentlich unterschieden, ob ein öffnender oder ein schließender Indexeintrag
% erzeugt werden soll. Dieser wird mit \cs{Process@@Index} erstellt. Um den 
% schließenden Indexeintrag zu erzeugen, ist die Ausführung des Hauptargumentes 
% der Deklarationsumgebung am Ende dieser notwendig. Dabei soll allerdings das
% Markup nicht noch ein weiteres Mal ausgegeben werden. Deshalb wird der
% Schalter hier genutzt, um lediglich zu Beginn einer Deklaration das Markup 
% auszuführen. Dabei Angaben für eine |'|\meta{Referenzvariante}|'| nicht
% erlaubt.
%   \begin{macrocode}
%<*class>
  \if@tud@declare%
    \if@openindex%
      \IfValueT{#6}{%
        \ClassWarning{tudscrmanual}{%
          It is not possible to use any cross-reference\MessageBreak%
          shorthand like '#6' within the arguments of\MessageBreak%
          the environment `Declaration'%
        }%
      }%
%    \end{macrocode}
% Für ein eventuell notwendiges Debugging, wird \cs{Process@@MarkupDeclare} mit
% seinen Argumenten ggf. in das Log-File geschrieben.
%    \begin{macrocode}
      \tud@trace@markup{declare}{\the\tud@toks@}%
    \else%
      \tud@toks@{\relax}%
    \fi%
%    \end{macrocode}
% Findet der Auszeichnungsbefehl außerhalb einer Umgebung für Deklarationen
% Verwendung, werden die optionalen Argumente für einen CTAN-Link 
% |<|\meta{CTAN-Paketname}|>| sowie die Formatierung des Querverweise
% |'|\meta{Referenzvariante}|'| hinzugefügt~-- falls vorhanden. Dies geschieht
% jedoch nur, wenn das Markup nicht innerhalb eines Index erfolgt, da hier
% URL-Hyperlinks und formatierte Querverweise unerwünscht sind. Mit der Angabe
% eines leeren Argumentes |<>| wird gar kein Hyperlink erstellt.
%    \begin{macrocode}
  \else%
%</class>
    \if@tud@index\else%
      \IfValueT{#5}{%
        \ifxblank{#5}{}{\addto@hook\tud@toks@{<http://www.ctan.org/pkg/#5>}}%
      }%
      \IfValueT{#6}{\addto@hook\tud@toks@{'#6'}}%
    \fi%
%<*class>
%    \end{macrocode}
% Auch hier wird ggf. ein Debug-Eintrag ins Log-File geschrieben. 
%    \begin{macrocode}
    \tud@trace@markup{output}{\the\tud@toks@}%
  \fi%
%    \end{macrocode}
% Ganz zum Schluss wird noch für beide Fälle geprüft, ob \parg{Bundleelement} 
% angegeben wurde. Ist dies der Fall, wird zum einen das \meta{Bundleelement}
% zur Überprüfung ggf. in das Log-File geschrieben, zum anderen wird geprüft, 
% ob für das \meta{Element} überhaupt ein Link definiert wurde, der verwendet
% werden soll. Ist dies nicht der Fall, wird eine Warnung ausgegeben. Dies ist 
% möglich, da durch die automatische Labelgenerierung normalerweise alle nicht 
% existierenden Labels mit \meta{Bundlepräfix} auf \cs{tud@bdl@dftl} verweisen
% müssten. Für die wenigen Fälle, in denen gezielt auf einen Querverweis 
% verzichtet werden soll, kann ein leeres Argument \parg{} angegeben werden.
%    \begin{macrocode}
  \if@tud@index\else%
    \IfValueT{#4}{%
      \ifxblank{#4}{}{%
        \tud@trace@bdl@add{#4}%
        \ifstr{#6}{none}{}{%
          \tud@if@lbl@exists{\tud@lbl@tmp}{}{%
            \ClassWarning{tudscrmanual}{%
              No existing label \tud@lbl@tmp\space for #4%
            }%
          }%
        }%
      }%
    }%
  \fi%
%    \end{macrocode}
% Nun erfolgt mit \cs{Process@@MarkupDeclare} bzw. \cs{Process@@MarkupInline} 
% die Ausführung der eigentlichen Auszeichnung.
%    \begin{macrocode}
%</class>
  \the\tud@toks@%
}
%    \end{macrocode}
% \end{macro}^^A \Markup@@Suppress
% \end{macro}^^A \Markup@Suppress
% \end{macro}^^A \Process@Markup
% \begin{macro}{\Process@@MarkupDeclare}
% \changes{v2.05}{2015/11/05}{neu}^^A
% Mit \cs{Process@@MarkupDeclare} wird die formatierte Ausgabe der Deklaration 
% realisiert. Wie diese genau gestaltet ist, wird durch \cs{tud@declare@num}
% bestimmt. Folgende Werte sind möglich:
% \begin{description}
% \item[\val{0}:] Ausgabe von Markup, Erzeugen eines Labels
% \item[\val{1}:] Ausgabe von Markup, kein Label
% \item[\val{2}:] keine Ausgabe, aber Erzeugen eines Labels
% \item[\val{3}:] keine Ausgabe, Erzeugen eines nach oben verschobenen Labels
% \end{description}
% Normalerweise kommt die erste Variante zum Einsatz. Die zweite Variante wird 
% genutzt, wenn mit mit der Umgebung \env{Obsolete} zwar eine Ausgabe aber kein 
% Label erstellt werden soll. Dies wird verwendet, um bei der Beschreibung 
% veralteter Werte oder Parameter diese im Kontext ihrer Option oder Umgebung
% bzw. ihres Befehls anzugeben.
%
% Die dritte Version kommt bei Erklärungen mit der Umgebung \env{Declaration*}
% zum Einsatz, welche im Fließtext verwendet wird und kein Markup, wohl aber
% eine Label für das beschriebene Element erzeugt werden soll. Für die vierte
% Variante, welche die Umgebung \env{Bundle*} betrifft, gilt dies im Prinzip
% gleichermaßen. Allerdings wird diese Umgebung zu Beginn der Deklarationen
% eines Bundleelements direkt nach einer Überschrift verwendet. Hier soll der 
% erzeugte Link direkt auf der Höhe der Überschrift platziert werden.
%
% Das erste Argument enthält das zu definierende Label, gefolgt vom Element, 
% welches ausgezeichnet werden soll. Die tatsächliche Ausgabe erfolgt mit dem 
% Makro \cs{Markup@Output}. An dieses werden die beiden optionalen Argumente 
% |=|\meta{Wert}|=| und \oarg{Ausgabesuffix} direkt weitergereicht.
%
% Vor der eigentlichen Ausführung wird mit \cs{Markup@CheckFormat} geprüft, ob 
% für das auszuzeichnende Element mit \cs{Markup@SetFormat} überhaupt eine 
% Ausgabeformatierung defineirt wurde. Nach der Ausgabe wird diese Formatierung 
% mit \cs{Markup@ClearFormat} zurückgesetzt.
%    \begin{macrocode}
%<*class>
\NewDocumentCommand\Process@@MarkupDeclare{m m d== o}{%
  \Markup@CheckFormat%
  \ifcase\tud@declare@num\relax%
    \tudhyperdef{#1}%
    \Markup@Output{#2}=#3=[#4]%
  \or%
    \Markup@Output{#2}=#3=[#4]%
  \or%
    \tudhyperdef{#1}%
  \or%
    \tudhyperdef*{#1}%
    \def\@tempa##1:##2:##3\@nil{\tudhyperdef*{##3:##2:##3}}%
    \@tempa#1\@nil%
  \fi%
  \Markup@ClearFormat%
}
%</class>
%    \end{macrocode}
% \end{macro}^^A \Process@@MarkupDeclare
% \begin{macro}{\Process@@MarkupInline}
% \changes{v2.05}{2015/11/05}{neu}^^A
% Mit \cs{Process@@MarkupInline} wird die formatierte Ausgabe außerhalb der 
% Deklarationsumgebungen umgesetzt. Die ersten vier Argument entsprechen denen
% des Befehls \cs{Process@@MarkupDeclare} und werden bis auf das Label ebenso an
% \cs{Markup@Output} übergeben, nachdem mit dem Befehl \cs{Markup@CheckFormat} 
% auf eine existierende Markup-Definition geprüft wurde.
% \ToDo{%
%   Hyperlinks auf erläuterte Element innerhalb der Umgenung \env{Declaration}
%   unterbinden? Könnte \cs{tud@declaration@list} genutzt werden?%
% }[v2.07]%
%    \begin{macrocode}
\NewDocumentCommand\Process@@MarkupInline{m m d== o d<> d''}{%
  \Markup@CheckFormat%
%    \end{macrocode}
% Um die aktuelle Definiton der temporären Makros nicht zu ändern, erfolgt die
% formatierte Ausgabe in einer Gruppe. Dabei wird in \cs{@tempc} die minimale 
% Variante gesichert.
%    \begin{macrocode}
  \begingroup%
    \def\@tempc{\Markup@Output{#2}=#3=[#4]}%
%    \end{macrocode}
% Danach wird eine Fallunterscheidung anhand des im ersten Argument übergebenen 
% Labels getroffen. Existiert dieses, soll das Element als Hyperlink ausgegeben 
% werden. Die Definition dafür wird in \cs{@tempa} gespeichert.
%    \begin{macrocode}
%<*class>
    \tud@if@lbl@exists{#1}{%
      \def\@tempa{\hyperref{}{\tud@manualname}{#1}{\@tempc}}%
%    \end{macrocode}
% Anschließend wird \cs{@tempb} ggf. für den erweiterten Querverweis verwendet.
% Wurde ein Format für den Querverweis gesetzt, muss dieses validiert werden. 
% Gültige Kürzel sind |'|\val{auto}|'|, |'|\val{page}|'| und |'|\val{full}|'|. 
% Damit werden die erzeugten Querverweise um den Gliederungsabschnitt, die
% Seite oder beides ergänzt. Außerdem kann mit |'|\val{none}|'| der Hyperlink 
% komplett unterbunden werden. Für den Befehl \cs{Package} kann mit zusätzlich 
% |'|\val{url}|'| noch der CTAN-Link forciert werden, auch wenn eigentlich ein
% Label für das Paket existiert.
%    \begin{macrocode}
      \let\@tempb\relax%
      \IfValueT{#6}{%
        \def\@tempb{%
%<*class>
          \ClassWarning{tudscrmanual}%
%</class>
%<*package>
          \PackageWarning{tudscrtutorial}%
%</package>
          {%
            You've used the unknown cross-reference\MessageBreak%
            shorthand '#6'. Only 'auto', 'page' and 'full'\MessageBreak%
            as well as 'none' \IfValueT{#5}{or 'url' }are valid%
          }%
        }%
        \ifstr{#6}{auto}{\def\@tempb{ in \autoref{#1}}}{%
        \ifstr{#6}{page}{\def\@tempb{ \vpageref{#1}}}{%
        \ifstr{#6}{ppage}{\def\@tempb{ \reftextfaraway{#1}}}{%
        \ifstr{#6}{full}{\def\@tempb{ in \fullref{#1}}}{%
        \ifstr{#6}{none}{\let\@tempa\@tempc\let\@tempb\relax}{}}}}}%
        \IfValueT{#5}{%
          \ifstr{#6}{url}{\def\@tempa{\href{#5}{\@tempc}}\let\@tempb\relax}{}%
        }%
      }%
    }{%
%</class>
%    \end{macrocode}
% Existiert kein Label, so wird einfach die minimale Markup-Variante verwendet. 
% Für Pakete wird ein CTAN-Link erzeugt.
%    \begin{macrocode}
      \let\@tempa\@tempc%
      \IfValueT{#5}{\def\@tempa{\href{#5}{\@tempc}}}%
%    \end{macrocode}
% Ein Querverweis kann logischerweise nicht erstellt werden. Wurde dennoch im 
% letzten optionalen Argument eine |'|\meta{Referenzvariante}|'| für den
% Querverweis angegeben, wird eine Warnung erzeugt.
%    \begin{macrocode}
      \let\@tempb\relax%
      \IfValueT{#6}{%
        \ifstr{#6}{none}{\let\@tempa\@tempc}{%
%<package>        \ifstr{#6}{manual}{\def\@tempa{\tudhyperref{#1}{\@tempc}}}{%
          \def\@tempb{%
%<*class> 
            \ClassWarning{tudscrmanual}%
%</class>
%<*package>
            \PackageWarning{tudscrtutorial}%
%</package>
            {%
%<*class>
              Label #1\MessageBreak%
              is missing, no cross-reference created%
%</class>
%<*package>
              You've used the unknown cross-reference\MessageBreak%
              shorthand '#6'. Only 'none' and 'manual'\MessageBreak%
              are valid%
%</package>
            }%
          }%
%<package>        }%
        }%
      }%
%    \end{macrocode}
% Um im Zweifel nach allen fehlenden Labels im Log-File prüfen zu können, gibt 
% es den passenden Debug-Befehl.
%    \begin{macrocode}
%<*class>
      \tud@trace@lbl@missing@add{#1}%
    }%
%</class>
%    \end{macrocode}
% Ganz zum Schluss erfolgt die Ausgabe als Hyperlink mit optional erweitertem 
% Querverweis nicht, wenn gerade eine Überschrift gesetzt wird.
%    \begin{macrocode}
    \ifdisposition{\@tempc}{\@tempa\@tempb}%
%    \end{macrocode}
% Zum Schluss wird die Formatierung mit \cs{Markup@ClearFormat} zurückgesetzt.
%    \begin{macrocode}
  \endgroup%
  \Markup@ClearFormat%
}
%    \end{macrocode}
% \end{macro}^^A \Process@@MarkupInline
% \begin{macro}{\Markup@Output}
% \changes{v2.05}{2015/11/05}{neu}^^A
% \begin{macro}{\Markup@@Output}
% \changes{v2.05}{2015/11/05}{neu}^^A
% Der Befehl \cs{Markup@Output} wird für die formatierte Markup-Ausgabe von den
% beiden Makros \cs{Process@@MarkupDeclare} sowie \cs{Process@@MarkupInline}
% aufgerufen. Das erste Argument enthält dabei das auszuzeichnende Element.
% Dabei ist die Besonderheit des Makros \cs{Key}\marg{Makro}\marg{Parameter} zu
% beachten, dessen beiden Argumente hier in der Form \marg{Makro!Parameter} 
% übergeben werden. Deshalb wird das erste Argument mit dem Argumentprozessor 
% \cs{SplitArgument}|{1}{!}| getrennt und dann weiter verarbeitet.
%    \begin{macrocode}
\NewDocumentCommand\Markup@Output{>{\SplitArgument{1}{!}}m r== r[]}{%
  \Markup@@Output#1=#2=[#3]%
}
%    \end{macrocode}
% Anschließend wird entweder \cs{Markup@Declare} oder \cs{Markup@Inline}
% aufgerufen~-- abermals abhängig vom Schalter \cs{if@tud@declare}. Auch hier 
% wird für den Befehlsaufruf mit dem dazugehörigen Argument zur Vereinfachung
% ein Token-Register verwendet.
%    \begin{macrocode}
\NewDocumentCommand\Markup@@Output{m m r== r[]}{%
%<*class>
  \if@tud@declare%
    \tud@toks@{\Markup@Declare}%
  \else%
%</class>
    \tud@toks@{\Markup@Inline}%
%<*class>
  \fi%
%</class>
%    \end{macrocode}
% Ist das aktuelle Element \cs{Key} wird nur das zweite Argument ausgegeben. 
% Das erste wird lediglich für Label und Indexeintrag benötigt.
%    \begin{macrocode}
  \IfValueTF{#2}{%
    \def\@tempa{#2}%
  }{%
    \def\@tempa{#1}%
  }%
%    \end{macrocode}
% Die beiden optionalen Argumente werden ggf. in der gleichen Formatierung 
% ausgegeben. 
%    \begin{macrocode}
  \IfValueT{#3}{\appto\@tempa{=#3}}%
  \IfValueT{#4}{\appto\@tempa{#4}}%
  \eaddto@hook\tud@toks@{\expandafter{\@tempa}}%
  \the\tud@toks@%
}
%    \end{macrocode}
% \end{macro}^^A \Markup@@Output
% \end{macro}^^A \Markup@Output
% \begin{macro}{\Markup@Inline}
% \changes{v2.05}{2015/11/05}{neu}^^A
% \begin{macro}{\Markup@Declare}
% \changes{v2.05}{2015/11/05}{neu}^^A
% \begin{macro}{\Markup@@Declare}
% \changes{v2.05}{2015/11/05}{neu}^^A
% Die Ausgabe selbst ist alles andere als spektakulär. Die zur Formatierung 
% verwendeten Makros \cs{Markup@Format} und \cs{Markup@Suffix} werden durch
% \cs{Markup@SetFormat} für jedes Markup-Element individuell definiert.
%    \begin{macrocode}
\newcommand*\Markup@Inline[1]{\mbox{\Markup@Format{#1}}}
%<*class>
\newcommand*\Markup@Declare{}
\newcommand*\Markup@@Declare[1]{\Markup@Format{#1}\Markup@Suffix}
%</class>
%    \end{macrocode}
% \end{macro}^^A \Markup@@Declare
% \end{macro}^^A \Markup@Declare
% \end{macro}^^A \Markup@Inline
% \begin{macro}{\Markup@SetFormat}
% \changes{v2.05}{2015/11/05}{neu}^^A
% \begin{macro}{\Markup@CheckFormat}
% \changes{v2.05}{2015/11/05}{neu}^^A
% \begin{macro}{\Markup@ClearFormat}
% \changes{v2.05}{2015/11/05}{neu}^^A
% \begin{macro}{\Markup@Format}
% \changes{v2.05}{2015/11/05}{neu}^^A
% \begin{macro}{\Markup@Suffix}
% \changes{v2.05}{2015/11/05}{neu}^^A
% Der Befehl \cs{Markup@SetFormat} setzt das gewünschte Format für die jeweilge
% Ausgabe. Dieses wird im ersten Argument angegeben. Das ordinäre optionale 
% Argument kann genutzt werden, wenn bei der Deklaration noch eine genauere 
% Beschreibung oder Spezifizierung erscheinen soll, beispielsweise etwas wie
% \enquote{Umgebung} oder \enquote{Parameter}. Mit dem optionalen Argument in 
% runden Klammern wird es möglich, die Definition von \cs{Markup@Declare}, die 
% ohne eine Angabe der von \cs{Markup@@Declare} entspricht, zu ändern. Dies
% wird insbesondere von \cs{Environment} genutzt, um bei der Deklaration eine
% Tabelle zu erzeugen.
%    \begin{macrocode}
\newcommand*\Markup@Format{}
\undef\Markup@Format
\newcommand*\Markup@Suffix{}
\NewDocumentCommand\Markup@SetFormat{o m o d()}{%
%    \end{macrocode}
% Außerdem ist noch eine weitere Besonderheit zu beachten. Sollte gerade eine
% Überschrift gesetzt werden, wird das Hauptargument nicht formatiert sondern 
% lediglich mit \cs{tud@sec@format} ausgegeben.
%    \begin{macrocode}
  \def\Markup@Format##1{%
    \def\tud@res@a{#2}%
    \tud@ifdin{\let\tud@res@a\tud@sec@format}{}%
    \begingroup\tud@res@a{\IfValueT{#1}{#1}##1}\endgroup%
  }%
  \IfValueT{#3}{\def\Markup@Suffix{\suffix{#3}}}%
%<*class>
  \IfValueT{#4}{\renewcommand*\Markup@Declare[1]{#4\Markup@Suffix}}%
%</class>
}
%    \end{macrocode}
% Nach der formatierten Ausgabe wird \cs{Markup@ClearFormat} aufgerufen, um die 
% aktuell definierte Formatierung zurückzusetzen und bei einem neuen Aufruf 
% eines der Markup-Befehle mit \cs{Markup@CheckFormat} auf die Verwendung von 
% \cs{Markup@SetFormat} prüfen zu können.
%    \begin{macrocode}
\newcommand*\Markup@ClearFormat{%
  \undef\Markup@Format%
  \let\Markup@Declare\Markup@@Declare%
  \let\Markup@Suffix\relax%
}
\Markup@ClearFormat
\newcommand*\Markup@CheckFormat{%
  \ifundef\Markup@Format{%
%<*class>
    \ClassError{tudscrmanual}%
%</class>
%<*package>
    \PackageError{tudscrtutorial}%
%</package>
      {\string\Markup@SetFormat\space unused}%
      {It seems you have forgotten to use \string\Markup@SetFormat.}%
  }{}%
}
%    \end{macrocode}
% \end{macro}^^A \Markup@Suffix
% \end{macro}^^A \Markup@Format
% \end{macro}^^A \Markup@ClearFormat
% \end{macro}^^A \Markup@SetFormat
% \end{macro}^^A \Markup@CheckFormat
% \begin{macro}{\tud@nonchangecase}
% \changes{v2.05}{2015/11/20}{neu}^^A
% \begin{macro}{\tud@x@textcase@uclcnotmath}
% Die \TUDScript-Hauptklassen definieren in Ergänzung zum Paket \pkg{textcase} 
% den internen Befehl \cs{tud@x@textcase@uclcnotmath}, womit die Definition von 
% Makros ermöglicht wird, welche von \cs{MakeTextUppercase} nicht beachtet
% werden sollen. Hierfür werden \emph{alle} Markup-Befehle der Liste zum
% Ignorieren hinzugefügt. Damit aber bei der eigentlichen Ausgabe alles in
% Majuskeln erscheint, wird für diesen Fall \cs{Markup@Format} genutzt.
%    \begin{macrocode}
\ifundef{\tud@x@textcase@uclcnotmath}{}{%
  \NewDocumentCommand\tud@nonchangecase{m}{%
    \DeclareExpandableDocumentCommand#1{s m}{%
      \IfBooleanT{##1}{\noexpand\@tud@indextrue}%
      \NoCaseChange{#1{##2}}%
    }%
  }%
  \apptocmd{\tud@x@textcase@uclcnotmath}{%
    \tud@nonchangecase\Application%
    \tud@nonchangecase\Distribution%
    \tud@nonchangecase\Engine%
    \tud@nonchangecase\File%
    \tud@nonchangecase\Class%
    \tud@nonchangecase\Package%
    \tud@nonchangecase\Option%
    \tud@nonchangecase\Environment%
    \tud@nonchangecase\Macro%
    \tud@nonchangecase\Length%
    \tud@nonchangecase\Counter%
%<*class>
    \DeclareExpandableDocumentCommand\Key{s m m}{%
      \IfBooleanT{##1}{\noexpand\@tud@indextrue}%
      \NoCaseChange{#1{##2}{##3}}%
    }%
    \tud@nonchangecase\Term%
    \tud@nonchangecase\PageStyle%
    \tud@nonchangecase\Font%
    \tud@nonchangecase\Color%
%</class>
  }{}{\tud@patch@wrn{tud@x@textcase@uclcnotmath}}%
}
%    \end{macrocode}
% \end{macro}^^A \tud@x@textcase@uclcnotmath
% \end{macro}^^A \tud@nonchangecase
%
% \subsection{Index}
%
% Hier erscheint alles, was für Erstellen und Ausgabe der einzelnen Indexe 
% notwendig ist. Hierfür wird das Paket \pkg{imakeidx} in Verbindung mit 
% \app{texindy} verwendet. Diese können sowohl mit dem Paket als auch mit der 
% Klasse erzeugt werden. 
%    \begin{macrocode}
\PassOptionsToPackage{xindy,splitindex}{imakeidx}
\RequirePackage{imakeidx}[2013/07/11]
%    \end{macrocode}
% Zuerst ein paar kleinere Einstellungen für das Layout\dots
%    \begin{macrocode}
\indexsetup{%
%<*class>
  level=\addsec,%
%</class>
%<*package>
  level=\subsection*,%
%</package>
  noclearpage,firstpagestyle=headings,headers={\indexname}{\indexname},%
  othercode={\renewcommand*\subitem{\@idxitem\hspace*{15\p@}}}%
}
%    \end{macrocode}
% \begin{environment}{theindex}
% \dots gefolgt von einem Patch für die Umgebung \env{theindex}, um unschöne 
% Seitenumbrüche direkt nach Überschriften zu vermeiden. Diese treten auf, das 
% das Paket \pkg{imakeidx} meiner Meinung nach die Überschriften zusammen mit 
% der \env{multicols}-Umgebung falsch setzt. Der Patch sorgt dafür, dass die
% Überschriften (\cs{imki@indexlevel}|{|\cs{indexname}|}|) nicht 
% vor der Umgebung sondern~-- wie für \pkg{multicol} auch dokumentiert~-- im
% optionalen Argument von \env{multicols} aufgerufen werden.
%    \begin{macrocode}
\ifimki@original\else%
  \CheckCommand\theindex{%
    \imki@maybeaddtotoc
    \imki@indexlevel{\indexname}\imki@indexheaders
    \thispagestyle{\imki@firstpagestyle}%
    \ifnum\imki@columns>\@ne
      \columnsep \imki@columnsep
      \ifx\imki@idxprologue\relax
        \begin{multicols}{\imki@columns}
      \else
        \begin{multicols}{\imki@columns}[\imki@idxprologue]
      \fi
    \else
      \imki@idxprologue
    \fi
    \global\let\imki@idxprologue\relax
    \parindent\z@
    \parskip\z@ \@plus .3\p@\relax
    \columnseprule \ifKV@imki@columnseprule.4\p@\else\z@\fi
    \raggedright
    \let\item\@idxitem
    \imki@othercode%
  }%
  \patchcmd{\theindex}{%
    \imki@indexlevel{\indexname}\imki@indexheaders%
  }{%
    \imki@indexheaders%
  }{}{\tud@patch@wrn{theindex}}%
  \patchcmd{\theindex}{%
    \ifnum\imki@columns>\@ne
      \columnsep \imki@columnsep
      \ifx\imki@idxprologue\relax
        \begin{multicols}{\imki@columns}
      \else
        \begin{multicols}{\imki@columns}[\imki@idxprologue]
      \fi
    \else
      \imki@idxprologue
    \fi
  }{%
    \ifnum\imki@columns>\@ne\relax%
      \columnsep \imki@columnsep%
      \ifx\imki@idxprologue\relax%
        \begin{multicols}{\imki@columns}[\imki@indexlevel{\indexname}]%
      \else%
        \begin{multicols}{\imki@columns}[%
          \imki@indexlevel{\indexname}%
          \imki@idxprologue%
        ]%
      \fi%
    \else%
      \imki@indexlevel{\indexname}\imki@idxprologue%
    \fi%
  }{}{\tud@patch@wrn{theindex}}%
\fi%
%    \end{macrocode}
% \end{environment}^^A theindex
% \begin{macro}{\makexdyindex}
% \changes{v2.05}{2015/11/01}{neu}^^A
% Um die einzelnen Indexe erstellen zu können, wird das Makro \cs{makexdyindex} 
% definiert. Die Sternversion dieses Befehle erstellt dabei neben dem normalen
% Spezialindex einen weiteren, welcher explizit für Deklarationen genutzt wird.
% Dieser wird anschließen als erstes an \app{texindy} weitergereicht, um die
% richtige Formatierung der einzelnen Einträge gewährleisten zu können. Erst 
% danach wird die Datei mit den normalen Einträgen verarbeitet. Siehe hierzu 
% auch \cs{Process@@Index}.
%    \begin{macrocode}
\NewDocumentCommand\makexdyindex{s o m !o}{%
  \let\@tempb\@empty%
%<*class>
  \IfBooleanT{#1}{%
    \def\@tempc{\jobname-#4-declare.idx}%
    \makeindex[name={#4-declare}]%
    \IfFileExists{\@tempc}{%
      \edef\@tempb{-o \jobname-#4.ind \@tempc}%
    }{%
      \ClassWarningNoLine{tudscrmanual}{File \@tempc\space not found}%
    }%
  }%
%</class>
  \protected@edef\@tempa{%
    title={#3},%
    columnsep=\noexpand\f@size\noexpand\p@,%
    \IfValueTF{#4}{%
      name={#4},%
      options={-M \@currname-ind.xdy -L german-din -t \jobname-#4.xlg \@tempb},%
    }{%
      options={-M \@currname-ind.xdy -L german-din -t \jobname.xlg},%
    }%
    \IfValueT{#2}{#2}%
  }%
  \expandafter\makeindex\expandafter[\@tempa]%
}
%    \end{macrocode}
% \end{macro}^^A \makexdyindex
% Nun können mit \cs{makexdyindex} die einzelnen Indexe erstellt werden.
%    \begin{macrocode}
\makexdyindex{Allgemeiner Index}
\makexdyindex*{Klassen- und Paketoptionen}[options]
\makexdyindex*{Befehle und Umgebungen mit zugeh\"origen Parametern}[macros]
%<*class>
\makexdyindex*{Sprachabh\"angige Bezeichner}[terms]
\makexdyindex*{Seitenstile, Schriftelemente und Farben}[elements]
%</class>
\makexdyindex{L\"angen und Z\"ahler}[misc]
\makexdyindex*{Klassen, Pakete und Dateien}[files]
%<*class>
\makexdyindex[columns=1]{\"Anderungsliste}[changelog]
%</class>
\undef\makexdyindex
%    \end{macrocode}
% \begin{macro}{\if@tud@index}
% \changes{v2.05}{2015/11/05}{neu}^^A
% Dieser Schalter wird zwei unterschiedliche Aufgaben verwendet. Zum einen wird 
% er bei der Ausgabe von Verzeichnissen auf \val{true} gesetzt, um das Setzen 
% von Indexeinträgen durch Markup-Befehle in den Verzeichnissen selbst zu 
% unterbinden. Weiterhin verhindert der aktivierte Schalter bei der Ausführung 
% von \cs{Process@Markup} das Erstellen von erweiterten Querverweisen sowie das 
% Setzen von CTAN-Links bei Paketen (\cs{Package}), was weder in Verzeichnissen 
% noch im Index erwünscht ist, weshalb vom Makro \cs{@printindex} ebenfalls
% \cs{@tud@indextrue} gesetzt wird.
%    \begin{macrocode}
\newif\if@tud@index
\addtoeachtocfile{\protect\@tud@indextrue}
%    \end{macrocode}
% \end{macro}^^A \if@tud@index
%
% \subsubsection{Formatierung von Indexeinträgen}
%
% \begin{macro}{\tud@idx@get}
% \changes{v2.05}{2015/11/05}{neu}^^A
% \begin{macro}{\tud@idx@key}
% \changes{v2.05}{2015/11/05}{neu}^^A
% \begin{macro}{\tud@idx@fmt}
% \changes{v2.05}{2015/11/05}{neu}^^A
% \begin{macro}{\tud@idx@val}
% \changes{v2.05}{2015/11/05}{neu}^^A
% \begin{macro}{\tud@idx@bdl}
% \changes{v2.05}{2015/11/05}{neu}^^A
% Mit dem Makro \cs{tud@idx@get} wird die Formatierung eines Indexeintrags für 
% einen Markup-Befehl realisiert. Für das Sortieren der Indexeinträge kommt 
% \app{texindy} zum Einsatz. Um die Indexeinträge wie gewünscht zu Formatieren, 
% gibt es zwei Möglichkeit. 
%
% Bei der ersten Variante würde der Indexeintrag einfach direkt mit den 
% Markup-Befehlen erstellt und in der \app{texindy}-Stildatei wird für alle
% Auszeichnungsbefehle die passende |merge-rule| erstellt, um die korrekte 
% Sortierung zu gewährleisten. Tatsächlich war in einer früheren Version genau
% das der Ansatz. Allerdings werden die manuell erstellten RegExp-Regeln durch 
% \hrfn{http://sourceforge.net/p/xindy/bugs/22/}{\app{texindy} nicht korrekt}
% genutzt werden, weshalb der Ansatz verworfen wurde.
%
% Damit das Sortieren verlässlich funktioniert, werden die Einträge für den 
% Index deshalb in der gewohnten Syntax von \app{makeindex} in der Form
% \meta{Schlüsselwort}|@|\meta{Markup} erstellt. Diese Syntax kann auch mit
% \app{texindy} genutzt werden. 
%
% An \cs{tud@idx@get} werden vier obligatorische gefolgt von vier optionalen 
% Argumenten übergeben. Die ersten beiden Argumente sind die Makros, in welche
% \meta{Schlüsselwort}~-- für gewöhnlich \cs{tud@idx@key}~-- sowie die 
% Formatierung \meta{Markup}~-- normalerweise \cs{tud@idx@fmt}~-- expandiert 
% werden. Mit dem Makro im dritten Argument wird es möglich, Untereinträge zu 
% einem \meta{Schlüsselwort} für beispielsweise Parameter von Umgebungen oder
% Befehlen (\cs{Key}) sowie spezielle Schlüsselwerte von Optionen zu erzeugen.
% Dabei wird vor der Erstellung des Untereintrages der direkt zuvor generierte
% Indexhaupteintrag ins dritte Argument~-- normalerweise \cs{tud@idx@val}~--
% gesichert. Das vierte und letzte obligatorische Argument enthält schließlich 
% den eigentlichen Markup-Befehl, für welchen der Indexeintrag erzeugt werden 
% soll. 
% 
% Daran schließen sich vier optionale Argumente an, mit denen die Formatierung 
% des Indexeintrags differenziert werden kann. Das erste optionale Argument 
% |=|\meta{Wert}|=| kann wie bereits gewohnt für das Hinzufügen eines
% Schlüsselwertes genutzt werden. Das zweite, ordinäre optionale Argument 
% \oarg{Anmerkung} fügt dem Eintrag in den Index eine zusätzliche Anmerkung
% hinzu. Soll der Indexeintrag für ein spezifisches Paket oder eine Klasse aus
% dem \TUDScript-Bundle erstellt werden, wird das optionale Argument in runden 
% Klammern \parg{\string\Class\marg{Klasse}/\string\Package\marg{Paket}} für 
% einen Verweis auf das \parg{Bundleelement} verwendet, der in \cs{@idxbundle} 
% gespeichert wird. Das Argument \verb=|=\meta{Indexmarkup}\verb=|= wird bei
% Deklarationen zur Unterdrückung von Untereinträgen verwendet.
% Der vollständige Funktionsaufruf lautet:
% \newline\begingroup\scriptsize
%   \cs{tud@idx@get}\cs{tud@idx@key}\cs{tud@idx@fmt}\ignorespaces%
%   \cs{tud@idx@val}\marg{Elem.}|=|\meta{Wert}|=|\oarg{Anm.}\ignorespaces%
%   \parg{Bundleelem.}\verb=|=\meta{Indexmark.}\verb=|=\ignorespaces%
% \endgroup
%    \begin{macrocode}
\newcommand*\tud@idx@key{}
\newcommand*\tud@idx@fmt{}
\newcommand*\tud@idx@val{}
\newcommand*\tud@idx@bdl{}
\NewDocumentCommand\tud@idx@get{s m m m m r== r[] r() d||}{%
%    \end{macrocode}
% Die Änderungen an allerhand temporären Makros sollen lokal bleibens.
%    \begin{macrocode}
  \begingroup%
%    \end{macrocode}
% In einer weiteren Gruppe wird der Markup-Befehl des vierten obligatorischen
% Arguments ausgewertet. Es wird eine Fallunterschiedung durchgeführt, ob es 
% sich dabei um einen Parameter (\cs{Key}) handelt oder nicht. Ist dies der
% Fall, wird für diesen Parameter der hier nachfolgend formatierte Indexeintrag
% als Untereintrag verwendet. Dafür wird zuvor ein Haupteintrag aus dem ersten
% Argument von \cs{Key} generiert und in das dritte obligatorische Argument
% gespeichert (\cs{tud@idx@val}).
%    \begin{macrocode}
    \begingroup%
%<*class>
      \let#4\@empty%
      \in@{\Key}{#5}%
%    \end{macrocode}
% Um den Haupteintrag zu extrahieren, wird mit \cs{Markup@Suppress} die Ausgabe
% des Markups deaktiviert und das erste Argument von \cs{Key} ausgeführt. Dies
% führt über \cs{Process@Index} zum rekursiven Aufruf von \cs{tud@idx@get}. Als
% Ergebnis sind im zweiten und dritten Argument~-- sprich in \cs{tud@idx@key} 
% und \cs{tud@idx@fmt}~-- \meta{Markup} und \meta{Schlüsselwort} für den
% Elterneintrag enthalten. Diese werden in das vierte Argument (\cs{idxentry})
% gespeichert.
%    \begin{macrocode}
      \ifin@%
        \Markup@Suppress%
        \let\Key\@firstoftwo%
        #5%
%    \end{macrocode}
% Da der Inhalt des vierten Arguments am Ende der Gruppe expandiert wird, wird 
% mit \cs{expandonce} dafür gesorgt, dass die Expansion nur einmalig erfolgt. 
% Zu beachten ist außerdem, dass der Hauptindexeintrag bereits in der Form 
% \meta{Schlüsselwort}|@|\meta{Markup}|!| zusammengefügt wird. Die Sternversion
% wird für die Änderungsnotizen verwendet. Für diese wird eine kleine Anmerkung
% im Haupteintrag ergänzt.
%    \begin{macrocode}
        \let\@tempa\@empty%
        \IfBooleanT{#1}{\def\@tempa{: Parameter angepasst}}%
        \def#4{%
          \expandonce#2\expandonce\@tempa{}@%
          \expandonce#3\expandonce\@tempa{}!%
        }%
      \fi%
%</class>
%    \end{macrocode}
% Nachdem ein möglicher Haupteintrag für Parameter erzeugt wurde, folgt nun 
% der formatierte Indexeintrag für das aktuelle Markup-Element. Um diesen zu
% erstellen, wird für \meta{Schlüsselwort} zunächst nur der Inhalt des
% Hauptargumentes benötigt. Dieser wird hier in das zweite Argument
% expandiert, der ggf. erzeugte Haupteintrag in das vierte. Für \cs{Key} ist
% hier nur noch das zweite Argument relevant.
%    \begin{macrocode}
      \Markup@Gobble%
%<class>      \let\Key\@secondoftwo%
      \edef\tud@reserved{%
        \noexpand\endgroup%
        \def\noexpand#2{#5}%
%<class>        \def\noexpand#4{#4}%
      }%
    \tud@reserved%
%    \end{macrocode}
% Die Auszeichnung für den Indexeintrag besteht in erster Linie aus dem 
% Markup-Befehl selbst. Damit ein |@|"~Zeichen beispielsweise in einem 
% Makronamen korrekt verarbeitet wird, wird dieses im Stil von \app{makeindex} 
% mit |"| maskiert.
%    \begin{macrocode}
    \def#3{#5}%
    \tud@replace#2{@}{"@}%
    \tud@replace#3{@}{"@}%
%    \end{macrocode}
% In \cs{@idxbundle} wird ggf. das Paket oder die Klasse gespeichert, für die 
% der aktuelle Eintrag erzeugt werden soll. Eine explizite Angabe wird in jedem 
% Fall genutzt,\dots
%    \begin{macrocode}
    \let\tud@idx@bdl\tud@bdl@dflt%
    \IfValueTF{#8}{%
      \def\tud@idx@bdl{#8}%
    }{%
%    \end{macrocode}
% \dots eine implizite nur, wenn auch tatsächlich das passende Label existiert, 
% da sonst mit Sicherheit falsche Einträge im Index erscheinen würden.
%    \begin{macrocode}
%<*class>
      \tud@if@bdl{%
        \tud@if@lbl@exists{\tud@bdl@curr:#5}{%
          \edef\tud@idx@bdl{\expandonce\tud@bdl@curr}%
%    \end{macrocode}
% Sollte der Eintrag selbst dem Suffix für Paket oder Klasse entsprechen, wird
% dieser ignoriert.
%    \begin{macrocode}
          \def\tud@reserved{#5}%
          \ifx\tud@reserved\tud@idx@bdl%
            \let\tud@idx@bdl\tud@bdl@dflt%
          \fi%
        }{}%
      }{}%
%</class>
    }%
%    \end{macrocode}
% Nachfolgend wird für den zu erzeugenden Indexeintrag die Zugehörigkeit des
% aktuellen Elements zu einem Paket oder einer Klasse aus dem \TUDScript-Bundle 
% durch das Anhängen des jeweiligen \TUDScript-Bundle-Elements signalisiert.
% Dabei werden für die Anpassung von Schlüssel und Formatierung des Eintrags
% die temporären Makros \cs{@tempa}~(Schlüssel) und \cs{@tempb}~(Formatierung)
% gespeichert. In \cs{@tempc} wird der Inhalt von \cs{@tempb} zwischenzeitlich
% zur späteren Verwendung gesichert.
%    \begin{macrocode}
    \let\@tempa\@empty%
    \let\@tempb\@empty%
    \let\@tempc\@empty%
%    \end{macrocode}
% Der temporäre Schalter \cs{@tempswa} wird genutzt, um das Hinzufügen der 
% Ergänzungen zu steuern. Diese sollen nur für Elemente, welche nicht zu den 
% Hauptklassen gehören, erfolgen.
%    \begin{macrocode}
    \@tempswatrue%
    \ifx\tud@idx@bdl\tud@bdl@dflt%
      \@tempswafalse%
%    \end{macrocode}
% Ebenso werden die Anmerkungen für \cs{Key}-Elemente unterdrückt, da diese 
% sowieso als Untereintrag erzeugt werden. Einträge in der Änderungsliste 
% werden ohnehin separat für jedes \TUDScript-Bundle-Element ausgegeben. In 
% jedem Fall ist jedoch Formatierung beim Markup für das Erzeugen des richtigen 
% Querverweises notwendig und wird in \cs{@tempb} und \cs{@tempc} abgelegt.
%    \begin{macrocode}
    \else%
%<class>      \in@{\Key}{#5}\ifin@\@tempswafalse\fi%
      \IfBooleanT{#1}{\@tempswafalse}%
      \edef\@tempb{(\expandonce\tud@idx@bdl)}%
    \fi%
    \let\@tempc\@tempb%
%    \end{macrocode}
% Der Indexeintrag kann zur besseren Kenntnisnahme außerdem standardmäßig mit 
% einer Beschreibung versehen werden.
%    \begin{macrocode}
    \IfValueT{#7}{%
      \appto\@tempa{ #7}%
      \appto\@tempb{\suffix{#7}}%
    }%
%    \end{macrocode}
% Wird die Anmerkung gesetzt, ist für den Schlüssel nur der Inhalt des in 
% \cs{@idxbundle} gesicherten Markup-Befehls relevant. Dieser wird in der 
% folgenden Gruppe extrahiert und \cs{@tempa} angehangen. Für die Formatierung 
% bleibt das Markup erhalten und wird \cs{tempb} hinzugefügt.
%    \begin{macrocode}
    \if@tempswa%
      \begingroup%
        \Markup@Gobble%
        \edef\tud@reserved{%
          \noexpand\endgroup%
          \noexpand\appto\noexpand\@tempa{ \tud@idx@bdl}%
        }%
      \tud@reserved%
      \eappto\@tempb{\suffix{\expandonce\tud@idx@bdl}}%
    \fi%
%    \end{macrocode}
% Sollte mit |=|\meta{Wert}|=| ein spezieller Schlüsselwert angegeben worden
% sein, wird für diesen normalerweise ein separater Untereintrag erstellt. Der
% Schalter \cs{@tempswa} wird genutzt, um das Erzeugen dieses Untereintrages im
% Zweifel zu unterdrücken und diesen stattdessen als normalen Eintrag zu
% setzen. Dies ist zum einen bei Deklarationen und zum anderen in der
% Änderungsliste notwendig. Außerdem kann dies auch mit der Angabe von 
% \val{default} als Indexargument erzwungen werden.
%    \begin{macrocode}
    \@tempswafalse%
    \IfBooleanT{#1}{\@tempswatrue}%
    \IfValueT{#9}{\ifstr{#9}{declare}{\@tempswatrue}{}}%
    \IfValueT{#9}{\ifstr{#9}{default}{\@tempswatrue}{}}%
%    \end{macrocode}
% In diesem Fall werden die Ergänzungen für Schlüssel und Formatierung ggf. um 
% die speziellen Schlüsselwerte für das Element ergänzt.
%    \begin{macrocode}
    \if@tempswa%
      \protected@eappto#2{\expandonce\@tempa}%
      \tud@doifPValue{#6}{\noexpand\appto\noexpand#2{=#6}}%
      \protected@eappto#3{\IfValueT{#6}{=#6=}\expandonce\@tempb}%
%    \end{macrocode}
% Andernfalls wird geprüft, ob ein verwertbarer Schlüsselwert angegeben wurde.
%    \begin{macrocode}
    \else%
      \tud@doifPValue{#6}{\noexpand\@tempswatrue}%
%    \end{macrocode}
% Ist dies der Fall, wird aus den bisherigen Eingaben der Haupteintrag und 
% daran anschließend der dazugehörige Untereintrag erstellt\dots
%    \begin{macrocode}
      \if@tempswa%
        \eappto#4{%
          \expandonce#2\expandonce\@tempa{}@%
          \expandonce#3\expandonce\@tempb{}!%
        }%
        \appto#2{=#6}%
        \appto#3{=#6=}%
        \protected@eappto#3{\expandonce\@tempc}%
%    \end{macrocode}
% \dots andernfalls bleibt es bei einem normalem Eintrag in den Index.
%    \begin{macrocode}
      \else%
        \protected@eappto#2{\expandonce\@tempa}%
        \protected@eappto#3{\expandonce\@tempb}%
      \fi%
    \fi%
%    \end{macrocode}
% Die Erzeugung eines formatierten Indexeintrages ist beendet, das Resultat 
% wird nach der letzten Gruppe definiert.
%    \begin{macrocode}
    \let\emph\@firstofone%
    \protected@edef\tud@reserved{%
      \noexpand\endgroup%
      \def\noexpand#2{#2}%
      \def\noexpand#3{#3}%
      \def\noexpand#4{#4}%
    }%
  \tud@reserved%
}
%    \end{macrocode}
% \end{macro}^^A \tud@idx@bdl
% \end{macro}^^A \tud@idx@val
% \end{macro}^^A \tud@idx@fmt
% \end{macro}^^A \tud@idx@key
% \end{macro}^^A \tud@idx@get
% \begin{macro}{\cleversee}
% \changes{v2.05}{2015/11/02}{neu}^^A
% \begin{macro}{\cleverseealso}
% \changes{v2.05}{2015/11/02}{neu}^^A
% \begin{macro}{\tud@cleversee}
% \changes{v2.05}{2015/11/02}{neu}^^A
% \changes{v2.05k}{2017/03/20}{neu}^^A
% Diese Befehle dienen zum smarten Setzen von Hinweisen im Index. Existiert 
% lediglich dieser Hinweis im Index, wird dieser mit dem Präfix \cs{seename} 
% ausgegeben. Sind jedoch für einen Indexeintrag auch Seitenzahlen vorhanden, 
% so werden diese zuerst ausgegeben, danach folgt der Querverweis mit dem 
% Präfix \cs{seealsoname}. Der Befehl \cs{cleverseealso} verwendet letzteren 
% Präfix in jedem Fall.
%    \begin{macrocode}
\newrobustcmd*\cleversee{\tud@cleversee{\seename}}
\newrobustcmd*\cleverseealso{%
%<*class>
  \ClassWarning{tudscrmanual}%
%</class>
%<*package>
  \PackageWarning{tudscrtutorial}%
%</package>
    {You should use `|see' instead of `|seealso'}%
  \tud@cleversee{\alsoname}%
}
%    \end{macrocode}
% Diese Makro übernimmt die eigentliche Arbeit. Es ist darauf angewiesen, dass 
% die angegebene Seitenzahlen eines Indexeintrages erst \emph{nach} den
% Hinweisen ausgegeben und mit \cs{relax} abgeschlossen werden. Dies wird mit
% der später definierten Stildatei für \app{texindy} sichergestellt. So wird es 
% möglich, dass das Makro \cs{tud@cleversee} auf alles ihm bis \cs{relax} 
% nachfolgende parsen kann und abhängig davon die Ausgabe gestaltet. Das zweite 
% obligatorische Argument beinhaltet dabei den Querverweis, das dritte Argument
% ist alles nachfolgende bis \cs{relax}.
%    \begin{macrocode}
\newcommand*\tud@cleversee{}
\def\tud@cleversee#1#2#3\relax{%
  \IfArgIsEmpty{#3}{%
    \emph{#1}\space#2%
  }{%
%    \end{macrocode}
% Folgen im Index nach \cs{cleversee}\marg{Querverweis} noch weitere Einträge, 
% so sind diese zu Beginn durch \texttt{,\textvisiblespace} getrennt. Diese 
% zwei Zeichen werden mithilfe von \cs{@tempa} ignoriert.
%    \begin{macrocode}
    \begingroup%
      \def\@tempa, ##1\@nil{##1,\space\emph{\alsoname}\space#2}%
      \@tempa#3\@nil%
    \endgroup%
  }%
}
%    \end{macrocode}
% \end{macro}^^A \tud@cleversee
% \end{macro}^^A \cleverseealso
% \end{macro}^^A \cleversee
% \begin{macro}{\seeref}
% \changes{v2.05}{2015/11/02}{neu}^^A
% \begin{macro}{\seeidx}
% Der Befehl \cs{seeref} kann anstelle von \cs{see} bzw. in der Sternversion
% statt \cs{alsosee} genutzt werden. Das Makro \cs{seeidx} verweist auf die 
% Seite eines Indexes.
%    \begin{macrocode}
%<*class>
\NewDocumentCommand\seeref{s m}{%
  \emph{\IfBooleanTF{#1}{\alsoname}{\seename}}\space#2%
}
\newcommand*\seeidx[2]{\pageref{idx:#1}}
%</class>
%    \end{macrocode}
% \end{macro}^^A \seeidx
% \end{macro}^^A \seeref
%
% \subsubsection{Erstellen von Indexeinträgen}
%
% \begin{macro}{\Process@Index}
% \changes{v2.05}{2015/11/15}{neu}^^A
% \begin{macro}{\Process@@Index}
% \changes{v2.05}{2015/11/15}{neu}^^A
% Mit \cs{Process@Index} wird geprüft, ob mit dem im zweiten obligatorischen 
% Argument enthaltenen Markup-Befehl ein Indexeintrag erzeugt werden soll. 
% Ein Indexeintrag kann mit der Sternversion eines Markup-Befehls explizit 
% unterdrückt werden, wobei der boolesche Wert im ersten obligatorischen 
% Argument an \cs{Process@Index} zu Prüfung übergeben wird. Außerdem wird für
% obsolete Deklarationen ebenso wie bei aktiviertem Schalter \cs{if@tud@index} 
% kein Eintrag erstellt. In allen genannten Fällen passiert beim Aufruf von 
% \cs{Process@Index} nichts, andernfalls wird \cs{Process@@Index} ausgeführt.
%
% Die auf die zwei obligatorischen Argumente folgenden optionalen sind in der
% Nomenklatur identisch zu den bisherigen Erläuterungen, der vollständige 
% Funktionsaufruf lautet:
% \newline\begingroup\scriptsize
%   \cs{Process@Index}\marg{Stern?}\marg{Element}\ignorespaces%
%   |=|\meta{Wert}|=|\oarg{Anmerkung}\parg{Bundleelement}\ignorespaces%
%   \verb=|=\meta{Indexmarkup}\verb=|=\ignorespaces%
% \endgroup
%    \begin{macrocode}
\NewDocumentCommand\Process@Index{m m d== o d() d||}{%
  \ifboolexpr{%
%<class>    bool {@tud@index} or bool {@tud@obsolete}%
%<package>    bool {@tud@index}%
  }{}{%
    \IfBooleanTF{#1}{%
      \IfValueT{#6}{%
%<*class> 
        \ClassWarning{tudscrmanual}%
%</class>
%<*package>
        \PackageWarning{tudscrtutorial}%
%</package>
        {%
          You've used the starred version. No index entry\MessageBreak%
          for #2 is created. The\MessageBreak%
          optional argument |#6| will be ignored%
        }%
      }%
    }{%
      \Process@@Index{#2}=#3=[#4](#5)|#6|%
    }%
  }%
}
%    \end{macrocode}
% Mit \cs{Process@@Index} erfolgt die eigentliche Erzeugung des Indexeintrages.
%    \begin{macrocode}
\NewDocumentCommand\Process@@Index{m r== r[] r() r||}{%
%    \end{macrocode}
% Dabei erledigt \cs{tud@idx@get} einen Großteil der Aufgabe. Mit besagtem 
% Makro werden Schlüssel und Format für den gewünschten Indexeintrag erzeugt.
% Sollte optional ein spezieller |=|\meta{Wert}|=| für einen Schlüssel gegeben
% sein, so wird ggf. auch der passende Haupteintrag erstellt.
%    \begin{macrocode}
%<*class>
  \if@tud@declare%
    \tud@idx@get\tud@idx@key\tud@idx@fmt\tud@idx@val{#1}=#2=[#3](#4)|declare|%
  \else%
%</class>
    \tud@idx@get\tud@idx@key\tud@idx@fmt\tud@idx@val{#1}=#2=[#3](#4)|#5|%
%<*class>
  \fi%
%</class>
%    \end{macrocode}
% Mit der Sternversion \cs{tud@attr@get*} wird der Zielindex für das im 
% ersten Argument übergebene \marg{Element} ermittelt.
%    \begin{macrocode}
  \tud@attr@get*\tud@lbl@tmp{#1}%
%    \end{macrocode}
% Markup-Befehle, welche in Deklarationen verwendet wurden, werden für eine 
% höhere Priorisierung in einem speziellen Index angelegt, wobei hier die 
% Formatierung dieses Eintrags fest vorgegeben ist.
%    \begin{macrocode}
%<*class>
  \if@tud@declare%
    \if@openindex%
      \IfValueT{#5}{%
        \ClassWarning{tudscrmanual}{%
          It is not possible to use any indexing format\MessageBreak%
          like |#5| within the arguments of the\MessageBreak%
          environments `Declaration' or `Declaration*'%
        }%
      }%
%    \end{macrocode}
% Jetzt folgt~-- abhängig vom Schalter \cs{if@openindex}~-- der öffnende oder 
% schließende Indexeintrag der Deklaration. Für den schließenden ist dabei eine
% Formatierung des Schlüssels nicht notwendig.
%    \begin{macrocode}
      \index[\tud@lbl@tmp-declare]{%
        \tud@idx@val\tud@idx@key{}@\tud@idx@fmt|(declare%
      }%
    \else%
      \index[\tud@lbl@tmp-declare]{\tud@idx@val\tud@idx@key|declare)}%
    \fi%
%    \end{macrocode}
% Für Markup-Befehle außerhalb von Deklarationsumgebungen wird ein einfacher 
% Eintrag in den Index erzeugt, wobei auch im Fließtext über das Indexmarkup 
% \verb=|=\val{declare}\verb=|= eine Quasi-Deklaration verwendet werden kann. 
% Dies wird beispielsweise für \KOMAScript-Optionen genutzt, um diese in der 
% gewünschten Formatierung im Index erscheinen zu lassen.
%    \begin{macrocode}    
  \else%
%</class>
    \IfValueT{#5}{%
%<*class>
      \ifstr{#5}{declare}{%
        \appto\tud@lbl@tmp{-declare}%
        \appto\tud@idx@fmt{|declare}%
      }{%
        \appto\tud@idx@fmt{|#5}%
      }%
%</class>
%<*package>
      \appto\tud@idx@fmt{|#5}%
%</package>
    }%
    \index[\tud@lbl@tmp]{\tud@idx@val\tud@idx@key{}@\tud@idx@fmt}%
%<*class>
  \fi%
%</class>
}
%    \end{macrocode}
% \end{macro}^^A \Process@@Index
% \end{macro}^^A \Process@Index
%
% \iffalse
%<*class>
% \fi
%
% \begin{macro}{\CrossIndex}
% Der Befehl \cs{CrossIndex} dient zum Eintragen eines Schlagwortes, welches im 
% ersten obligatorischen Argument angegeben wird, in den allgemeinen Index und
% verweist auf den im zweiten obligatorischen Argument, zum Schlagwort gehörigen
% Spezialindex. Das ordinäre optionale Argument kann verwendet werden, um ggf.
% den Eintrag in den allgemeinen Index, der normalerweise mit dem jeweiligen
% Schlagwort erfolgt, zu überschrieben.
%
% Wird die Sternversion von \cs{CrossIndex} genutzt, so wird der Querverweis 
% als Eintrag in der obersten Ebene gesetzt, andernfalls wird ein Untereintrag
% gesetzt. Mit der normalen Variante können so zu einem Schlagwort weitere
% Verweise im allgemeinen Index erscheinen. Das zweite obligatorische Argument
% kann eine kommagetrennte Liste enthalten.
%    \begin{macrocode}
\NewDocumentCommand\CrossIndex{s m o m}{%
  \def\@tempa##1##2##3{%
    \index{%
      \IfBooleanTF{#1}{##3@##3}{##3!"|@}%
      \,\textrightarrow\,%
      \tudhyperref{idx:##2}{\IfValueTF{##1}{##1}{Index der ##3}}|seeidx{##2}%
    }%
  }%
  \forcsvlist{\@tempa{#3}{#4}}{#2}%
}
%    \end{macrocode}
% \end{macro}^^A \CrossIndex
% \begin{macro}{\SeeRef}
% \changes{v2.05}{2015/11/05}{neu}^^A
% Mit \cs{SeeRef} wird unter dem im ersten Argument gegebenen Schlagwort ein
% Indexuntereintrag erzeugt, der wiederum auf einen Indexeintrag des zweiten 
% Argumentes verweist. Dabei kann im zweiten Argument ein normaler Begriff oder 
% ein Markup-Befehl verwendet werden. Normalerweise wird für den Querverweis 
% die Referenzklasse \verb=|see= verwendet, die Sternversion des Befehls
% hingegen nutzt \verb=|seeunverified=, um auf eine Prüfung des verwiesenen
% Eintrags zu verzichten.
%    \begin{macrocode}
\NewDocumentCommand\SeeRef{s m m}{%
  \sbox\z@{%
    \let\tud@idx@fmt\relax%
    #3%
%    \end{macrocode}
% Wurde im zweiten Argument ein Markup-Befehl angegeben, so wurde durch das 
% Erstellen der Box das dazugehörige Indexformat in \cs{tud@idx@fmt} für den
% Eintrag erstellt. Andernfalls wurde die Definiton von \cs{tud@idx@fmt} nicht 
% geändert und das Argument wird nach der Box direkt in den Index eingetragen.
%    \begin{macrocode}
    \ifx\tud@idx@fmt\relax%
      \gdef\tud@reserved{#3}%
    \else%
      \global\let\tud@reserved\tud@idx@fmt%
    \fi%
  }%
  \IfBooleanTF{#1}{\def\tud@idx@fmt{seeunverified}}{\def\tud@idx@fmt{see}}%
  \index{#2!#3|\tud@idx@fmt{\tud@reserved}}%
}
%    \end{macrocode}
% \end{macro}^^A \SeeRef
%
% \iffalse
%</class>
% \fi
%
% \subsubsection{Indexausgabe}
%
% Dies sind alle Befehle, die zur Ausgabe der erzeugten Indexe benötigt werden.
% \begin{macro}{\PrintIndex}
% \changes{v2.02}{2014/08/20}{neu}^^A
% \begin{macro}{\tud@indexprologue}
% \changes{v2.02}{2014/08/20}{neu}^^A
% Mit \cs{PrintIndex} erfolgt die Ausgabe aller erstellten Indexe. Dabei wird 
% für bestimmte Spezialindexe das Makro \cs{tud@indexprologue} neu gesetzt, um 
% für diese vor der eigentlichen Ausgabe einige Anmerkungen machen zu können.
% Mit \cs{tud@indexprologue} kann der einleitende Teil für einen Index gesetzt
% werden.
%    \begin{macrocode}
%<*class>
\newcommand*\tud@indexprologue{}
%</class>
\newcommand*\PrintIndex{%
  \begingroup%
    \providecommand*\lettergroup[1]{%
      \par\textbf{\textsf{##1}}\par%
      \nopagebreak%
    }%
%<*class>
    \renewcommand\tud@indexprologue{%
      Die im Folgenden aufgelisteten Schlagworte sollen f\"ur den Umgang mit %
      \hologo{LaTeXe} im Allgemeinen sowie dem \TUDScript-Bundle im Speziellen %
      sowohl Antworten bei generellen Fragen liefern als auch L\"osungen f\"ur %
      typische Probleme bereitstellen. Falls ein gesuchter Begriff hier nicht %
      zu finden ist oder trotz vorhandener Hinweise kein zufriedenstellendes %
      Ergebnis erzielt werden kann, sollte das \Forum* erster Anlaufpunkt %
      sein, um weitere Hilfe bei der Nutzung von \TUDScript zu erhalten.%
    }%
%</class>
%<*package>
    \begingroup%
    \let\lettergroup\@gobble%
    \let\indexspace\relax%
%</package>
    \print@index%
%<*package>
    \endgroup%
%</package>
%<*class>
    \clearpage%
    \renewcommand\tud@indexprologue{%
      Dies ist der Index aller im Handbuch erl\"auterten sowie erw\"ahnten %
      Optionen, wobei bei den meisten auch Untereintr\"age mit Seitenangaben %
      f\"ur spezielle Wertzuweisungen existieren.%
    }%
%</class>
    \print@index[options]%
%<*class>
    \renewcommand\tud@indexprologue{%
      Dies ist eine Auflistung aller zuvor erl\"auterten sowie erw\"ahnten %
      Befehle und Umgebungen. Bei einigen sind zus\"atzliche Untereintr\"age %
      f\"ur Schl\"ussel-Wert-Parameter zu finden, die im optionalen Argument %
      der jeweiligen Anweisung verwendet werden k\"onnen. Gegebenenfalls sind %
      f\"ur explizite Wertzuweisungen an diese Parameter Seitenverweise zu % 
      finden.%
    }%
%</class>
    \print@index[macros]%
%<*class>
    \renewcommand\tud@indexprologue{%
      Dies sind die von \TUDScript definierten, sprachabh\"angigen Bezeichner. %
      Informationen zur Verwendung sowie den Anpassungsm\"oglichkeiten sind in %
      \autoref{sec:localization} zu finden.%
    }%
    \print@index[terms]%
    \print@index[elements]%
%</class>
%<*package>
    \begingroup%
    \let\lettergroup\@gobble%
    \let\indexspace\relax%
%</package>
    \print@index[misc]%
%<*package>
    \endgroup%
%</package>
    \print@index[files]%
  \endgroup%
}
%    \end{macrocode}
% \end{macro}^^A \tud@indexprologue
% \end{macro}^^A \PrintIndex
% \begin{macro}{\print@index}
% \changes{v2.02}{2014/07/25}{neu}^^A
% \begin{macro}{\tud@idx@skip}
% \changes{v2.02}{2014/09/02}{neu}^^A
% Mit \cs{@printindex} erfolgt die Ausgabe der einzelnen Indexe. Dabei wird in 
% der Klasse \cls{tudscrmanual} zuvor für jeden Index ein referenzierbares
% Label erstellt und ggf. der Prolog gesetzt.
%
% Aufgrund der Verwendung der \env{multicols}-Umgebung durch das Paket
% \pkg{imakeidx} werden normalerweise ohne einen Prolog zu große vertikale
% Abstände gesetzt. Dies wird mit \cs{tud@idx@skip} behoben.
%    \begin{macrocode}
\newcommand*\tud@idx@skip{%
  \ifnum\imki@columns>\@ne\relax\vspace{-\multicolsep}\fi%
}
\newcommand*\print@index[1][]{%
%    \end{macrocode}
% Falls ein Seitenumbruch notwendig ist, sollte dieser vor dem Index und nicht 
% unmittelbar nach dem Beginn erfolgen.
%    \begin{macrocode}
  \pagebreak[3]%
  \@tud@indextrue%
%<*class>
%    \end{macrocode}
% Das zu erstellende Label wird in \cs{tud@reserved} gespeichert. Dabei wird 
% die Sternversion von \cs{tudhyperdef} genutzt, um das Label auf die Höhe der 
% Überschrift zu setzen.
%    \begin{macrocode}
  \ifblank{#1}{%
    \def\tud@reserved{\tudhyperdef*{idx:main}}%
  }{%
    \def\tud@reserved{\tudhyperdef*{idx:#1}}%
  }%
%    \end{macrocode}
% Ohne Prolog wird der fehlerhafte vertikale Abstand behoben. 
%    \begin{macrocode}
  \ifx\tud@indexprologue\@empty%
    \indexprologue[\tud@idx@skip]{\tud@reserved}%
%    \end{macrocode}
% Mit Prolog wird der fehlerhafte vertikale Abstand nach diesem korrigiert.
%    \begin{macrocode}
  \else%
    \indexprologue{%
      \tud@reserved%
      \tud@indexprologue%
      \tud@idx@skip%
    }%
  \fi%
%</class>
%    \end{macrocode}
% Fur das Paket \pkg{tudscrtutorial} erfolgt eine einfache Ausgabe der Indexe.
%    \begin{macrocode}
%<package>  \indexprologue[\tud@idx@skip]{}%
%    \end{macrocode}
% Die Ausgabe des gewünschten Index und des leeren des Prologs.
%    \begin{macrocode}
  \ifblank{#1}{\printindex}{\printindex[#1]}%
  \ifnum\imki@columns>\@ne\relax\vspace{-\multicolsep}\fi%
%<class>  \let\tud@indexprologue\@empty%
}
%    \end{macrocode}
% \end{macro}^^A \tud@idx@skip
% \end{macro}^^A \print@index
%
% \iffalse
%<*class>
% \fi
%
% \subsection{Änderungsliste}
%
% Um dem Anwender einen schnellen Überblick zu Änderungen der aktuellen Version 
% bereitstellen zu können, werden Befehle zur automatisierten Erzeugung einer 
% solchen Liste definiert. Die Änderungsliste ist der Klasse \cls{tudscrmanual} 
% vorbehalten.
%
% Änderungsnotizen lassen sich auf zwei unterschiedlichen Wegen generieren. 
% Entweder über die Nutzung von \cs{ChangedAt} oder über das optionale Argument
% einer der Deklarationsumgebungen. Diese sind in der Form 
% \meta{Versionsnummer}|:|\meta{Änderungsnotiz} anzugeben. Die alleinige Angabe
% der Versionsnummer ohne Änderungsnotiz ist ebenso möglich.
% 
% Wird die Änderungsnotiz für ein bestimmtes Element wie eine Option oder ein 
% Befehl erstellt, wie es bei den Deklarationsumgebungen implizit oder bei der 
% Verwendung von \cs{ChangedAt} mit optionalem Argument geschieht, gibt es die 
% zusätzliche Möglichkeit, einen Untereintrag zu erstellen. Dabei ist dieser 
% nach Versionsnummer und vor Änderungsnotiz in folgender Form anzugeben: 
% \meta{Versionsnummer}|!|\meta{Untereintrag}|:|\meta{Änderungsnotiz}. Der
% Untereintrag ist für die explizite Angabe eines Schlüssel-Wert-Paares gedacht.
% \begin{macro}{\if@tud@changedat}
% \begin{macro}{\Process@ChangedAt}
% \changes{v2.02}{2014/07/25}{erweitert}^^A
% \changes{v2.02}{2014/10/08}{Sternversion ergänzt}^^A
% Der Schalter \cs{if@tud@changedat} wird im weiteren Verlauf dazu verwendet, 
% das Standardverhalten der einzelnen Markup-Befehle kurzeitig zu ändern und
% anstelle der Makros \cs{Process@Markup} und \cs{Process@Index} den Befehl 
% \cs{Process@ChangedAt} auszuführen, mit welchem die Formatierung der Einträge 
% der Änderungsliste erfolgt.
%   \begin{macrocode}
\newif\if@tud@changedat
%    \end{macrocode}
% Die Formatierung der Änderungseinträge erfolgt~-- wie auch für den Index~-- 
% mit dem Makro \cs{tud@idx@get}, wobei die Sternversion für die Identifikation
% als Änderungseintrag fungiert. Aufgerufen wird \cs{Process@ChangedAt} während 
% der Erstellung der Änderungseinträge mit \cs{Changed@At@CreateEntry}. Der
% Aufruf des Makros erfolgt dabei in gewohnter Form mit einem obligatorischen 
% gefolgt von drei optionalen Argumenten:
% \newline\begingroup\scriptsize
%   \cs{Process@ChangedAt}\marg{Element}|=|\meta{Wert}|=|\ignorespaces%
%   \oarg{Ausgabesuffix}\parg{Bundleelement}\ignorespaces%
% \endgroup
%   \begin{macrocode}
\NewDocumentCommand\Process@ChangedAt{m d== o d()}{%
  \tud@idx@get*\tud@idx@key\tud@idx@fmt\tud@idx@val{#1}=#2=[#3](#4)%
}
%    \end{macrocode}
% \end{macro}^^A \Process@ChangedAt
% \end{macro}^^A \if@tud@changedat
% \begin{macro}{\ChangedAt}
% \changes{v2.02}{2014/07/25}{erweitert}^^A
% \changes{v2.02}{2014/10/15}{Sternversion neu (keine Randnotiz)}^^A
% \changes{v2.06}{2018/09/03}{optionale Angabe des Bundles}^^A
% \begin{macro}{\@ChangedAt}
% \changes{v2.06}{2019/06/24}{neu}^^A
% \begin{length}{\changedatskip}
% \changes{v2.04}{2015/06/08}{neu}^^A
% Der Befehl \cs{ChangedAt} kann im Fließtext für das Erstellen einer freien 
% Änderungsnotiz genutzt werden. Das obligatorische Argument sollt wie kurz 
% zuvor beschrieben verwendet werden.
%
% Eine Änderungsnotiz wird für die angegebene Versionsnummer normalerweise im
% Abschnitt \enquote{Allgemein} aufgeführt. Soll jedoch eine Notiz explizit für
% eine Option, eine Umgebung, einen Befehlt etc. erstellt werden, kann dieses 
% Element vor dem obligatorischen im optionalen Argument angegeben werden. Der 
% Eintrag erfolgt dann im Abschnitt \enquote{Implementierung} für das gegebene
% Element. Im obligatorischen Argument können auch mehrere Änderungsnotizen 
% gleichzeitig erzeugt werden. hierfür sind diese mit Semikolon voneinander zu 
% trennen. Das Erzeugen der Liste erfolgt mit \cs{Changed@At@CreateList}.
% 
% Normalerweise wird bei der Nutzung von \cs{ChangedAt} zusätzlich zum Eintrag
% in die Änderungsliste eine Randnotiz mit den angegebenen Versionsnummern
% erzeugt. Mit dem letzten optionalen Argument nach dem obligatorischen kann
% ein vertikaler Versatz dieser angegeben werden. Die Sternversion des Befehls 
% unterdrückt die Ausgabe der Randnotiz. 
%    \begin{macrocode}
\newlength\changedatskip
\NewDocumentCommand\ChangedAt{s o d() m !O{\changedatskip}}{%
  \IfValueTF{#2}{%
    \Changed@At@CreateList[#2](#3){#4}%
  }{%
    \Changed@At@CreateList(#3){#4}%
  }%
%    \end{macrocode}
% Nach dem Erstellen der Liste wird diese abgearbeitet und ggf. die Randnotiz 
% ausgegeben.
%    \begin{macrocode}
  \print@changedatlist{#1}{#5}%
}
%    \end{macrocode}
% Die expandierbare Version des Befehls ist innerhalb von Überschriften nötig.
%    \begin{macrocode}
\NewExpandableDocumentCommand\@ChangedAt{s o d() m}{}
\AfterPackage*{hyperref}{%
  \pdfstringdefDisableCommands{%
    \let\ChangedAt\@ChangedAt%
  }%
}
\BeforeStartingTOC{\let\ChangedAt\@ChangedAt}
%    \end{macrocode}
% \end{length}^^A \changedatskip
% \end{macro}^^A \@ChangedAt
% \end{macro}^^A \ChangedAt
%
% \subsubsection{Sukzessive Erstellung der Einträge für die Änderungsliste}
%
% Das Erstellen von Änderungsnotizen erfolgt in zwei Schritten. Diese werden 
% zuerst mit dem Befehl \cs{Changed@At@CreateList} in einer standardisierten 
% Form in der temporären Liste \cs{tud@changedat@list} gespeichert und später 
% mit dem Makro \cs{Changed@At@CreateEntry} verarbeitet. Dieses Vorgehen liegt
% in den Umgebungen für Deklarationen begründet, welche verschachtelt werden
% können, jedoch einmalig mit \cs{printdeclarationlist} abgearbeitet werden.
% 
% \begin{macro}{\tud@changedat@list}
% \changes{v2.02}{2014/07/25}{neu}^^A
% \begin{macro}{\forssvlist}
% \changes{v2.05}{2015/11/17}{neu}^^A
% Hier werden die Liste \cs{tud@changedat@list} sowie das Makro \cs{forssvlist} 
% für die Aufspaltung der semikolongetrennten Änderungsnotizen definiert.
%    \begin{macrocode}
\newcommand*\tud@changedat@list{}
\let\tud@changedat@list\relax
\DeclareListParser*{\forssvlist}{;}
%    \end{macrocode}
% \end{macro}^^A \forssvlist
% \end{macro}^^A \tud@changedat@list
% \begin{macro}{\Changed@At@CreateList}
% \changes{v2.02}{2014/09/02}{neu}^^A
% \begin{macro}{\Changed@At@@CreateList}
% \changes{v2.02}{2014/10/09}{neu}^^A
% \begin{macro}{\Changed@At@@@CreateList}
% \begin{macro}{\Changed@At@@@@CreateList}
% Mit diesen Befehlen wird eine Liste im Format von \pkg{etoolbox} erzeugt, 
% welche alle notwendigen Angaben für das Erzeugen der Änderungshistorie in 
% Form eines Indexes mit \cs{Changed@At@CreateEntry} enthält.
%
% Der Befehl \cs{Changed@At@CreateList} kann vor dem obligatorischen Argument 
% mit einem ordinären optionalen Argument verwendet werden, in welchem ein 
% Makro, eine Umgebung, eine Option etc. angegeben wird. Ist dies der Fall, so 
% wird der Eintrag in die Änderungsliste für dieses Element generiert, sonst 
% wird eine allgemeiner Eintrag erzeugt. Das optionale Argument in runden 
% Klammern wird lediglich verwendet, wenn für eine der beiden Spezialumgebungen 
% \env{Declaration*} bzw. \env{Bundle*} eine Änderungsnotiz erstellt wird, um 
% diese für das jeweilige \TUDScript-Bundle-Element in den entsprechenden
% Abschnitt der Änderungsliste einzutragen. Das Hauptargument verarbeitet
% schließlich eine semikolongetrennte Liste und reicht jeden Bestandteil 
% zusammen mit den beiden optionalen Argumenten an \cs{Changed@At@@CreateList} 
% weiter. 
%    \begin{macrocode}
\NewDocumentCommand\Changed@At@CreateList{o d() m}{%
  \IfValueT{#3}{\forssvlist{\Changed@At@@CreateList[#1](#2)}{#3}}%
}
%    \end{macrocode}
% Mit \cs{Changed@At@@CreateList} wird lediglich die Versionsnummer von der 
% eigentlichen Änderungsnotiz getrennt.
%    \begin{macrocode}
\NewDocumentCommand\Changed@At@@CreateList{r[] r() >{\SplitArgument{1}{:}}m}{%
  \Changed@At@@@CreateList[#1](#2)#3%
}
%    \end{macrocode}
% Zuletzt wird durch \cs{Changed@At@@@CreateList} ein möglicher Untereintrag
% in ein separates Argument abgetrennt und \cs{Changed@At@@@@CreateList} mit
% allen Argumenten aufgerufen.
%    \begin{macrocode}
\NewDocumentCommand\Changed@At@@@CreateList{%
  r[] r() >{\SplitArgument{1}{!}}m m%
}{%
  \Changed@At@@@@CreateList[#1](#2)#3{#4}%
}
%    \end{macrocode}
% Der Befehl \cs{Changed@At@@@@CreateList} übernimmt nun die Erstellung des 
% Eintrags in die Liste \cs{tud@changedat@list}. Wie bereits erwähnt, enthalten 
% die beiden ersten Argumente ggf. Markup-Befehl und \TUDScript-Bundle-Element. 
% Danach folgt die Versionsnummer der Änderungsnotiz, der mögliche Untereintrag 
% für einen Markup-Befehl sowie die Änderungsnotiz selbst. 
%
% Die oberste Gliederungsebene der Änderungshistorie ist die Versionsnummer. 
% Unter dieser werden für jedes \TUDScript-Bundle-Element die Änderungsnotizen 
% separat ausgegeben, wobei diese nochmal in einen allgemeinen Teil und einen 
% zur Implementierung getrennt werden. 
%
% Der im Folgenden erzeugte Eintrag für die Liste \cs{tud@changedat@list} wird
% in zwei unterschiedlichen Varianten generiert. In der Form 
% \newline\begingroup\scriptsize
%   \meta{Versionsnummer}|!Allgemein!|\meta{Änderungsnotiz}\ignorespaces%
% \endgroup\newline
%  werden allgemeine Änderungsnotizen erstellt, für Einträge bezüglich der
% Implementierung wird die folgende Form genutzt:
% \newline\begingroup\scriptsize
%   \meta{Versionsnummer}|!Implementierung!|\meta{Markup-Befehl}\ignorespaces%
%   |!|\meta{Untereintrag}|!|\meta{Änderungsnotiz}\ignorespaces%
% \endgroup
%    \begin{macrocode}
\NewDocumentCommand\Changed@At@@@@CreateList{r[] r() m m m}{%
%    \end{macrocode}
% Die Aufteilung der Änderungsliste in die einzelnen \TUDScript-Bundle-Elemente 
% erfolgt entweder anhand des explizit angegebenen optionalen Argumentes in
% runden Klammern oder aber implizit über den gegenwärtigen Inhalt des Makros
% \cs{tud@bdl@curr}. Dafür wird das Makro \cs{@tempa} so definiert, dass es
% das angegebene Argument in einer Box ausführt und das gesuchte Bundle-Element
% in \cs{@tempb} gespeichert ist.
%    \begin{macrocode}
  \def\@tempa##1{%
    \sbox\z@{%
      \let\tud@idx@fmt\relax%
      \@tud@changedattrue%
      ##1%
      \@tud@changedatfalse%
      \ifx\tud@idx@fmt\relax%
        \gdef\@tempb{##1}%
      \else%
        \global\let\@tempb\tud@idx@fmt%
      \fi%
    }%
  }%
  \let\@tempb\@empty%
  \IfValueTF{#2}{%
    \@tempa{#2}%
  }{%
    \tud@if@bdl{\@tempa{\tud@bdl@curr}}{}%
  }%
%    \end{macrocode}
% Nachdem klar ist, für welches Bundle-Element ggf. der Änderungslisteneintrag 
% erzeugt werden soll, wird nun bestimmt, ob dieser im allgemeinen Teil oder in 
% den zur Implementierung erscheinen soll. Dies wird an der Existenz des ersten
% Argumentes entschieden. Ist es vorhanden, handelt es sich um einen Eintrag 
% für den Implementierungsteil, welcher in \cs{tud@changedat@list} in der Form 
% \newline\begingroup\scriptsize
%   \meta{Versionsnummer}|!Implementierung!|\meta{Markup-Befehl}\ignorespaces%
%   |!|\meta{Untereintrag}|!|\meta{Änderungsnotiz}\ignorespaces%
% \endgroup\newline
% gespeichert wird, wobei der Untereintrag auch leer bleiben kann.
%    \begin{macrocode}
  \IfValueTF{#1}{%
    \toks@{#3!Implementierung }%
%    \end{macrocode}
% Hier erfolgt ggf. der Eintrag des \TUDScript-Bundle-Elements, danach folgt 
% der Markup-Befehl.
%    \begin{macrocode}
    \eaddto@hook\toks@{\@tempb}%
    \addto@hook\toks@{!#1!}%
%    \end{macrocode}
% Wurde ein Untereintrag angegeben, wird dieser hier gefolgt von der 
% eigentlichen Änderungsnotiz eingefügt. Für obsolete Deklarationen für die 
% Änderungsnotiz in jedem Fall ignoriert, da beim späteren Aufruf des Makros  
% \cs{Changed@At@CreateEntry} eine Standardmeldung ausgegeben wird.
%    \begin{macrocode}
    \IfValueT{#4}{\addto@hook\toks@{#4}}%
    \addto@hook\toks@{!}%
    \if@tud@obsolete\else%
      \IfValueT{#5}{\addto@hook\toks@{#5}}%
    \fi%
  }{%
%    \end{macrocode}
% Handelt es sich um einen allgemeinen Änderungseintrag, wird dieser in der Form
% \newline\begingroup\scriptsize
%   \meta{Versionsnummer}|!Allgemein!|\meta{Änderungsnotiz}\ignorespaces%
% \endgroup\newline
% in der Liste \cs{tud@changedat@list} gespeichert. Für den allgemeinen Teil 
% wird mit der Angabe eines \TUDScript-Bundle-Elements etwas anders verfahren.
% Wurde dieses~-- wenn auch nur indirekt über eine der beiden Spezialumgebungen
% \env{Declaration*} bzw. \env{Bundle*}~-- angegeben, so wird der Eintrag in
% den allgemeinen Hauptteil als Untereintrag erzeugt. Erfolgte die Verwendung
% jedoch innerhalb einer \TUDScript-Bundle-Deklaration, so erscheint der
% Eintrag im allgemeinen Teil für das explizite Bundle-Element.
%    \begin{macrocode}
    \toks@{#3!Allgemein }%
    \IfValueTF{#2}{%
      \addto@hook\toks@{!}%
      \eaddto@hook\toks@{\@tempb:\space}%
      \IfValueTF{#5}{%
        \addto@hook\toks@{#5}%
      }{%
        \addto@hook\toks@{\emph{neu}}%
      }%
    }{%
      \eaddto@hook\toks@{\@tempb}%
      \addto@hook\toks@{!}%
      \addto@hook\toks@{#5}%
    }%
    \IfValueT{#4}{%
      \ClassError{tudscrmanual}{Using !#4 isn't allowed for a change notice}{%
        At least, it has to be implemented!%
      }%
    }%
  }%
  \listeadd\tud@changedat@list{\the\toks@}%
}
%    \end{macrocode}
% \end{macro}^^A \Changed@At@@@@CreateList
% \end{macro}^^A \Changed@At@@@CreateList
% \end{macro}^^A \Changed@At@@CreateList
% \end{macro}^^A \Changed@At@CreateList
%
% \subsubsection{Erzeugen der Änderungseinträge}
%
% Mit den nachfolgenden Makros erfolgt das eigentliche Eintragen der 
% Änderungseinträge aus der Liste \cs{tud@changedat@list} in den entsprechenden 
% Index~\val{changelog}. 
%
% \begin{macro}{\Changed@At@CreateEntry}
% \changes{v2.02}{2014/09/02}{neu}^^A
% \begin{macro}{\Changed@At@@CreateEntry}
% \changes{v2.02}{2014/09/02}{neu}^^A
% Der Befehl \cs{Changed@At@CreateEntry} wird durch \cs{@printchangedatlist} 
% aufgerufen und teilt die gegebene Liste am Delimiter Ausrufezeichen |!| in
% fünf Argumente auf und reicht sie an \cs{Changed@At@@CreateEntry} weiter.
% Sind weniger als vier Ausrufezeichen vorhanden, so werden die ungenutzten
% Argumente mit dem~-- durch das Paket \pkg{xparse} definierten~-- leeren Wert
% \val{-NoValue-} an den Befehl \cs{Changed@At@@CreateEntry} übergeben.
%    \begin{macrocode}
\NewDocumentCommand\Changed@At@CreateEntry{>{\SplitArgument{4}{!}}m}{%
  \Changed@At@@CreateEntry#1%
}
%    \end{macrocode}
% Mit \cs{Changed@At@@CreateEntry} wird der Eintrag in die Änderungshistorie
% erzeugt, falls denn überhaupt ein Eintrag generiert werden soll. Dies wird
% daran erkannt, ob das dritte Argument dew Wert \val{-NoValue-} entspricht 
% oder eben nicht. Der zweite Fall tritt auf, wenn lediglich eine Randnotiz
% gesetzt werden soll. Die in \cs{tud@changedat@list} gespeicherten Einträge 
% haben die Form 
% \newline\begingroup\scriptsize
%   \meta{Versionsnummer}|!Allgemein!|\meta{Änderungsnotiz}\ignorespaces%
% \endgroup\newline
% oder 
% \newline\begingroup\scriptsize
%   \meta{Versionsnummer}|!Implementierung!|\meta{Markup-Befehl}\ignorespaces%
%   |!|\meta{Untereintrag}|!|\meta{Änderungsnotiz}\ignorespaces%
% \endgroup
%    \begin{macrocode}
\newcommand*\Changed@At@@CreateEntry[5]{%
  \IfValueT{#3}{%
%    \end{macrocode}
% Bei der Deklaration von Optionen, Befehlen etc. werden diese automatisch in 
% der Liste der Änderungen im Bereich \enquote{Implementierung} vor der
% gegebenen Erklärung selbst ausgegeben. Hierfür wird die Existenz des vierten
% Arguments geprüft. Ist dieses vorhanden\dots
%    \begin{macrocode}
    \IfValueTF{#4}{%
%    \end{macrocode}
% \dots befindet sich im dritten Argument der Markup-Befehl für den Eintrag im
% Implementierungsteil. Mit dem zwischenzeitlichen Aktivieren des Schalters
% \cs{if@tud@changedat} wird beim nachfolgenden Aufruf des Markup-Befehls das 
% Makro \cs{Process@ChangedAt} ausgeführt und dadurch auch \cs{tud@idx@get*} 
% aufgerufen, wodurch in den Makros \cs{tud@idx@key}, \cs{tud@idx@fmt} und
% \cs{tud@idx@val} die passend formatierten Indexeinträge enthalten sind.
%    \begin{macrocode}
      \@tud@changedattrue%
      #3%
      \@tud@changedatfalse%
%    \end{macrocode}
% Das vierte Argument erstellt in der Änderungsliste einen Untereintrag für den
% im dritten Argument gegebenen Markup-Befehl. Dies wird für Anmerkungen zu 
% expliziten Schlüssel-Wert-Paaren bei Optionen oder Parametern genutzt. 
%    \begin{macrocode}
      \ifxblank{#4}{}{%
%    \end{macrocode}
% Hierfür wird die zuvor erhaltene Formatierung des Markup-Befehls des dritten
% Argumentes an den Haupteintrag \cs{tud@idx@val} mit der zusätzlichen
% Anmerkung \enquote{Werte angepasst} angehängt.
%    \begin{macrocode}
        \eappto\tud@idx@val{%
          \expandonce\tud@idx@key: Werte angepasst@%
          \expandonce\tud@idx@fmt: Werte angepasst!%
        }%
%    \end{macrocode}
% Anschließend werden die Formatierungsbefehle des vierten Arguments genutzt, 
% wobei vom Befehl \cs{suffix} nur das Argument selbst benötigt wird. Um den 
% Inhalt von \cs{tud@idx@val} zu schützen, erfolgt das ganze in einer Gruppe.
%    \begin{macrocode}
        \begingroup%
          \@tud@changedattrue%
          #4%
          \@tud@changedatfalse%
          \let\suffix\@gobble%
          \protected@edef\tud@reserved{%
            \noexpand\endgroup%
            \def\noexpand\tud@idx@key{\tud@idx@key}%
            \def\noexpand\tud@idx@fmt{\tud@idx@fmt}%
          }%
        \tud@reserved%
      }%
%    \end{macrocode}
% Nachdem die Formatierung der Haupt- und Untereinträge soweit zusammengebaut 
% wurden, wird die eigentliche Änderungsnotiz des fünften Arguments angehangen. 
% Dabei wird für den Fall, dass \emph{keine} Erklärung angegeben wurde, in der 
% Änderungsliste vermerkt, ob es sich bei dem Markup-Befehl respektive dessen 
% Untereintrag um eine \emph{neue} bzw. eine \emph{obsolete} Variante handelt.
%    \begin{macrocode}
      \ifxblank{#5}{%
        \if@tud@obsolete%
          \appto\tud@idx@fmt{: \emph{entf\"allt}}%
        \else%
          \appto\tud@idx@fmt{: \emph{neu}}%
        \fi%
%    \end{macrocode}
% Existiert ein Eintrag, wird dieser für Schlüssel und Format verwendet.
%    \begin{macrocode}
      }{%
        \appto\tud@idx@key{: #5}%
        \appto\tud@idx@fmt{: #5}%
      }%
%    \end{macrocode}
% Nachdem alles abgearbeitet wurde, wird der Indexeintrag erstellt.
%    \begin{macrocode}
      \index[changelog]{#1!#2!\tud@idx@val\tud@idx@key @\tud@idx@fmt}%
%    \end{macrocode}
% Existiert das vierte Argument nicht, handelt es sich um einen allgemeinen 
% Eintrag in der Änderungshistorie. Dieser wird direkt ausgeführt.
%    \begin{macrocode}
    }{%
      \index[changelog]{#1!#2!#3}%
    }%
  }%
%    \end{macrocode}
% Außerdem wird der Liste \cs{tud@templist} die aktuelle Version einmalig 
% hinzugefügt, um alle Versionsangaben in einer Randnotiz ausgeben zu können.
%    \begin{macrocode}
  \ifinlist{#1}{\tud@templist}{}{\listadd\tud@templist{#1}}%
}
%    \end{macrocode}
% \end{macro}^^A \Changed@At@@CreateEntry
% \end{macro}^^A \Changed@At@CreateEntry
%
% \subsubsection{Ausgabe der Änderungsliste}
%
% Ähnlich wie für den Index wird auch für die Änderungsliste bei der Ausgabe 
% verfahren. 
%
% \begin{macro}{\PrintChangelog}
% \begin{environment}{theindex}
% Mit \cs{PrintChangelog} erfolgt die Ausgabe der Änderungsliste. Dabei ist zu 
% beachten, dass die Einträge im Index bis zur vierten Ebene erfolgen.
%    \begin{macrocode}
\newcommand*\PrintChangelog{%
  \begingroup%
%    \end{macrocode}
% Die oberste Ebene enthält die Versionsnummer mit dem Präfix |v|. Diese Ebene
% soll nicht als Indexeintrag selber sondern vielmehr als Überschrift der Ebene
% \cs{subsection} genutzt werden. Dafür wird zum einen das durch das Paket
% \pkg{imakeidx} definierte Makro \cs{lettergroup} \emph{unschädlich} gemacht,
% zum anderen wird der Befehl \cs{item} so umdefiniert, dass dieser alles bis
% zum nächsten Zeilenumbruch als Argument einliest. Alle darunter liegenden
% \cs{item}-Befehle werden eine Ebene nach oben gehoben.
%    \begin{macrocode}
    \let\lettergroup\@gobble%
    \apptocmd{\theindex}{%
      \let\subsubsubitem\subsubitem%
      \let\subsubitem\subitem%
      \let\subitem\item%
      \renewcommand*\item{%
        \begingroup%
          \escapechar=`\\%
          \catcode\endlinechar=\active%
          \csname\string\item\endcsname%
      }%
      \begingroup%
      \escapechar=`\\%
      \lccode`\~=\endlinechar%
      \lowercase{%
        \expandafter\endgroup%
        \expandafter\def\csname\string\item\endcsname##1~%
      }{%
        \endgroup%
        \let\@tempb\@firstofone%
        \if@tud@cdoldfont@active\def\@tempb##1{\NoCaseChange{##1}}\fi%
        \addsec[##1]{\TUDScript \@tempb{##1}}%
        \tudhyperdef*{idx:changelog:##1}%
        \def\@tempa{\indexname: \TUDScript \@tempb{##1}}%
        \@mkboth{\@tempa}{\@tempa}%
      }%
    }{}{\tud@patch@wrn{theindex}}%
%    \end{macrocode}
% Mit der so angepassten \env{theindex}-Umgebung erfolgt die Ausgabe der 
% Änderungshistorie.
%    \begin{macrocode}
    \clearpage%
    \def\imki@indexlevel{\addchap}%
    \print@index[changelog]%
  \endgroup%
}
%    \end{macrocode}
% \end{environment}^^A theindex
% \end{macro}^^A \PrintChangelog
% \begin{macro}{\print@changedatlist}
% \changes{v2.02}{2014/10/09}{neu}^^A
% Die Einträge in der Liste der Änderungen \cs{tud@changedat@list} werden mit 
% dem Makro \cs{Changed@At@CreateList} abgearbeitet, sortiert und erstellt. Das 
% zweite obligatorische Argument von \cs{@printchangedatlist} steurrt den 
% vertikalen Versatz der Randnotiz. Wird im ersten Argument das boolesche Flag 
% zu \cs{BooleanTrue} gesetzt, wird die Randnotiz unterdrückt.
%    \begin{macrocode}
\NewDocumentCommand\print@changedatlist{m m}{%
  \ifx\tud@changedat@list\relax\else%
    \let\tud@templist\relax%
    \forlistloop\Changed@At@CreateEntry{\tud@changedat@list}%
    \global\let\tud@changedat@list\relax%
%    \end{macrocode}
% Nach der Verarbeitung der Einräge sind in \cs{tud@templist} alle angegebenen
% Versionsnummern genau einmal enthalten. Mit \cs{tud@list@sort} werden diese 
% in die richtige Reihenfolge sortiert und anschließend als Randnotiz in einer
% Tabelle ausgegeben, falls nicht die Sternversion verwendet wurde.
%    \begin{macrocode}
    \IfBooleanF{#1}{%
      \tud@list@sort\tud@templist%
      \strut%
      \marginnote{%
        \def\@tempa####1{%
          \ignorespaces%
          \tudhyperref{idx:changelog:####1}{####1}%
          \tabularnewline%
        }%
        \begin{tabular}{|l|}%
          \hline%
          \forlistloop\@tempa{\tud@templist}%
          \hline%
        \end{tabular}%
      }[#2]%
      \ignorespaces%
    }%
  \fi%
}
%    \end{macrocode}
% \end{macro}^^A \print@changedatlist
%
% \subsection{Erweiterte Listen für Erläuterungen im Handbuch}
%
% Es folgen einige Umgebungen, mit welchen bestimmte Aspekte oder Pakete im
% Handbuch erläutert werden.
%
% Das Paket \pkg{enumitem} erlaubt das Beeinflussen der Standardlisten und die 
% Ableitung neuer Listen aus diesen. Mit \cs{setlist\{noitemsep\}} werden die 
% zusätzlichen Zwischenräume innerhalb der Listen beseitigt. Weiterhin wird 
% linksbündiger Flattersatz für die Standardlisten aktiviert.
%    \begin{macrocode}
\RequirePackage{enumitem}[2011/09/28]
\setlist{before*={\endgraf\tud@RaggedRight},partopsep=0pt,noitemsep}
%    \end{macrocode}
% \begin{macro}{\stditem}
% \begin{macro}{\tud@before@item}
% Mit den Möglichkeiten des Paketes \pkg{enumitem} werden folgend Listen für 
% das Aufführen von möglichen Werten einer Option sowie der Beschreibung von 
% nützlichen Paketen erstellt. Für beide Listen soll dabei die originale 
% Definition von \cs{item} in \cs{stditem} gesichert und anschließend angepasst
% werden. Die Sicherung und Redefinition von \cs{item} erfolgt dabei über das 
% Makro \cs{tud@before@item} über einen bereitgestellten \emph{hook} zu Beginn
% der jeweiligen Liste.
%    \begin{macrocode}
\newcommand*\stditem{}
\newcommand*\tud@before@item[1]{%
  \ifdefvoid{\stditem}{%
    \let\stditem\item%
    \letcs\item{tud@item@#1}%
%    \end{macrocode}
% Damit die Änderungsnotizen in der passenden vertikalen Position erscheinen, 
% wird die Länge \cs{changedatskip} entsprechend angepasst.
%    \begin{macrocode}
    \setlength\changedatskip{-\baselineskip}%
  }{}%
}
%    \end{macrocode}
% \end{macro}^^A \tud@before@item
% \end{macro}^^A \stditem
%
% \subsubsection{Listen für gültige Werte von Optionen}
%
% \begin{environment}{values}
% \begin{environment}{@values}
% \begin{macro}{\tud@values}
% Die Liste \env{@values} dient zum Benennen der möglichen Werte für einen 
% bestimmten Schlüssel. Im Handbuch kommt diese Liste vor allem bei Optionen 
% aber ggf. auch bei bestimmten Befehls- und Optionsparametern zum Einsatz. 
% Diese wird in der Umgebung \env{values} eingebettet, damit zum einen ein 
% \emph{obligatorisches} Argument verwendet werden kann und zum anderen 
% \emph{nach} diesem ein optionales für die Liste selbst verwendbar ist. Das 
% obligatorische Argument wird in \cs{tud@values} gesichert und sollte die 
% genauer zu beschreibende Option bzw. das Makro etc. enthalten.
%    \begin{macrocode}
\newcommand*\tud@values{}
\NewDocumentEnvironment{values}{mo}{%
  \def\tud@values{#1}%
  \IfValueTF{#2}{\@values[#2]}{\@values}%
}{%
  \end@values%
}
\newlist{@values}{description}{1}
\setlist[@values]{%
  topsep=\smallskipamount,labelwidth=\linewidth,labelsep=0pt,%
  font=\normalfont,parsep=\parskip,listparindent=\parindent,%
  before=\tud@before@item{values}%
}
%    \end{macrocode}
% \end{macro}^^A \tud@values
% \end{environment}^^A @values
% \end{environment}^^A values
% \begin{macro}{\tud@item@values}
% \begin{macro}{\forslashlist}
% Das Makro \cs{tud@item@values} ersetzt in der Umgebung \env{values} die 
% originale Definiton von \cs{item}. Die Sternversion kann dabei verwendet 
% werden, um die Einstellung für den Säumniswert hervorzuheben. Im ersten 
% optionalen Argument können die möglichen Werte einer Option oder eines 
% Parameters angegeben werden. Sind mehrere Werte gleichbedeutende Werte
% verwendbar, so sind diese durch |/| zu trennen. Dabei wird der als erstes
% angegebene Wert für den Untereintrag in den Index verwendet. Das zweite
% optionale Argument kann genutzt werden, falls beispielsweise für
% unterschiedliche Klassen sich der Säumniswert unterscheidet. Das letzte 
% optionale Argument in runden Klammern fügt zum Schluss einen ergänzenden 
% Eintrag hinzu.
%    \begin{macrocode}
\DeclareListParser*{\forslashlist}{/}
\NewDocumentCommand\tud@item@values{s o o d()}{%
%    \end{macrocode}
% Alle Auszeichnungen kommen nur zum Tragen, wenn ein optionales Argument für 
% die Werte angegeben wurde.
%    \begin{macrocode}
  \IfValueTF{#2}{%
%    \end{macrocode}
% Die durch Schrägstrich getrennte Liste im optionalen Argument wird mit 
% \cs{forslashlist} durchlaufen. Der erste Eintrag wird für den Indexeintrag 
% benötigt und in \cs{@tempc} gesichert, alle weiteren Einträge werden in 
% \cs{@tempb} gespeichert und lediglich im Fließtext aufgelistet.
%    \begin{macrocode}
    \let\@tempc\@empty%
    \def\tud@reserved##1{%
      \if@tempswa%
        \appto\@tempb{/##1}%
      \else%
        \@tempswatrue%
        \let\@tempb\@empty%
        \def\@tempc{##1}%
      \fi%
    }%
    \@tempswafalse%
    \forslashlist\tud@reserved{#2}%
%    \end{macrocode}
% Jetzt beginnt die Formatierung der Ausgabe. Zuerst wird die zu beschreibende 
% Option in \cs{@tempa} expandiert. Danach wird der erste Wert angehangen, 
% wobei dieser für die Sternversion als Säumniswert mit \cs{emph} hervorgehoben 
% wird. Zuletzt wird dafür Sorge getragen, dass auf den anschließend erzeugten 
% Eintrag selbst kein Hyperlink erzeugt wird und dieser im passenden Index als 
% fettmarkierter Eintrag erscheint.
%    \begin{macrocode}
    \edef\@tempa{\expandonce\tud@values}%
    \protected@eappto\@tempa{%
      =\IfBooleanTF{#1}{\noexpand\emph}{\@firstofone}{\@tempc}=%
    }%
    \appto\@tempa{'none'|!|}%
%    \end{macrocode}
% Danach werden alle weiteren möglichen Werte angegeben, wobei auch diese im 
% Zweifelsfall mit \cs{emph} hervorzuheben sind.
%    \begin{macrocode}
    \edef\@tempb{%
      \IfBooleanTF{#1}{\noexpand\emph}{\@firstofone}{%
        \noexpand\PValue{\@tempb}%
      }%
    }%
%    \end{macrocode}
% Zuletzt erfolgt die Auszeichnung, ob es sich bei den gelisteten Werten um 
% Säumniswerte handelt, gefolgt vom optionael Suffix.
%    \begin{macrocode}
    \IfBooleanTF{#1}{%
      \appto\@tempb{\suffix{S\"aumniswert}}%
    }{%
      \IfValueT{#3}{\appto\@tempb{\suffix{S\"aumniswert f\"ur #3}}}%
    }%
    \IfValueT{#4}{\appto\@tempb{\suffix{#4}}}%
%    \end{macrocode}
% Die in \cs{tud@values} gespeicherte, zu beschreibende Option bzw. Makro etc. 
% und der erste im optionalen Argument angegebene Wert werden verwendet, um ein
% Label zu erstellen.
%    \begin{macrocode}
    \tud@lbl@get@curr*\tud@lbl@tmp{\tud@bdl@curr:\tud@values}=\@tempc=%
%    \end{macrocode}
% Abschließend erfolgt wird die Ausgabe des generierten Eintrags mit dem 
% originalen Makro \cs{stditem}.
%    \begin{macrocode}
    \stditem[\tudhyperdef{\tud@lbl@tmp}\@tempa\@tempb]%
  }{%
    \stditem%
  }%
}
%    \end{macrocode}
% \end{macro}^^A \forslashlist
% \end{macro}^^A \tud@item@values
% \begin{macro}{\itemtrue}
% \begin{macro}{\itemfalse}
% \begin{macro}{\tud@item@bool}
% Die folgenden beiden Befehle sind äquivalent zu \cs{tud@item@values} genutzt 
% werden. Diese enthalten allerdings im optionalen Argument zumindest die
% Standardwerte für positive bzw. negative boolesche Schalter. Diese werden 
% durch den Befehl \cs{item@bool} gesetzt.
%    \begin{macrocode}
\NewDocumentCommand\itemtrue{!s !o !o !d()}{%
  \item@bool{#1}{true/on/yes}[#2][#3](#4)%
}
\NewDocumentCommand\itemfalse{!s !o !o !d()}{%
  \item@bool{#1}{false/off/no}[#2][#3](#4)%
}
%    \end{macrocode}
% Der Befehl \cs{item@bool} setzt je nach angegebenen Argumenten über ein
% token-Register den auszuführenden \cs{item}-Befehl zusammen.
%    \begin{macrocode}
\NewDocumentCommand\item@bool{m m r[] r[] r()}{%
  \toks@{\item}%
  \IfBooleanT{#1}{\addto@hook\toks@{*}}%
%    \end{macrocode}
% Die optionalen Werte werden im Zweifel den booleschen Standardwerten 
% angehängt und als einziges optionales Argument an \cs{item} übergeben.
%    \begin{macrocode}
  \protected@edef\@tempa{#2\IfValueT{#3}{/#3}}%
  \eaddto@hook\toks@{\expandafter[\@tempa]}%
  \IfValueT{#4}{\addto@hook\toks@{[#4]}}%
  \IfValueT{#5}{\addto@hook\toks@{(#5)}}%
  \the\toks@%
}
%    \end{macrocode}
% \end{macro}^^A \itemfalse
% \end{macro}^^A \itemtrue
% \end{macro}^^A \tud@item@bool
%
% \subsubsection{Listen für die Beschreibung von Paketen}
%
% \begin{environment}{packages}
% \begin{macro}{\tud@item@packages}
% \changes{v2.02}{2014/10/09}{optionales Argument für CTAN-Link}^^A
% \changes{v2.04}{2015/03/12}{optionales Argument für weitere Pakete}^^A
% Ähnlich wie für die Auflistung von Werten eines Schlüssels wird auch mit der 
% Charakterisierung von Paketen verfahren. Diese werden in der Auflistung 
% speziell gekennzeichnet.
%    \begin{macrocode}
\newlist{packages}{description}{2}
\setlist[packages]{%
  topsep=\smallskipamount,labelwidth=\linewidth,labelsep=0pt,%
  font=\normalfont,parsep=\parskip,listparindent=\parindent,%
  before=\tud@before@item{packages}%
}
%    \end{macrocode}
% Hiermit können ein oder mehrere Pakete erläutert werden. Des Weiteren werden 
% Textanker definiert, auf die im Zweifelsfall in der Dokumentation verlinkt 
% wird.
%    \begin{macrocode}
\NewDocumentCommand\tud@item@packages{o d<> d()}{%
%    \end{macrocode}
% Die kommagetrennte Liste im optionalen Argument wird verwendet, um jedes
% einzelnen Paket aufzulisten und einen dazugehörigen Textanker zu erzeugen. 
% Das optionale Argument in Guillemets kann genutzt werden, um \emph{alle} 
% gelisteten Pakete auf den gleichen CTAN-Link zielen zu lassen. Dies ist 
% beispielsweise für mehrere Pakete aus dem gleichen Bundle sinnvoll.
%    \begin{macrocode}
  \def\@tempa##1{%
    \@tempc%
    \tud@lbl@get@curr*\tud@lbl@tmp{\tud@bdl@curr:\Package{##1}}%
    \tudhyperdef{\tud@lbl@tmp}%
    \begingroup%
      \Package{##1}<#2>'url'|!|%
    \endgroup%
    \def\@tempc{, }%
  }%
  \stditem[{%
    \IfValueT{#1}{%
      \let\@tempc\relax%
      \forcsvlist\@tempa{#1}%
%    \end{macrocode}
% Wird im dritten optionalen Argument in runden Klammern ein Paket angegeben,
% so wird ebenfalls ein Textanker erzeugt, allerdings auf einen Hyperlink zu
% CTAN verzichtet.
%    \begin{macrocode}
      \IfValueT{#3}{%
        \tud@lbl@get@curr*\tud@lbl@tmp{\tud@bdl@curr:\Package{#3}}%
        \quad(\tudhyperdef{\tud@lbl@tmp}\Package{#3}'none'|!|)%
      }%
    }%
  }]%
}
%    \end{macrocode}
% \end{macro}^^A \tud@item@packages
% \end{environment}^^A packages
%
% \subsection{Erläuterungen und weiterführende Ergänzungen im Handbuch}
% \subsubsection{Tabelle mit automatisch berechneter Mehrspaltenzellenbreite}
%
% \begin{environment}{tabularm}
% \begin{macro}{\tabularm@allcolumnpar}
% \begin{macro}{\tabularm@box}
% \begin{macro}{\tabularm@num}
% \begin{length}{\tabularm@wd}
% \begin{length}{\tempdim}
% \changes{v2.02}{2014/08/16}{neu}^^A
% Die Umgebung \env{tabularm} ist eine Erweiterung der \env{tabular}-Umgebung.
% In dieser wird die Breite der zu setzenden Tabelle mit Hilfe der Box 
% \cs{tabularm@box} in \cs{tabularm@wd} gespeichert. Anschließend kann mit dem
% Befehl \cs{allcolumnpar} eine Zeile über alle Spalten der Tabelle ausgegeben 
% werden. Dabei ist das zu übergebende Argument die Anzahl der zu erzeugenden, 
% linksbündigen Spalten.
%    \begin{macrocode}
\newlength\tempdim
\newsavebox\tabularm@box
\newlength\tabularm@wd
\newcommand*\tabularm@num{1}
\NewEnviron{tabularm}[1]{%
  \begin{lrbox}{\tabularm@box}%
    \let\allcolumnpar\@gobble%
    \begin{tabular}{*{#1}l}\BODY\end{tabular}%
  \end{lrbox}%
  \setlength\tabularm@wd{\wd\tabularm@box}%
  \def\tabularm@num{#1}%
  \let\allcolumnpar\tabularm@allcolumnpar%
  \begin{tabular}{*{#1}l}\BODY\end{tabular}%
}
\newcommand*\tabularm@allcolumnpar[1]{%
  \multicolumn{\tabularm@num}{@{}p{\tabularm@wd}@{}}{#1}%
}
%    \end{macrocode}
% \end{length}^^A \tempdim
% \end{length}^^A \tabularm@wd
% \end{macro}^^A \tabularm@box
% \end{macro}^^A \tabularm@num
% \end{macro}^^A \tabularm@allcolumnpar
% \end{environment}^^A tabularm
%
% \subsubsection{Tabelle für mehrsprachigen Bezeichner}
%
% Mit Hilfe von \pkg{tabularx} können Tabellen bestimmter Breite und
% automatisch berechneten Spaltenbreiten gesetzt werden.
%    \begin{macrocode}
\RequirePackage{tabularx}[1999/01/07]
%    \end{macrocode}
% \begin{macro}{\TermTable}
% \begin{macro}{\Term@Table}
% \begin{macro}{\Term@@Table}
% Für sprachabhängige Bezeichner wird zusätzlich der Befehl \cs{TermTable} 
% definiert, dem eine kommaseparierte Liste übergeben werden kann. Anhand 
% dieser Liste werden die enthalten Begriffe für die Sprachen Deutsch und
% English in einer Tabelle ausgegeben. Die Sternversion dieses Befehls nutzt
% dafür die \env{tabularx}-Umgebung. Für diese kann ein zweites Argument
% angegeben werden, welches die gewünschte Breite der Tabelle angibt.
%    \begin{macrocode}
\NewDocumentCommand\TermTable{s m !g}{%
  \let\tud@templist\relax%
  \forcsvlist{\listadd\tud@templist}{#2}%
  \ifhmode\par\fi%
  \vskip\medskipamount%
  \@afterindentfalse\@afterheading%
  \IfBooleanTF{#1}{%
    \IfValueTF{#3}{\setlength\@tempdima{#3}}{\setlength\@tempdima{\linewidth}}%
    \setlength\@tempdimc{-.7\baselineskip}%
    \begin{tabularx}{\@tempdima}{lXX}\Term@Table\end{tabularx}%
  }{%
    \setlength\@tempdimc{\z@}%
    \begin{tabular}{lll}\Term@Table\end{tabular}%
  }%
  \par\addvspace{\medskipamount}%
}
\newcommand*\Term@Table{%
  \toprule%
  \textbf{Bezeichner} & \textbf{Deutsch} & \textbf{Englisch}\tabularnewline%
  \midrule%
  \forlistloop\Term@@Table{\tud@templist}%
  \bottomrule%
}
\newcommand*\Term@@Table[1]{%
  \Term{#1}'none' & % 
  \ifstr{\csuse{#1}}{}{\PName{leer}}{\csuse{#1}} & %
  \selectlanguage{english}\vspace*{\@tempdimc}%
  \ifstr{\csuse{#1}}{}{\PName{empty}}{\csuse{#1}}\tabularnewline%
}
%    \end{macrocode}
% \end{macro}^^A \Term@@Table
% \end{macro}^^A \Term@Table
% \end{macro}^^A \TermTable
%
% \subsubsection{Umgebung für die Angabe eines Beispiels}
%
% \begin{environment}{Example}
% \begin{environment}{Example*}
% Diese Umgebung wird für die exemplarische Erläuterung von bestimmten Aspekten 
% im Handbuch verwendet. Wenn dafür ein Quelltextauszug nötig ist, kann 
% innerhalb dieser die \env{Code}-Umgebung genutzt werden. Die Sternversion 
% unterdrückt dabei das Zurücksetzen des eigentlich entfernten vertikalen 
% Abstandes.
%    \begin{macrocode}
\newenvironment{Example}{\csuse{Example*}}{%
  \csuse{endExample*}%
  \addvspace{-\topsep}%
}
\newenvironment{Example*}{%
  \labeling{{\usekomafont{disposition}{Beispiel:\nobreakspace}}}%
  \item[{\usekomafont{disposition}{Beispiel:\nobreakspace}}]%
}{%
  \endlabeling%
}
%    \end{macrocode}
% \end{environment}^^A Example*
% \end{environment}^^A Example
%
% \iffalse
%</class>
% \fi
%
% \subsection{Lesen und Schreiben von Hilfsdateien}
%
% Für den Anwender werden im \TUDScript-Bundle einige Tutorials mit Hinweisen 
% zur Nutzung von \LaTeXe{} bereitgestellt. Innerhalb dieser werden Auszüge von
% Quelltexten verwendet und erläutert. Zum einen soll die Möglichkeit 
% geschaffen werden, diese gesammelt am Ende eines Tutorials für ein
% Copy"~\&"~Paste"~Beispiel zu verwenden. Andererseits soll auch das sofortige
% Ausführen des beschriebenen Quelltextauszuges direkt im Tutorial möglich
% sein.
% 
% Dafür werden im Folgenden die Umgebungen \env{Preamble}, \env{Trunk} und
% \env{Hint} sowie daraus abgeleitet Varianten definiert, womit sich die Angabe
% und Erläuterung der Quelltexte im Tutorial logisch strukturieren lässt und
% dennoch die notwendige Ordnung dieser für das Copy"~\&"~Paste"~Beispiel
% aufrechterhalten werden kann. Hierfür zeichnen die beiden Umgebungen
% \env{Preamble} und \env{Trunk} und deren Derivate verantwortlich. Die
% Umgebung \env{Hint} steht für zusätzliche Erläuterungen bereit, welche jedoch
% nicht in das Copy"~\&"~Paste"~Beispiel übernommen werden.
%
% \begin{macro}{\tud@write@a}
% \changes{v2.02}{2014/10/22}{neu}^^A
% \begin{macro}{\tud@write@b}
% \changes{v2.02}{2014/11/02}{neu}^^A
% \begin{macro}{\tud@read}
% \changes{v2.02}{2014/08/19}{neu}^^A
% Für die Erstellung der Indexe wird zum Sortieren das Programm \app{texindy} 
% genutzt. Für dieses wird eine eigene Stildatei verwendet, welche zur Laufzeit 
% erzeugt wofür zuerst ein Ausgabestream reserviert wird.
%    \begin{macrocode}
\newcommand*\tud@write@a{}
\newwrite\tud@write@a
%    \end{macrocode}
% Für die Tutorials werden später zusätzlich noch ein weiterer Ausgabe- sowie 
% ein Eingabestream benötigt, die hier gleich mit initialisiert werden.
%    \begin{macrocode}
%<*package>
\newcommand*\tud@write@b{}
\newwrite\tud@write@b
\newcommand*\tud@read{}
\newread\tud@read
%</package>
%    \end{macrocode}
% \end{macro}^^A \tud@read
% \end{macro}^^A \tud@write@b
% \end{macro}^^A \tud@write@a
%
% \subsection{Einbinden von Quelltextauszügen}
%
% Sowohl im Handbuch als auch in den Tutorials werden Quelltextauszüge für das 
% bessere Verständnis des Anwenders angegeben. Um diese darzustellen wird das 
% Paket \pkg{listings} verwendet.
%    \begin{macrocode}
\RequirePackage{listings}[2014/03/04]
\lstdefinestyle{tudscr}{%
  basicstyle=\ttfamily\ifdef{\setstretch}{\setstretch{1}}{},%
  keywordstyle=,%
  columns=fullflexible,keepspaces,%
  aboveskip=\smallskipamount,%
  belowskip=\smallskipamount,%
  language=[LaTeX]TeX,%
  texcl,%
}
%    \end{macrocode}
%
% \iffalse
%<*class>
% \fi
%
% \begin{environment}{Code}
% \begin{macro}{\tud@currentHref}
% Folgend werden Befehle und Umgebungen für die Darstellung von Quelltexten im 
% Handbuch definiert. Mit dieser Umgebung können kurze Quelltextbeispiele 
% ausgegeben werden. Sicherlich sinnvoll ist, die \env{Code}-Umgebung innerhalb
% von \env{Example} oder \env{quoting} zu verwenden, um den Quelltext etwas
% abzuheben.
%    \begin{macrocode}
\lstnewenvironment{Code}[1][]{\lstset{style=tudscr,#1}}{}
%    \end{macrocode}
% Innerhalb der Umgebung \env{Code} wird \cs{@currentHref} unerwünschter Weise
% geändert. Um dies zu vermeiden, wird dieser Befehl gesichert und anschließend
% zurückgesetzt.
%    \begin{macrocode}
\newcommand*\tud@currentHref{}
\BeforeBeginEnvironment{Code}{\global\let\tud@currentHref\@currentHref}
\AfterEndEnvironment{Code}{\global\let\@currentHref\tud@currentHref}
%    \end{macrocode}
% \end{macro}^^A \tud@currentHref
% \end{environment}^^A Code
%
% \iffalse
%</class>
% \fi
%
% \subsection{Kompilieren von externen Dateien und Querverweise auf diese}
%
% Das Paket \pkg{filemod} wird genutzt, um externer Dateien nur zu kompilieren,
% wenn dies aufgrund einer Änderung auch notwendig ist.
%    \begin{macrocode}
\RequirePackage{filemod}[2011/09/19]
%    \end{macrocode}
% Um \cs{ShellEscape} sowohl für Windows als auch für unixoide Betriebssysteme 
% nutzen zu können, wird das Paket \pkg{ifplatform} geladen.
%    \begin{macrocode}
\RequirePackage{ifplatform}[2010/10/22]
%    \end{macrocode}
% \begin{macro}{\tud@shellescape@wrn}
% \changes{v2.02}{2014/10/14}{neu}^^A
% Dieses Makro wird verwendet, wenn eine gesuchte Datei nicht gefunden wurde, 
% weil das Dokument noch nicht mit dem Schalter \file{-{}-shell-escape} 
% kompiliert wurde.
%    \begin{macrocode}
\newcommand*\tud@shellescape@wrn[1]{%
%<*class>
  \ClassWarning{tudscrmanual}%
%</class>
%<*package>
  \PackageWarning{tudscrtutorial}%
%</package>
  {%
    File `#1' is missing!\MessageBreak%
    You have to recompile this document with\MessageBreak%
    activated shell escape option%
  }%
}
%    \end{macrocode}
% \end{macro}^^A \tud@shellescape@wrn
% \begin{macro}{\tud@latex@ext}
% \changes{v2.06}{2019/06/21}{neu}^^A
% Für das verschachtelte Aufrufen von \app{pdflatex} bzw. \app{lualatex}.
%    \begin{macrocode}
\newcommand*\tud@latex@ext{}
\ifpdftex%
  {\def\tud@latex@ext{pdflatex\space}}%
  {\def\tud@latex@ext{lualatex\space}}%
%    \end{macrocode}
% \end{macro}^^A \tud@latex@ext
% \begin{macro}{\Tutorial}
% \changes{v2.02}{2014/08/22}{neu}^^A
% \changes{v2.02}{2014/10/22}{Reimplementierung}^^A
% \begin{macro}{\hypertut}
% \changes{v2.02}{2014/09/02}{neu}^^A
% Die gegebenenfalls notwendige Kompilierung und die Referenzierung eines 
% Tutorials aus dem Handbuch erfolgt mit \cs{Tutorial}. Außerdem erfolgt mit
% \cs{hypertut} ein verlinkter Eintrag in den Index.
%    \begin{macrocode}
%<*class>
\newrobustcmd*\hypertut[2]{%
  \hyperref{tutorials/#1.pdf}{#1}{tutorials:#1}{\File*{#1.pdf}#2}%
}
\NewDocumentCommand\Tutorial{s m !d||}{%
  \ifnum\pdf@shellescape=\@ne\relax%
%    \end{macrocode}
% Um mit den verschachtelten Aufrufen von \app{pdflatex} umgehen zu können,
% wird das ganze Prozedere in ein Skript ausgelagert. Somit können die Aufrufe
% von \app{pdflatex} für die Tutorials aus dem entsprechenden Ordner und mit
% den notwendigen Optionen erfolgen.
%    \begin{macrocode}
    \filemodCmp{tutorials/#2.pdf}{tutorials/#2.tex}{}{%
      \let\@tempa\@empty%
      \ifdef{\tudfinalflag}{\appto\@tempa{\def\noexpand\tudfinalflag{}}}{}%
      \ifdef{\tudprintflag}{\appto\@tempa{\def\noexpand\tudprintflag{}}}{}%
      \appto\@tempa{\noexpand\input{#2.tex}}%
      \immediate\openout\tud@write@a=tutorials.sh\relax%
      \def\@tempb##1{\immediate\write\tud@write@a{##1}}%
      \@tempb{cd tutorials}%
      \@tempb{\tud@latex@ext -shell-escape "\@tempa"}%
      \ifwindows%
        \@tempb{if exist #2.bcf biber #2}%
      \else%
        \@tempb{[ -f #2.bcf ] && biber #2}%
      \fi%
      \@tempb{\tud@latex@ext "\@tempa"}%
      \@tempb{\tud@latex@ext -shell-escape "\@tempa"}%
      \@tempb{\tud@latex@ext -shell-escape "\@tempa"}%
      \immediate\closeout\tud@write@a%
%    \end{macrocode}
% Hier erfolgt die Unterscheidung der Befehle für die unterschiedlichen OS.
%    \begin{macrocode}
      \ifwindows%
        \ShellEscape{rename tutorials.sh tutorials.bat}%
        \ShellEscape{tutorials.bat}%
        \ShellEscape{del tutorials.bat}%
      \else%
        \ShellEscape{bash tutorials.sh}%
        \ShellEscape{rm tutorials.sh}%
      \fi%
    }%
  \fi%
%    \end{macrocode}
% Hyperlink und Indexeintrag.
%    \begin{macrocode}
  \IfFileExists{tutorials/#2.pdf}{%
    \hypertut{#2}{}%
  }{%
    \File*{#2.pdf}%
    \tud@shellescape@wrn{tutorials/#2.pdf}%
  }%
  \IfBooleanF{#1}{%
    \index[files]{\hypertut{#2}{\suffix{Tutorial}}\IfValueT{#3}{|#3}}%
  }%
}
%</class>
%    \end{macrocode}
% Der Befehl für die Querverweise innerhalb von Tutorials.
%    \begin{macrocode}
%<*package>
\newcommand*\Tutorial[1]{\href{#1.pdf}{\textsbn{#1.pdf}}}
%</package>
%    \end{macrocode}
% \end{macro}^^A \hypertut
% \end{macro}^^A \Tutorial
% \begin{macro}{\IncludeExample}
% \changes{v2.02}{2014/07/25}{Skalierung der eingefügten Seiten}^^A
% Mit diesem Befehl wird eine \LaTeX-Datei ggf. übersetzt und anschließend als
% PDF"~Datei wieder eingebunden. Zuvor wird der dazugehörige Quelltext mittels
% \cs{lstinputlisting} im Dokument ausgegeben. Dieser Befehl wird lediglich für 
% die Klasse \cls{tudscrmanual} benötigt.
%    \begin{macrocode}
%<*class>
\newcommand*\IncludeExample[1]{%
  \lstinputlisting[style=tudscr]{examples/#1}%
%    \end{macrocode}
% Die Beispiele werden nur erneut kompiliert, wenn sich diese geändert haben 
% oder noch gar kein entsprechendes PDF"~Dokument existiert.
%    \begin{macrocode}
  \ifnum\pdf@shellescape=\@ne\relax%
    \filemodCmp{examples/#1.pdf}{examples/#1.tex}{}{%
      \edef\@tempa{\tud@latex@ext -output-directory examples examples/#1.tex}%
      \ShellEscape{\@tempa}%
      \ShellEscape{\@tempa}%
    }%
%    \end{macrocode}
% Sollte es eine spezielle Version eines beispiels für die Druckausgabe geben, 
% wird auch dieses kompiliert.
%    \begin{macrocode}
    \filemodCmp{examples/#1_print.pdf}{examples/#1_print.tex}{}{%
      \edef\@tempa{%
        \tud@latex@ext -output-directory examples examples/#1_print.tex%
      }%
      \ShellEscape{\@tempa}%
      \ShellEscape{\@tempa}%
    }%
  \fi%
%    \end{macrocode}
% Ist die gesuchte Datei vorhanden, wird diese auch eingebunden. Andernfalls 
% wird eine Warnung mit dem Hinweis auf die notwendige Verwendung von 
% \app{pdflatex} mit dem Schalter \file{-{}-shell-escape} erzeugt.
%    \begin{macrocode}
  \def\@tempa{examples/#1.pdf}%
  \ifdef{\tudprintflag}{%
    \IfFileExists{examples/#1_print.pdf}{\def\@tempa{examples/#1_print.pdf}}{}%
  }{}%
  \IfFileExists{\@tempa}{%
    \edef\@tempa{%
      \noexpand\includepdf[%
        pages=-,noautoscale,frame,keepaspectratio,pagecommand={},%
        height=\noexpand\textheight,width=\noexpand\textwidth,offset=5mm 0mm%
      ]{\@tempa}%
    }%
    \@tempa%
  }{%
    \tud@shellescape@wrn{\@tempa}%
  }%
}
%</class>
%    \end{macrocode}
% \end{macro}^^A \IncludeExample
%
% \iffalse
%<*package>
% \fi
%
% \begin{macro}{\tud@tut@temp}
% \changes{v2.05}{2016/05/01}{neu}^^A
% \begin{macro}{\tud@tut@readtostream}
% \changes{v2.02}{2014/12/16}{neu}^^A
% \begin{macro}{\tud@tut@append}
% \changes{v2.02}{2014/08/19}{neu}^^A
% Bei den Tutorials soll zum Schluss ein komplett kompilierbares Beispiel für
% Copy~\&~Paste gegeben werden. Um dieses zusammenzubauen, werden alle im 
% Dokument innerhalb der Varianten der \env{Trunk}-Umgebungen gegebenen
% Codefragmente zusammengesetzt. Um die Quelltexte einlesen und verarbeiten zu
% können, werden zwei Eingabe- sowie ein Ausgabestream benötigt, welche bereits 
% zuvor initialisiert wurden.
%
% Mit \cs{tud@tut@readtostream} wird der Inhalt der Datei im ersten Argument 
% in den Ausgabestream des zweiten Argumentes geschrieben. Dies wird durch 
% \cs{tud@tut@append} genutzt.
%    \begin{macrocode}
\newcommand*\tud@tut@temp{}
\edef\tud@tut@temp{\@currname-temp}
\newcommand*\tud@tut@readtostream[2]{%
  \begingroup%
    \endlinechar=\m@ne\relax%
    \openin\tud@read=#1%
%    \end{macrocode}
% Der Schalter \cs{if@tempswa} wird verwendet, um unnötige Leerzeilen in der
% Ausgabedatei zu entfernen. Wurde eine nicht leere Zeile gefunden, wird diese 
% auf jeden Fall geschrieben, indem \cs{@tempswatrue} gesetzt wird. Eine leere 
% Zeile wird~-- bis auf den Sonderfall, dass es sich um die letzte Zeile in der
% Datei handelt~-- zunächst nicht weiter beachtet. Im Zweifelsfall wird diese
% (erste) Leerzeile geschrieben. 
%    \begin{macrocode}
    \@tempswafalse%
    \loop\unless\ifeof\tud@read%
      \readline\tud@read to\tud@reserved%
      \ifx\tud@reserved\@empty%
        \ifeof\tud@read\@tempswafalse\fi%
      \else%
        \@tempswatrue%
      \fi%
%    \end{macrocode}
% Hier erfolgt das eigentliche Schreiben im selektierten Stream.
%    \begin{macrocode}
      \if@tempswa%
        \immediate\write#2{\expandonce\tud@reserved}%
      \fi%
%    \end{macrocode}
% Im Nachgang wird überprüft, ob die aktuell geschriebene Zeile leer war. Ist 
% dies der Fall, so wird \cs{@tempswafalse} gesetzt, um eine etwaig folgende 
% Leerzeile nicht in die Ausgabedatei zu schrieben.
%    \begin{macrocode}
      \ifx\tud@reserved\@empty\@tempswafalse\fi%
    \repeat%
    \closein\tud@read%
  \endgroup%
}
%    \end{macrocode}
% Der innerhalb der Umgebungen \env{Preamble} und \env{Trunk}~-- sowie deren 
% Derivaten~-- angegebene Inhalt wird beim Schließen dieser Umgebungen in die 
% Hilfsdatei \cs{tud@tut@temp.tex} zwischengespeichert. Mit \cs{tud@tut@append}
% wird der Inhalt dieser Datei ausgelesen und den Hilfsdateien
% \cs{tud@tut@temp-preamble.tex} respektive \cs{tud@tut@temp-trunk.tex}
% angehangen und damit gesichert, um die jeweiligen Quelltextauszüge aus dem
% Tutorial für die Präambel sowie den Dokumentteil trennen beziehungsweise
% sortieren zu können.
%    \begin{macrocode}
\newcommand*\tud@tut@append[1]{%
  \ifstr{#1}{preamble}{%
    \tud@tut@readtostream{\tud@tut@temp.tex}{\tud@write@a}%
  }{%
    \ifstr{#1}{trunk}{%
      \tud@tut@readtostream{\tud@tut@temp.tex}{\tud@write@b}%
    }{%
      \tud@tut@readtostream{\tud@tut@temp.tex}{#1}%
    }%
  }%
}
%    \end{macrocode}
% \end{macro}^^A \tud@tut@append
% \end{macro}^^A \tud@tut@readtostream
% \end{macro}^^A \tud@tut@temp
% \begin{macro}{\StartTutorial}
% \changes{v2.02}{2014/08/21}{neu}^^A
% \begin{macro}{\FinishTutorial}
% \changes{v2.02}{2014/08/21}{neu}^^A
% Mit \cs{StartTutorial} wird ein Hyperlink für den Querverweis aus dem 
% Handbuch definiert. Außerdem wird gleich der Titel und~-- falls im optionalen 
% Argument angegeben~-- eine Zusammenfassung gesetzt. Die Sternversion gibt 
% nach der optionalen Beschreibung einen einleitenden Satz zur Dokumnentklasse 
% und sinnvollen Paketen an. Diese sollten direkt nach \cs{StartTutorial} in 
% der \env{Preamble}-Umgebung angegeben werden.
%    \begin{macrocode}
\TUDoptions{headingsvskip=-10ex}
\newcommand\StartTutorial[1][]{%
  \immediate\openout\tud@write@a=\tud@tut@temp-preamble.tex\relax%
  \immediate\openout\tud@write@b=\tud@tut@temp-trunk.tex\relax%
  \immediate\write\tud@write@b{^^J\string\begin{document}^^J}%
  \Hy@raisedlink{\hyperdef{\jobname}{tutorials:\jobname}{}}%
  \faculty{}%
  \maketitle%
  \ifblank{#1}{}{%
    \noindent%
    \begin{abstract}%
    \noindent#1%
    \end{abstract}%
    \medskip%
  }%
  \noindent\ignorespaces%
}
%    \end{macrocode}
% Der Befehl \cs{FinishTutorial} schleißt den noch offenen Ausgabestream und
% gibt den vorgestellten Quelltext vollständig aus.
%    \begin{macrocode}
\newcommand\FinishTutorial[1][]{%
  \immediate\write\tud@write@b{^^J\string\end{document}^^J}%
  \immediate\closeout\tud@write@b%
  \immediate\closeout\tud@write@a%
  \immediate\openout\tud@write@a\jobname-example.tex\relax%
  \tud@tut@readtostream{\tud@tut@temp-preamble.tex}{\tud@write@a}%
  \tud@tut@readtostream{\tud@tut@temp-trunk.tex}{\tud@write@a}%
  \immediate\closeout\tud@write@a%
  \clearpage%
  \addsec{Copy\nobreakspace\&\nobreakspace{}Paste}%
  \thispagestyle{empty}%
  \pagestyle{empty}%
  Zum Ende des Dokumentes wird das vorgestellte Tutorial als \"ubersetzbarer %
  Quelltext ausgegeben, um dieses via Copy~\&~Paste verwenden und alle Punkte %
  nachvollziehen zu k\"onnen. Bitte beachten Sie, dass~-- abh\"angig vom %
  genutzten PDF-Betrachter~-- beim Kopieren die dargestellten Einz\"uge und %
  Abs\"atze m\"oglicherweise verloren gehen k\"onnen. Dies kann insbesondere %
  aufgrund fehlender Leerzeilen zu einem unvorteilhaften Ausgabeergebnis %
  f\"uhren. Alternativ finden Sie den folgenden \hologo{LaTeX}-Quelltext auch %
  im Pfad \Path{\PName{texmf}/source/latex/tudscr/doc/examples/} als Datei %
  \File{\jobname-example.tex}. \par #1%
  %
  \begin{quoting}[rightmargin=0pt]%
  \lstinputlisting[style=tudscr]{\jobname-example.tex}%
  \end{quoting}%
  %
%    \end{macrocode}
% Falls es möglich ist, werden nach dem Durchlauf alle nicht mehr benötigten 
% Hilfsdateien gelöscht.
%    \begin{macrocode}
  \ifnum\pdf@shellescape=\@ne\relax%
    \ifwindows%
      \def\@tempa{del}%
    \else%
      \def\@tempa{rm}%
    \fi%
    \ShellEscape{\@tempa\space\tud@tut@temp*.*}%
  \fi%
}
%    \end{macrocode}
% \end{macro}^^A \FinishTutorial
% \end{macro}^^A \StartTutorial
% \begin{macro}{\CodePreamble}
% \changes{v2.02}{2014/08/20}{neu}^^A
% \begin{macro}{\Code@Preamble}
% \changes{v2.02}{2014/10/22}{neu}^^A
% \begin{macro}{\CodeHook}
% \changes{v2.02}{2014/10/11}{neu}^^A
% \begin{macro}{\Code@Hook}
% \changes{v2.02}{2014/10/11}{neu}^^A
% Bereitstellung von Hilfsmakros für die formatierte Ausgabe von Quelltexten.
%    \begin{macrocode}
\newcommand\Code@Preamble{}
\let\Code@Preamble\relax
\newcommand\CodePreamble[1]{%
  \long\gdef\Code@Preamble{%
    \hskip.5\leftmargin\textit{Die resultierende Ausgabe:}\space#1%
  }%
}
\newcommand*\Code@Hook{}
\let\Code@Hook\relax
\newcommand*\CodeHook[1]{\gdef\Code@Hook{#1}}
%    \end{macrocode}
% \end{macro}^^A \Code@Hook
% \end{macro}^^A \CodeHook
% \end{macro}^^A \Code@Preamble
% \end{macro}^^A \CodePreamble
% \begin{macro}{\StandaloneFile}
% \changes{v2.02}{2014/08/19}{neu}^^A
% \begin{macro}{\StandaloneDate}
% \changes{v2.05}{2015/11/18}{neu}^^A
% Die Umgebungen \env{Trunk!} und \env{Hint!} erzeugen nach der Ausgabe des
% Quelltextes mit dem Makro \cs{tud@tut@pdf} eine PDF-Datei, welche mit
% \cs{IncludeStandalone} in das Tutorial eingebunden werden kann.
%    \begin{macrocode}
\newcommand*\StandaloneFile{}
\let\StandaloneFile\relax
\newcommand*\StandaloneDate{}
%    \end{macrocode}
% \end{macro}^^A \StandaloneDate
% \end{macro}^^A \StandaloneFile
% \begin{macro}{\tud@tut@pre}
% \changes{v2.02}{2014/08/19}{neu}^^A
% \begin{macro}{\tud@tut@post}
% \changes{v2.02}{2014/08/19}{neu}^^A
% Dies sind Hilfsmakros, welche zu Beginn und Ende der Quelltextumgebungen 
% \env{Preamble}, \env{Trunk} und \env{Hint} für das Sichern der Inhalte in
% eine Hilfsdatei verantwortlich sind, wofür die \env{filecontents}-Umgebung 
% genutzt wird.
%    \begin{macrocode}
\newcommand*\tud@tut@pre{%
  \csuse{filecontents*}{\tud@tut@temp.tex}%
}
\newcommand*\tud@tut@post{%
  \csuse{endfilecontents*}%
}
%    \end{macrocode}
% \end{macro}^^A \tud@tut@post
% \end{macro}^^A \tud@tut@pre
% Die nachfolgenden Umgebungen verwenden die beiden, gerade zuvor eingeführten 
% Hilfsmakros \cs{tud@tut@pre} und \cs{tud@tut@post}, welche den gegebenen 
% Inhalt in die temporäre Datei \file{\string\tud@tut@temp.tex} schreiben. Die 
% im Dokument in den Umgebungen gesetzten Quelltextauszüge werden mit dem Makro
% \cs{tud@tut@append}~-- abhängig vom gegebenen Argument~-- zusätzlich für ein 
% Copy"~\&"~Paste"~Beispiel in \file{\string\tud@tut@temp-preamble.tex} oder
% \file{\string\tud@tut@temp-trunk.tex} gepseichert. Die damit gesammelten 
% Inhalte werden am Ende mit \cs{FinishTutorial} ausgegeben.
% \begin{environment}{Preamble}
% \changes{v2.02}{2014/11/02}{neu}^^A
% \begin{environment}{Preamble*}
% \changes{v2.02}{2014/11/03}{neu}^^A
% \begin{environment}{Preamble+}
% \changes{v2.02}{2014/11/03}{neu}^^A
% Die Umgebung \env{Preamble} dient für die Ausgabe von Quelltextauszügenen, 
% welche in einem \LaTeXe-Dokument in der Präambel verwendet werden müssen oder
% sollten. Die in ihr gesetzten Inhalte werden für das Copy"~\&"~Paste-Beispiel
% vor den Inhalten aus der Umgebung \env{Trunk} ausgegeben. Die Sternversion
% führt den Inhalt zusätzlich im Dokument aus. Die Pluszechen"=Version fügt dem
% Ausgabestream ihren Inhalt hinzu, ohne das dieser ausgegeben oder ausgeführt
% wird.
%    \begin{macrocode}
\newenvironment{Preamble}{\tud@tut@pre}{%
  \tud@tut@post%
  \tud@tut@append{preamble}%
  \gdef\@tempa{%
    \tud@tut@lst%
    \global\let\Code@Preamble\relax%
  }%
  \aftergroup\@tempa%
}
\newenvironment{Preamble*}{\tud@tut@pre}{%
  \tud@tut@post%
  \tud@tut@append{preamble}%
  \gdef\@tempa{%
    \tud@tut@lst%
    \tud@tut@input%
    \global\let\Code@Preamble\relax%
  }%
  \aftergroup\@tempa%
}
\newenvironment{Preamble+}{\tud@tut@pre}{%
  \tud@tut@post%
  \tud@tut@append{preamble}%
  \gdef\@tempa{%
    \global\let\Code@Preamble\relax%
  }%
  \aftergroup\@tempa%
}
%    \end{macrocode}
% \end{environment}^^A Preamble+
% \end{environment}^^A Preamble*
% \end{environment}^^A Preamble
% \begin{environment}{Trunk}
% \changes{v2.02}{2014/08/16}{neu}^^A
% \begin{environment}{Trunk*}
% \changes{v2.02}{2014/08/19}{neu}^^A
% \begin{environment}{Trunk+}
% \changes{v2.02}{2014/11/03}{neu}^^A
% \begin{environment}{Trunk!}
% \changes{v2.02}{2014/10/07}{neu}^^A
% Die Umgebung \env{Trunk} dient zur Ausgabe von exemplarischen Quelltexten, 
% welche in einem \LaTeXe-Dokument innerhalb der \env{document}-Umgebung
% verwendet werden müssen oder sollten. Für das Copy"~\&"~Paste-Beispiel werden 
% die gegebenen Inhalte nach den Inhalten aus der Umgebung \env{Preamble} 
% ausgegeben. Die Sternversion der Umgebung führt ihren Inhalt zusätzlich im
% Dokument aus. Die Pluszeichen"=Version fügt dem Ausgabestream ihren Inhalt
% hinzu, ohne diesen auszugeben oder auszuführen. Die Ausrufezeichen"=Version
% fügt den Quelltext im Dokument ein und führt diesen zusätzlich in einer
% separaten Datei mit einem minimalen Dokumentkörper aus, um daraus eine
% PDF-Datei zu erzeugen, welche im Nachhinein mit \cs{IncludeStandalone} als
% Grafik eingebunden werden kann. Dies ist für Quelltextabschnitte gedacht,
% deren Ausgabe zu groß ist, um diese direkt anzuzeigen und dennoch dargestellt
% werden sollen.
%    \begin{macrocode}
\newenvironment{Trunk}{\tud@tut@pre}{%
  \tud@tut@post%
  \tud@tut@append{trunk}%
  \gdef\@tempa{%
    \tud@tut@lst%
    \global\let\Code@Preamble\relax%
  }%
  \aftergroup\@tempa%
}
\newenvironment{Trunk*}{\tud@tut@pre}{%
  \tud@tut@post%
  \tud@tut@append{trunk}%
  \gdef\@tempa{%
    \tud@tut@lst%
    \tud@tut@input%
    \global\let\Code@Preamble\relax%
  }%
  \aftergroup\@tempa%
}
\newenvironment{Trunk+}{\tud@tut@pre}{%
  \tud@tut@post%
  \tud@tut@append{trunk}%
  \gdef\@tempa{%
    \global\let\Code@Preamble\relax%
  }%
  \aftergroup\@tempa%
}
\newenvironment{Trunk!}[1]{%
  \gdef\StandaloneFile{#1}%
  \tud@tut@pre%
}{%
  \tud@tut@post%
  \tud@tut@append{trunk}%
  \gdef\@tempa{%
    \tud@tut@lst%
    \tud@tut@pdf%
    \global\let\Code@Preamble\relax%
  }%
  \aftergroup\@tempa%
}
%    \end{macrocode}
% \end{environment}^^A Trunk!
% \end{environment}^^A Trunk+
% \end{environment}^^A Trunk*
% \end{environment}^^A Trunk
% \begin{environment}{Hint}
% \changes{v2.02}{2014/09/16}{neu}^^A
% \begin{environment}{Hint*}
% \changes{v2.02}{2014/10/13}{neu}^^A
% \begin{environment}{Hint?}
% \changes{v2.02}{2014/12/09}{neu}^^A
% \begin{environment}{Hint!}
% \changes{v2.02}{2014/11/13}{neu}^^A
% Um Quelltextausschnitte zur weiterführenden Information anzugeben, wird die
% Umgebung \env{Hint} definiert. Der Inhalt wird ausgegben und~-- falls die
% Sternversion genutzt wurde~-- auch ausgeführt, allerdings nicht in das 
% Copy"~\&"~Paste"~Beispiel übernommen. Ansonsten entsprechen alle Umgebungen 
% in ihren Eigenschaften den äquivalenten \env{Trunk}-Umgebungen. Die Version 
% mit Fragezeichen ist für Quelltextauszüge gedacht, die ungeprüft \emph{vor}
% dessen Ausgabe ausgeführt werden sollen. 
%    \begin{macrocode}
\newenvironment{Hint}{\tud@tut@pre}{%
  \tud@tut@post%
  \gdef\@tempa{%
    \tud@tut@lst%
    \global\let\Code@Preamble\relax%
  }%
  \aftergroup\@tempa%
}
\newenvironment{Hint*}{\tud@tut@pre}{%
  \tud@tut@post%
  \gdef\@tempa{%
    \tud@tut@lst%
    \tud@tut@input%
    \global\let\Code@Preamble\relax%
  }%
  \aftergroup\@tempa%
}
\newenvironment{Hint?}{\tud@tut@pre}{%
  \tud@tut@post%
  \gdef\@tempa{%
    \InputCode%
    \tud@tut@lst%
    \global\let\Code@Preamble\relax%
  }%
  \aftergroup\@tempa%
}
\newenvironment{Hint!}[1]{%
  \gdef\StandaloneFile{#1}%
  \tud@tut@pre%
}{%
  \tud@tut@post%
  \gdef\@tempa{%
    \tud@tut@lst%
    \tud@tut@pdf%
    \global\let\Code@Preamble\relax%
  }%
  \aftergroup\@tempa%
}
%    \end{macrocode}
% \end{environment}^^A Hint!
% \end{environment}^^A Hint?
% \end{environment}^^A Hint*
% \end{environment}^^A Hint
% \begin{macro}{\tud@tut@lst}
% \changes{v2.02}{2014/10/07}{neu}^^A
% Damit wird der Quelltext eingezogen ausgegeben.
%    \begin{macrocode}
\newcommand*\tud@tut@lst{%
  \begin{quoting}[rightmargin=0pt]%
  \lstinputlisting[style=tudscr]{\tud@tut@temp.tex}%
  \end{quoting}%
}
%    \end{macrocode}
% \end{macro}^^A \tud@tut@lst
% \begin{macro}{\tud@tut@input}
% \changes{v2.02}{2014/10/13}{neu}^^A
% Für die Ausgabe des Quelltextergebnisses erfolgt ein linker Einzug. Außerdem 
% werden unter anderem abhängig von der Verwendung eines erklärenden Textes die 
% eingefügten vertikalen Abstände angepasst.
%    \begin{macrocode}
\newcommand*\tud@tut@input{%
%    \end{macrocode}
% Nach dem Ausführen des ersten Argumentes von \cs{@tempa} wird geprüft, ob
% dieses überhaupt eine sichtbare Ausgabe erzeugt hat. Nur für diesen Fall wird
% nach dieser vertikaler Leerraum eingefügt. Der Anfang entspricht prinzipiell
% dem Standardbefehl \cs{settoheight}, jedoch wird im Inneren zusätzlich eine
% \cs{vbox} verwendet, um damit auch mathematische Umgebungen testen zu können. 
% Das Ganze findet innerhalb einer Gruppierung statt, um keine Änderungen zu
% erzeugen.
%    \begin{macrocode}
  \def\@tempa##1##2{%
    ##1%
    \begingroup%
      ##2%
      \setbox\@tempboxa\hbox{\vbox{##1}}%
      \@tempdima=\ht\@tempboxa%
      \setbox\@tempboxa\box\voidb@x%
      \ifdim\@tempdima>\z@\relax%
        \endgraf%
        \vspace{\medskipamount}%
        \noindent\ignorespaces%
      \fi%
    \endgroup%
  }%
%    \end{macrocode}
% Da aufgrund der Verwendung der \env{filecontents}-Umgebung ein optionales
% Argument für alle Abwandlungen der Quelltextumgebungen nicht möglich ist,
% wird \cs{CodePreamble} bereitgestellt, womit der Ausgabe ein erläuternder
% Text vorangestellt werden kann.
%    \begin{macrocode}
  \@tempa{\Code@Preamble}{}%
  \setlength\@tempdimc{\leftskip}%
  \setlength\leftskip{\leftmargin}%
%    \end{macrocode}
% Durch die zweimalige Verwendung von \file{\string\tud@tut@temp.tex}~-- einmal 
% direkt und einmal innerhalb einer Box~-- können darin verwendete Befehle wie
% \cs{newcommand} zu Fehler führen. Mit dem Makro \cs{CodeHook} lassen sich 
% diese Konflikte durch eine lokale Redefinition ebensolcher Befehle auflösen
% (bspw. \cs{let}\cs{newcommand}\cs{renewcommand}).
%    \begin{macrocode}
  \@tempa{\input{\tud@tut@temp.tex}}{\Code@Hook}%
  \global\let\Code@Hook\relax%
  \setlength\leftskip{\@tempdimc}%
  \noindent\ignorespaces%
}
%    \end{macrocode}
% \end{macro}^^A \tud@tut@input
% \begin{macro}{\InputCode}
% \changes{v2.02}{2014/10/21}{neu}^^A
% Mit \cs{InputCode} kann der Inhalt der letzen \env{Trunk}-Umgebung 
% direkt ausgeführt werden, ohne diesen zuvor mit \cs{tud@tut@input} zu prüfen 
% und zu formatieren.
%    \begin{macrocode}
\newcommand*\InputCode{\input{\tud@tut@temp.tex}}
%    \end{macrocode}
% \end{macro}^^A \InputCode
% \begin{macro}{\tud@tut@pdf}
% \changes{v2.02}{2014/10/22}{neu}^^A
% Die Umgebungen \env{Trunk!} und \env{Hint!} verwenden nach der Ausgabe des
% Quelltextes diesen abermals, um daraus eine PDF-Datei zu erzeugen. Dafür wird
% eine temporäre \LaTeX-Datei mithilfe einer \env{filecontents}-Umgebung 
% erzeugt. Diese bindet die mit \env{Trunk!} erzeugte Datei ein und wird
% anschließend via \cs{ShellEscape} kompiliert. Dabei sollte man sich nicht von 
% der doppelten Verwendung von \cs{jobname} verwirren lassen. Der Dateiname der
% temporär erzeugten \LaTeX-Datei ist abhängig vom gesetzten Tutorial. Beim
% Aufruf dieser wird über die Optionen von \app{pdflatex} der verwendete
% \cs{jobname} angepasst (siehe \cs{tud@tut@pdf}).
%    \begin{macrocode}
\IfFileExists{\tud@tut@temp-standalone.tex}{\@tempswafalse}{\@tempswatrue}
\if@tempswa%
\begin{filecontents*}{\tud@tut@temp-standalone.tex}
\documentclass[english,ngerman]{tudscrreprt}
\ifpdftex{
  \usepackage[T1]{fontenc}
  \usepackage[ngerman=ngerman-x-latest]{hyphsubst}
}{
  \usepackage{fontspec}
}
\usepackage{babel}
\usepackage{tudscrsupervisor}
\usepackage{isodate}
\usepackage{enumitem}
\setlist{noitemsep}
\begin{document}
\ifdefvoid{\StandaloneDate}{}{\date{\StandaloneDate}}%
\input{\jobname.tex}%
\end{document}
\end{filecontents*}
\fi
%    \end{macrocode}
% Der Quelltext aus den Umgebungen \env{Trunk!} und \env{Hint!} wird~-- wie bei
% allen anderen \env{Trunk}-Derivaten~-- in \file{\string\tud@tut@temp.tex}
% gesichert. Beim Aufruf von \app{pdflatex} wird durch die Nutzung der Option
% \file{-jobname=\string\tud@tut@temp} die Definition von \cs{jobname} in der
% aufgerufenen Datei \file{\string\tud@tut@temp-standalone.tex} überschrieben
% und der Inhalt von \file{\string\tud@tut@temp.tex} wie gewünscht eingebunden
% und kompiliert. Damit die so erstellte PDF-Datei mit \cs{IncludeStandalone}
% im weiteren Verlauf in das Tutorial eingebunden werden kann, wird diese 
% außerdem noch in \file{\string\jobname-standalone-\string\StandaloneFile.pdf} 
% umbenannt. Das Makro \cs{StandaloneDate} wird auf das Datum der Master-Datei
% gesetzt, damit das voreingestellte Datum der erzeugten Datei angepasst werden
% kann.
%    \begin{macrocode}
\newcommand*\tud@tut@pdf{%
  \ifnum\pdf@shellescape=\@ne\relax%
    \filemodCmp{\jobname-standalone-\StandaloneFile.pdf}{\jobname.tex}{}{%
      \begingroup%
        \let\printdate\@firstofone%
        \edef\@tempa{\expandonce\@date}%
        \def\@tempb{\today}%
        \ifx\@tempa\@tempb%
          \let\@tempb\@empty%
        \else%
          \edef\@tempb{\etex@unexpanded{\def\StandaloneDate}{\@date}}%
        \fi%
        \edef\@tempa{%
          \tud@latex@ext -jobname=\tud@tut@temp\space%
          "\noexpand\unexpanded{\expandonce\@tempb\noexpand\input}%
            {\tud@tut@temp-standalone.tex}"%
        }%
        \ShellEscape{\@tempa}%
        \ShellEscape{\@tempa}%
        \ShellEscape{\@tempa}%
        \ifwindows%
          \def\@tempa{rename}%
        \else%
          \def\@tempa{mv}%
        \fi%
        \ShellEscape{%
          \@tempa\space\tud@tut@temp.pdf\space%
          \jobname-standalone-\StandaloneFile.pdf%
        }%
      \endgroup%
    }%
  \fi%
  \global\let\StandaloneFile\relax%
}
%    \end{macrocode}
% \end{macro}^^A \tud@tut@pdf
% \begin{macro}{\IncludeStandalone}
% \changes{v2.02}{2014/09/10}{neu}^^A
% Der Befehl \cs{IncludeStandalone} bindet die durch \env{Trunk!} generierten
% PDF-Dateien ein. Das angehängte optionale Argument dient zur Angabe der
% einzubindenden Seiten der PDF-Datei, wobei diese automatisch auf die passende
% Breite skaliert werden.
%    \begin{macrocode}
\NewDocumentCommand\IncludeStandalone{o m !O{1}}{%
  \IfFileExists{\jobname-standalone-#2.pdf}{%
    \@tempcnta\z@%
    \@for\@tempa:=#3\do{\advance\@tempcnta\@ne}%
    \ifnum\@tempcnta>\z@\relax%
      \setlength\@tempdima{\textwidth}%
      \divide\@tempdima\@tempcnta%
      \advance\@tempcnta\m@ne%
      \multiply\@tempcnta 2%
      \addtolength\@tempdima{-\@tempcnta\tabcolsep}%
      \@for\@tempa:=#3\do{%
        \advance\@tempcnta\m@ne%
        \edef\@tempb{%
          keepaspectratio,page=\@tempa,width=\the\@tempdima,%
          \IfValueT{#1}{#1}%
        }%
        \fbox{%
          \expandafter\includegraphics\expandafter[\@tempb]{%
            \jobname-standalone-#2.pdf%
          }%
        }%
        \ifnum\@tempcnta>\z@\relax\hfill\fi%
      }%
    \fi%
  }{%
    \tud@shellescape@wrn{\jobname-standalone-#2.pdf}%
  }%
}
%    \end{macrocode}
% \end{macro}^^A \IncludeStandalone
%
% \iffalse
%</package>
% \fi
%
% \subsection{Automatische Erstellung von \pkg{pstricks}-Grafiken}
%
% Falls innerhalb eines Tutorials das Paket \pkg{pstricks} verwendet wird, muss
% dafür Sorge getragen werden, dass die automatische Kompilierung reibungslos
% mit \pkg{auto-pst-pdf} funktioniert.
%    \begin{macrocode}
%<*package>
\AfterPackage*{pstricks}{%
  \ifnum\pdf@shellescape=\@ne\relax%
    \filemodCmp{\jobname-pics.pdf}{\jobname.tex}{%
      \PassOptionsToPackage{off}{auto-pst-pdf}%
    }{}%
  \else%
    \PassOptionsToPackage{off}{auto-pst-pdf}%
  \fi%
  \RequirePackage{auto-pst-pdf}[2009/04/26]%
%    \end{macrocode}
% Nach dem Paket \pkg{auto-pst-pdf} seine Arbeit verrichtet hat, werden alle 
% etwaigen temporär erzeugten Dateien radikal gelöscht.
%    \begin{macrocode}
  \ifnum\pdf@shellescape=\@ne\relax%
    \edef\@tempa{\app@exe{\app@rm "*\app@suffix*"}}%
    \@tempa%
  \fi%
}
%</package>
%    \end{macrocode}
%
% \subsection{ToDo-Liste}
%
% Für Klasse und Paket besteht die Möglichkeit, Änderungsnotizen zu nutzen, 
% wofür das Paket \pkg{todonotes} geladen wird.
%    \begin{macrocode}
\PassOptionsToPackage{obeyFinal}{todonotes}
%    \end{macrocode}
% Für den Druck wird die farbige Ausgabe der ToDo-Notizen deaktiviert.
%    \begin{macrocode}
\ifdef{\tudprintflag}{%
  \PassOptionsToPackage{color=white}{todonotes}%
}{%
  \PassOptionsToPackage{colorinlistoftodos,color=HKS92!10}{todonotes}%
}
\RequirePackage{todonotes}[2012/07/25]
\AtEndPreamble{%
  \ifdim\marginparwidth<2cm\relax%
    \setlength\marginparwidth{2cm}%
  \fi%
}
%    \end{macrocode}
%
% \iffalse
%</body>
%<*option>
% \fi
%
% \begin{option}{ToDo}
% \changes{v2.02}{2014/07/10}{neu}^^A
% \changes{v2.05}{2015/10/27}{Nutzung von Positiv- und Negativ-Liste}^^A
% \begin{macro}{\if@tud@todo}
% \changes{v2.02}{2014/07/10}{neu}^^A
% \begin{macro}{\tud@todo@type@pos}
% \changes{v2.05}{2015/10/27}{neu}^^A
% \begin{macro}{\tud@todo@type@neg}
% \changes{v2.05}{2015/10/27}{neu}^^A
% Diese Option ist verantwortlich für den Schalter \cs{if@tud@todo}. Über 
% diesen wird gesteuert, ob ToDo-Notizen ausgegeben werden sollen.
%    \begin{macrocode}
\newif\if@tud@todo
\newcommand*\tud@todo@type@pos{}
\let\tud@todo@type@pos\relax
\newcommand*\tud@todo@type@neg{}
\let\tud@todo@type@neg\relax
\TUD@key{ToDo}[true]{%
  \TUD@set@ifkey{ToDo}{@tud@todo}{#1}%
%    \end{macrocode}
% Wird der Option ein boolescher Wert übergeben, sind sowohl die Positiv- als 
% auch die Negativ-Liste hinfällig.
%    \begin{macrocode}
  \ifx\FamilyKeyState\FamilyKeyStateProcessed%
    \let\tud@todo@type@pos\relax%
    \let\tud@todo@type@neg\relax%
  \else%
%    \end{macrocode}
% Wird die Option nicht mit einem booleschen Wert verwendet, so kann über diese 
% gezielt der Typ einer (nicht) auszugebenden ToDo-Notiz angegeben werden. So 
% können bestimmte Typen von Notizen entweder ausschließlich ausgegeben oder 
% aber unterdrückt werden, wobei für letztere Variante das Suffix \val{not} 
% vor dem eigentlichen Typen verwendet werden muss.
%    \begin{macrocode}
    \def\@tempa not##1##2##3##4\@nil{%
      \ifstr{##1##2##3}{not}{%
        \IfArgIsEmpty{##4}{}{%
          \listeadd\tud@todo@type@neg{##4}%
          \@tud@todotrue%
          \FamilyKeyStateProcessed%
        }%
      }{%
        \IfArgIsEmpty{##1}{}{%
          \listeadd\tud@todo@type@pos{##1##2##3##4}%
          \@tud@todotrue%
          \FamilyKeyStateProcessed%
        }%
      }%
    }%
    \edef\@tempb{not\trim@spaces{#1}}%
    \expandafter\@tempa\@tempb\@empty\@empty\@empty\@nil%
  \fi%
}
\TUDExecuteOptions{ToDo=true}
%    \end{macrocode}
% \end{macro}^^A \tud@todo@type@neg
% \end{macro}^^A \tud@todo@type@pos
% \end{macro}^^A \if@tud@todo
% \end{option}^^A ToDo
%
% \iffalse
%</option>
%<*body>
% \fi
%
% \begin{macro}{\ToDo}
% \changes{v2.02}{2014/07/10}{Verwendung von \pkg{todonotes}}^^A
% \changes{v2.03}{2015/01/25}{Ausgabe auf bestimmten Typ beschränkbar}^^A
% \changes{v2.05}{2015/11/02}{Farbkodierung durch optionale Versionsnummer}^^A
% \changes{v2.06}{2018/08/02}{farbige Ausgabe abhängig von Versionsnummer}^^A
% \begin{macro}{\tud@todo@type@use}
% \changes{v2.06}{2018/08/02}{neu}^^A
% \begin{macro}{\ListOfToDo}
% \changes{v2.02}{2014/07/10}{neu}^^A
% Mit dem Befehl \cs{ToDo}\oarg{Option}\marg{ToDo-Notiz}\oarg{Versionsnummer}
% kann bei aktivierter Option \opt{ToDo} eine Notiz mit einer offenen Aufgabe
% erstellt werden. Das erste optionale Argument kann die Kategorisierungstypen
% \val{doc}, \val{imp} und \val{rls} beinhalten, welche für eine spezifische
% farbliche Kodierung sorgen, um die Aufgaben zu kategorisieren.
%    \begin{macrocode}
\AfterPackage*{todonotes}{%
  \newcommand*\tud@todo@type@use{}%
%    \end{macrocode}
% Nun folgt die Definition des eigentlichen Befehls. Dabei wird zu Beginn
% \cs{@currentHref} gesichert und am Ende wiederhergestellt.
%    \begin{macrocode}
  \NewDocumentCommand\ToDo{s o m !o}{%
    \ifbool{@tud@todo}{%
      \global\let\tud@currentHref\@currentHref%
      \begingroup%
%    \end{macrocode}
% In \cs{tud@todo@type@use} werden alle im optionalen Argument angegeben Werte 
% gesischer, welche entweder über die Liste \cs{tud@todo@type@pos} explizit 
% angefordert oder nicht über die Liste \cs{tud@todo@type@neg} ausgeschlossen 
% wurden.
%    \begin{macrocode}
        \let\tud@todo@type@use\@empty%
%    \end{macrocode}
% Das optionale Argument enthält die gewüscnhten Kategorisierungstypen.
%    \begin{macrocode}
        \IfValueTF{#2}{%
          \IfArgIsEmpty{#2}{}{%
%    \end{macrocode}
% Die Ausgabe erfolgt entweder für alle ToDo-Notizen oder nur für die per 
% Option festgelegten Typen. Dabei wird in einer Schleife geprüft, ob der in 
% der aktuellen Notiz angegebene Kategorisierungstyp entweder in der Positv- 
% oder aber in der Negativ-Liste enthalten ist.
%    \begin{macrocode}
            \let\tud@todo@type@use\relax%
            \ifx\tud@todo@type@pos\relax%
              \def\tud@res@a##1{%
                \ifinlist{##1}{\tud@todo@type@use}{}{%
                  \ifinlist{##1}{\tud@todo@type@neg}{}{%
                    \listadd\tud@todo@type@use{##1}%
                  }%
                }%
              }%
            \else%
              \def\tud@res@a##1{%
                \ifinlist{##1}{\tud@todo@type@use}{}{%
                  \ifinlist{##1}{\tud@todo@type@pos}{%
                    \listadd\tud@todo@type@use{##1}%
                  }{}%
                }%
              }%
            \fi%
            \forcsvlist\tud@res@a{#2}%
          }%
        }{}%
%    \end{macrocode}
% Nur wenn mindestens ein aktiver Kategorisierungstyp gefunden wurde oder aber
% gar keine Angabe gemacht wurde, erfolgt die Ausgabe.
%    \begin{macrocode}
        \ifx\tud@todo@type@use\relax\else%
%    \end{macrocode}
% In den temporären Makros werden die Einstellungen für die farbige Ausprägung 
% von Randnotiz (\cs{tud@res@a}) und der Textbox im Fließtex(\cs{tud@res@b}) 
% sowie das Label der Randnotiz (\cs{tud@res@c}) gespeichert.
%    \begin{macrocode}
          \let\tud@res@a\@empty%
          \let\tud@res@b\@empty%
          \let\tud@res@c\@empty%
%    \end{macrocode}
% Der Kategorisierungstyp für Releases \val{rls} ist quasi vorkonfiguriert. 
% Sind weitere Kategorisierungstypen angegeben, werden diese ignoriert. Der Typ 
% für Implemtierung \val{imp} wird vorangig zur Dokumentation \val{doc} 
% behandelt.
%    \begin{macrocode}
          \ifinlist{rls}{\tud@todo@type@use}{%
            \def\tud@res@a{color=HKS44!30,}%
            \def\tud@res@b{color=HKS44!30,}%
            \edef\tud@res@c{v\TUDScriptVersionNumber}%
          }{%
            \ifinlist{imp}{\tud@todo@type@use}{%
              \def\tud@res@a{color=HKS57!50,}%
            }{%
              \ifinlist{doc}{\tud@todo@type@use}{%
                \def\tud@res@a{color=HKS41!30,}%
              }{}%
            }%
%    \end{macrocode}
% Das angestellte optionale Argument erzeugt zusätzlich eine Notiz am Rand, in
% welcher vorzugsweise die angedachte Version eingetragen werden kann. Sollte 
% dieser angegeben sein, wird auf eine gültige Versionsnummer geprüft. Dabei 
% wird ein möglicherweise vorangestelltes |v| entfernt.
%    \begin{macrocode}
            \def\tud@res@c{ToDo}%
            \IfValueT{#4}{%
              \begingroup%
%    \end{macrocode}
% Die angegebene Versionsnummer wird geprüft, ob bereits zu einer vergangenen 
% Version gehört. In diesem Fall wird die ToDo-Notiz entsprechend als dringlich 
% gekennzeichnet. Entspricht die angegebene Versionsnummer einer späteren 
% Version, so wird ToDo-Notiz ebenfalls für diese gekennzeichnet.
%    \begin{macrocode}
                \tud@v@get\tud@res@a{#4}%
                \def\tud@res@c##1.##2##3##4\@nil{##1##2##3}%
                \edef\tud@res@b{\expandafter\tud@res@c\tud@res@a.000\@nil}%
                \ifnumber{\tud@res@b}{%
                  \edef\tud@res@c{%
                    \expandafter\tud@res@c\TUDScriptVersionNumber.000\@nil%
                  }%
                  \ifnum\tud@res@b>\tud@res@c\relax%
                    \def\tud@res@b{color=HKS92!30,}%
                  \else%
                    \ifnum\tud@res@b<\tud@res@c\relax%
                      \def\tud@res@b{color=HKS07!50,}%
                    \else%
                      \def\tud@res@b{color=HKS41!30,}%
                    \fi%
                  \fi%
                  \def\tud@res@c{v\tud@res@a}%
                }{%
                  \let\tud@res@b\@empty%
                  \def\tud@res@c{#4}%
                }%
                \edef\tud@res@c{%
                  \endgroup%
                  \def\noexpand\tud@res@b{\tud@res@b}%
                  \def\noexpand\tud@res@c{\tud@res@c}%
                }%
              \tud@res@c%
            }%
          }%
%    \end{macrocode}
% Vor der Ausgabe wird ein kleiner vertikaler Abstand eingefügt.
%    \begin{macrocode}
          \ifhmode\par\fi%
          \let\par\relax%
          \vskip\medskipamount%
          \noindent%
%    \end{macrocode}
% Dann erfolgt die Ausgabe der eigentlichen ToDo-Notiz im Textbereich. Die 
% Sternversion von \cs{ToDo} unterdrückt dabei die Aufnahme in die ToDo-Liste.
%    \begin{macrocode}
          \toks@\expandafter{\tud@res@b inline}%
          \IfBooleanT{#1}{\addto@hook\toks@{,nolist}}%
          \sbox\z@{%
            \expandafter\todo\expandafter[\the\toks@]{\trim@spaces{#3}\strut}%
          }\usebox\z@%
%    \end{macrocode}
% Es folgt die dazugehörige Randnotiz. Damit diese nicht verrutscht, wird
% \cs{marginnote} anstelle von \cs{marginpar} verwendet. Wurde das angestellte
% optionale Argument verwendet, wird dieses für die Randnotiz genutzt.
% Andernfalls erscheint im Rand die Anmerkung \emph{ToDo}.
%    \begin{macrocode}
          \renewcommand*\marginpar[2][]{%
            \marginnote[##1]{##2}[\dimexpr-\ht\z@+1.6ex+1.75pt\relax]%
          }%
          \toks@\expandafter{\tud@res@a noline,nolist}%
          \settowidth\marginparwidth{\tud@res@c}%
          \addtolength\marginparwidth{\dimexpr1.6ex+1pt\relax}%
          \expandafter\todo\expandafter[\the\toks@]{\tud@res@c\strut}%
          \aftergroup\par\aftergroup\noindent%
%    \end{macrocode}
% Wurden im ersten optionalen Argument unbekannte Werte gefunden, werden diese 
% mit einer Warnung gemeldet.
%    \begin{macrocode}
          \listremove{\tud@todo@type@use}{rls}%
          \listremove{\tud@todo@type@use}{imp}%
          \listremove{\tud@todo@type@use}{doc}%
          \ifx\tud@todo@type@use\@empty\else%
            \begingroup%
              \let\tud@res@c\@empty%
              \renewcommand*{\do}[1]{%
                \appto\tud@res@c{,##1}%
              }%
              \dolistloop{\tud@todo@type@use}%
              \edef\tud@res@c{%
                \endgroup%
                \edef\noexpand\tud@res@c{\expandafter\@gobble\tud@res@c}%
              }%
            \tud@res@c%
            \ClassWarning{tudscrmanual}{%
              Unknown key(s) `\string\ToDo[\tud@res@c]'%
            }%
          \fi%
        \fi%
      \endgroup%
      \global\let\@currentHref\tud@currentHref%
    }{}%
    \ignorespaces%
  }%
%    \end{macrocode}
% Zum Schluss wird der Befehl \cs{ListOfToDo} definiert, der die Liste der noch 
% zu erledigenden Punkte ausgibt.
%    \begin{macrocode}
  \newcommand*\ListOfToDo{\if@tud@todo\clearpage\listoftodos\fi}%
}
%    \end{macrocode}
% Sollte das Paket \pkg{todonotes} nicht geladen werden, erfolgt ein Definition 
% der beiden notwendigen Befehle als Dummy.
%    \begin{macrocode}
\TUD@UnwindPackage{todonotes}{%
  \NewDocumentCommand\ToDo{s o m !o}{}%
  \newcommand*\ListOfToDo{}%
}
%    \end{macrocode}
% \end{macro}^^A \ListOfToDo
% \end{macro}^^A \tud@todo@type@use
% \end{macro}^^A \ToDo
%
% \iffalse
%</body>
%</!doc>
% \fi
%
% \subsection{Verschiedenes für die Dokumentationsklassen und -pakete}
%
% Sowohl für die Klasse \cls{tudscrmanual} als auch \cls{tudscrdoc} werden ein 
% paar Befehle zur komfortablen Verwendung im Fließtext definiert. Außerdem 
% erfolgen mittels einiger Pakete verschiedene Formateinstellungen.
%
% \subsubsection{Ergänzend geladene Pakete}
%
% Sprachabhängiges Setzen von Anführungszeichen. Das Laden des Paketes darf 
% erst nach \pkg{inputenc} oder \pkg{inputenx} erfolgen.
%    \begin{macrocode}
%<*body|class&doc>
\RequirePackage{csquotes}[2011/10/22]
%</body|class&doc>
%    \end{macrocode}
%
% \iffalse
%<*body>
% \fi
%
% Für den Fließtext werden Pfeile u.\,ä. durch \pkg{textcomp} bereitgestellt.
%    \begin{macrocode}
\RequirePackage{textcomp}[2005/09/27]
%    \end{macrocode}
% Vergrößerung des Durchschusses.
%    \begin{macrocode}
\RequirePackage{setspace}[2011/12/19]
\setstretch{1.1}
%    \end{macrocode}
% Verbesserte Zitate.
%    \begin{macrocode}
\PassOptionsToPackage{vskip=\smallskipamount}{quoting}
\RequirePackage{quoting}[2014/01/28]
%    \end{macrocode}
% Automatisierte Datumsformatierung.
%    \begin{macrocode}
\RequirePackage{isodate}[2010/01/03]
%    \end{macrocode}
% Verschiedenste Symbole aus dem \LaTeX-Universum.
%    \begin{macrocode}
\RequirePackage{hologo}[2012/04/26]
%    \end{macrocode}
%
% \iffalse
%<*class>
% \fi
%
% Mit dem Paket \pkg{ragged2e} wird~-- falls benötigt~-- die Silbentrennung im 
% Flattersatz aktiviert. 
%    \begin{macrocode}
\RequirePackage{ragged2e}[2009/05/21]
%    \end{macrocode}
% Die Fußnoten werden nicht mit jedem Kapitel zurückgesetzt\dots
%    \begin{macrocode}
\@removefromreset{footnote}{chapter}
%    \end{macrocode}
% \dots und im Flattersatz ausgegeben.
%    \begin{macrocode}
\renewcommand*\raggedfootnote{\tud@RaggedRight}%
%    \end{macrocode}
% Das Paket wird für die Erstellung von Tabellen verwendet.
%    \begin{macrocode}
\RequirePackage{booktabs}[2005/04/14]
%    \end{macrocode}
% Die Formatierung von Gleitobjekten.
%    \begin{macrocode}
\RequirePackage{caption}[2015/09/17]
\RequirePackage{floatrow}[2008/08/02]
\DeclareCaptionSubType[alph]{figure}
\DeclareCaptionSubType[alph]{table}
\captionsetup{font=sf,labelfont=bf,labelsep=space}
\captionsetup{singlelinecheck=off,format=hang,justification=raggedright}
\captionsetup[subfloat]{labelformat=brace,list=off}
\KOMAoption{captions}{tableheading,figuresignature}
\floatsetup[table]{style=plaintop}
%    \end{macrocode}
% Erstellen von Grafiken.
%    \begin{macrocode}
\RequirePackage{tikz}[2013/12/13]
%    \end{macrocode}
% Einbinden von ganzseitigen PDF"~Dokumenten als Beispiel im Handbuch.
%    \begin{macrocode}
\RequirePackage{pdfpages}[2013/08/25]
%    \end{macrocode}
% Typographisch saubere Einheiten.
%    \begin{macrocode}
\RequirePackage{units}[1998/08/04]
%    \end{macrocode}
% Das Paket \pkg{ellipsis} sorgt für korrekte Auslassungpunkte.
%    \begin{macrocode}
\AtEndPreamble{%
  \RequirePackage{ellipsis}[2004/9/28]%
  \let\ellipsispunctuation\relax%
}
%    \end{macrocode}
%
% \iffalse
%</class>
%</body>
%<*body|class&doc>
% \fi
%
% \subsubsection{Gezieltes Ersetzen von Inhalten in Strings}
%
% \begin{macro}{\tud@replace}
% \changes{v2.05}{2015/11/01}{neu}^^A
% Mit diesem Befehl kann in einem gegebenen Makro ein bestimmtes Zeichen durch 
% ein anderes ersetzt werden. Dies wird verwendet, um für Indexbefehle oder 
% Labels die möglicherweise enthaltenen |@|"~Zeichen zu ersetzen.
%    \begin{macrocode}
\newcommand*\tud@replace[3]{%
  \begingroup%
%    \end{macrocode}
% Zunächst wird \cs{@tempa} als ein durch \cs{@nil} abgegrenztes Makro mit 
% zwei Argumenten definiert, wobei diese durch das im zweiten Argument gegebene 
% \meta{Zeichen} voneinander abgegrenzt werden. Mit diese beiden Argumente wird
% das Makro \cs{@tempb} aufgerufen. 
%    \begin{macrocode}
    \toks@{\def\@tempa##1}%
    \eaddto@hook\toks@{\detokenize{#2}}%
    \addto@hook\toks@{##2\@nil{\@tempb{##1}{##2}}}%
%    \end{macrocode}
% Der Inhalt von \cs{toks@} ist
% \cs{def}\cs{@tempa\#1\meta{Zeichen}\#2}\cs{@nil\{\cs{@tempb\{\#1\}\{\#2\}}\}}
% und definiert \cs{@tempa}.
%    \begin{macrocode}
    \the\toks@%
%    \end{macrocode}
% Nun wird \cs{@tempb} definiert. Dieses Makro fügt dem Zielmakro das erste
% Argumente gefolgt von \meta{Ersatzzeichen} hinzu. Mit dem zweiten Argument
% wird anschließend \cs{@tempa\#\#2}\cs{@nil} so lange rekursiv aufgerufen, bis
% es wirklich leer ist, wodurch das zweite Argument \meta{Zeichen} sukzessive 
% durch das dritte Argument \meta{Ersatzzeichen} ersetzt wird.
%    \begin{macrocode}
    \def\@tempb##1##2{%
      \IfArgIsEmpty{##2}{%
        \appto#1{##1}%
        \let\@tempc\relax%
      }{%
        \appto#1{##1#3}%
        \def\@tempc{\@tempa##2\@nil}%
      }%
      \@tempc%
    }%
%    \end{macrocode}
% Nachdem die beiden notwendigen Makros definiert wurden, erfolgt jetzt die 
% eigentliche Ersetzung. Dafür wird der Inhalt des Zielmakros als Argument für 
% \cs{@tempa} verwendet. Diesem wird das zu ersetzende Zeichen gefolgt von
% \cs{@nil} hinzugefügt, um das Argument abschließend zu begrenzen.
%    \begin{macrocode}
    \toks@{\@tempa}%
    \edef\@tempc{\expandafter\detokenize\expandafter{#1}}%
    \eaddto@hook\toks@{\@tempc}%
    \eaddto@hook\toks@{\detokenize{#2}\@nil}%
%    \end{macrocode}
% Der Inhalt von \cs{toks@} hat die Form \cs{@tempa\#1\meta{Zeichen}}\cs{@nil},
% wobei der Inhalt von \meta{\#1} expandiert wird.
%    \begin{macrocode}
    \def#1{}%
    \the\toks@%
    \edef\tud@reserved{%
      \noexpand\endgroup%
      \def\noexpand#1{#1}%
    }%
  \tud@reserved%
}
%    \end{macrocode}
% \end{macro}^^A \tud@replace
%
% \iffalse
%</body|class&doc>
%<*body>
% \fi
%
%
% \subsubsection{Zusätzliche Markup-Befehle}
%
% \begin{macro}{\Attention}
% \changes{v2.02}{2014/08/16}{neu}^^A
% \begin{macro}{\Forum}
% \begin{macro}{\CTAN}
% \changes{v2.05}{2015/11/22}{neu}^^A
% \begin{macro}{\GitHubRepo}
% \changes{v2.02}{2014/08/16}{neu}^^A
% \begin{macro}{\Download}
% \changes{v2.05g}{2016/11/08}{neu}^^A
% \begin{macro}{\notudscrartcl}
% \begin{macro}{\scrguide}
% \changes{v2.02}{2014/09/04}{neu}^^A
% \changes{v2.05}{2015/07/23}{Hyperlinktext über optionales Argument}^^A
% Für die Anwenderdokumentation werden weitere Auszeichnungsbefehle definiert.
%    \begin{macrocode}
\newcommand*\Attention[2][\z@]{%
  \marginnote{%
    \setlength\fboxsep{0.25em}%
    \fbox{Achtung!}%
  }[#1]%
  \emph{\trim@spaces{#2}}%
}
\NewDocumentCommand\Forum{!s !t'}{%
  \IfBooleanTF{#1}{\toks@{\href}}{\toks@{\hrfn}}%
  \IfBooleanT{#2}{\toks@{\url}}%
  \addto@hook\toks@{{http://latex.wcms-file3.tu-dresden.de/phpBB3/}}%
  \IfBooleanF{#2}{\addto@hook\toks@{{TUD-\LaTeX-Forum}}}%
  \the\toks@\xspace%
}
\NewDocumentCommand\CTAN{!s !o !g}{%
  \IfBooleanTF{#1}{\toks@{\href}}{\toks@{\hrfn}}%
  \addto@hook\toks@{{http://www.ctan.org/\IfValueT{#2}{#2}}}%
  \addto@hook\toks@{%
    {\tud@english{Comprehensive TeX Archive Network (CTAN\IfValueT{#3}{~#3})}}%
  }%
  \the\toks@\xspace%
}
%<*class>
\NewDocumentCommand\GitHubRepo{!s !t' !O{releases}}{%
  \IfBooleanTF{#1}{\toks@{\href}}{\toks@{\hrfn}}%
  \IfBooleanT{#2}{\toks@{\url}}%
  \addto@hook\toks@{{https://github.com/tud-cd/tudscr/#3}}%
  \IfBooleanF{#2}{%
    \addto@hook\toks@{{\tud@english{GitHub-Repository~\Distribution*{tudscr}}}}%
  }%
  \the\toks@\xspace%
}
\newcommand*\Download[1]{%
  https://github.com/tud-cd/tudscr/releases/download/#1%
}
\newcommand*\notudscrartcl{%
  F\"ur die Klassen \Class{tudscrartcl} sowie \Class{tudscrposter} ist diese %
  Option nicht verf\"ugbar.%
}
%</class>
\NewDocumentCommand\scrguide{!s !O{\KOMAScript-Handbuch}}{%
  \IfBooleanTF{#1}{\toks@{\href}}{\toks@{\hrfn}}%
  \addto@hook\toks@{%
    {http://mirrors.ctan.org/macros/latex/contrib/koma-script/doc/scrguide.pdf}%
  }%
  \addto@hook\toks@{{#2}}%
  \the\toks@\xspace%
}
%    \end{macrocode}
% \end{macro}^^A \scrguide
% \end{macro}^^A \notudscrartcl
% \end{macro}^^A \Download
% \end{macro}^^A \GitHubRepo
% \end{macro}^^A \CTAN
% \end{macro}^^A \Forum
% \end{macro}^^A \Attention
%
% \iffalse
%</body>
%<*body|class&doc>
% \fi
%
% \begin{macro}{\CD}
% \begin{macro}{\CDs}
% \begin{macro}{\TUD}
% \begin{macro}{\TnUD}
% \begin{macro}{\TUDCD}
% \changes{v2.05}{2015/11/01}{neu}^^A
% \begin{macro}{\TUDCDs}
% \changes{v2.05}{2015/11/01}{neu}^^A
% \begin{macro}{\DDC}
% Diese Befehle stellen regelmäßig in der Quelltextdokumentatuion und im 
% Handbuch genutzte Textbausteine bereit. Dazu wird der Befehl \cs{xspace} aus 
% dem \pkg{xspace}-Paket genutzt.
%    \begin{macrocode}
\newcommand*\CD{\tud@english{Corporate Design}\xspace}
\newcommand*\CDs{\tud@english{Corporate Designs}\xspace}
\newcommand*\TUD{Technische Universit\"at Dresden\xspace}
\newcommand*\TnUD{Technischen Universit\"at Dresden\xspace}
\newcommand*\TUDCD{\CD der \TnUD}
\newcommand*\TUDCDs{\CDs der \TnUD}
\newrobustcmd*\DDC{%
  \mbox{%
    D\kern.05em R\kern.05em E\kern.05em S\kern.05em %
    D\kern.05em E\kern.05em N\kern.1em-\kern.1em concept%
  }\xspace%
}
\AfterPackage*{hyperref}{%
  \pdfstringdefDisableCommands{%
    \def\DDC{DRESDEN-concept}%
    \def\TUDScript{TUD-Script}%
    \def\KOMAScript{KOMA-Script}%
  }%
}
%    \end{macrocode}
% \end{macro}^^A \DDC
% \end{macro}^^A \TUDCDs
% \end{macro}^^A \TUDCD
% \end{macro}^^A \TnUD
% \end{macro}^^A \TUD
% \end{macro}^^A \CDs
% \end{macro}^^A \CD
% \begin{macro}{\OpenSans}
% \changes{v2.06}{2018/07/02}{neu}^^A
% \begin{macro}{\tud@cdfont@texteb}
% \changes{v2.06}{2018/08/31}{neu}^^A
% \begin{macro}{\Univers}
% \begin{macro}{\DIN}
% Für die Nennung der Schriften spezielle Markos bereitgestellt.
%    \begin{macrocode}
\DeclareTextFontCommand\tud@cdfont@texteb{%
  \fontfamily{\tud@cdfont@fam}\fontseries{\tud@cdfont@ebf}\selectfont%
}
\newrobustcmd*\OpenSans{%
  \texorpdfstring{\tud@cdfont@texteb{Open~Sans}}{Open Sans}\xspace%
}
\newrobustcmd*\Univers{\texorpdfstring{\textcdbi{Univers}}{Univers}\xspace}
\newrobustcmd*\DIN{\texorpdfstring{\textcdxi{DIN~Bold}}{DIN Bold}\xspace}
%    \end{macrocode}
% \end{macro}^^A \DIN
% \end{macro}^^A \Univers
% \end{macro}^^A \tud@cdfont@texteb
% \end{macro}^^A \OpenSans
% \begin{macro}{\@pnumwidth}
% \begin{macro}{\@tocrmarg}
% Für das Inhaltsverzeichnis werden die beiden Längen angepasst.
%    \begin{macrocode}
\renewcommand*\@pnumwidth{2.1em}%
\renewcommand*\@tocrmarg{3.1em}%
%    \end{macrocode}
% \end{macro}^^A \@tocrmarg
% \end{macro}^^A \@pnumwidth
%
% \iffalse
%</body|class&doc>
%<*body>
% \fi
%
% \begin{macro}{\textsbn}
% \changes{v2.02}{2014/08/16}{neu}^^A
% \begin{macro}{\sbnfont}
% \changes{v2.02}{2014/08/16}{neu}^^A
% \begin{macro}{\textsbs}
% \changes{v2.02}{2014/08/16}{neu}^^A
% \begin{macro}{\sbsfont}
% \changes{v2.02}{2014/08/16}{neu}^^A
% Für die Anwenderdokumentation werden weitere Auszeichnungsbefehle definiert.
%    \begin{macrocode}
\newcommand*\textsbn{}
\newrobustcmd*\sbnfont{\sffamily\bfseries\upshape}
\DeclareTextFontCommand\textsbn{\sbnfont}
\newcommand*\textsbs{}
\newrobustcmd*\sbsfont{\sffamily\bfseries\slshape}
\DeclareTextFontCommand\textsbs{\sbsfont}
%    \end{macrocode}
% \end{macro}^^A \sbsfont
% \end{macro}^^A \textsbs
% \end{macro}^^A \sbnfont
% \end{macro}^^A \textsbn
%
% \iffalse
%</body>
% \fi
%
% \Finale
%
\endinput
